\documentclass[dvips,10pt]{book}
%\documentclass[dvips,10pt]{report}

\usepackage{latexsym}
%\usepackage{amstex}
%\usepackage{amssymb}
%\usepackage{amsfonts}
\usepackage{amstext}
\usepackage{amsbsy}
\usepackage{amscd}
\usepackage{makeidx}
\usepackage{graphicx}
%\usepackage{calc}
\usepackage{longtable}

% -- to reduce the margin pages --
\topmargin 2pt
\headsep 20pt
\headheight 2pt

%\textwidth 440pt
%\textheight 690pt
%\oddsidemargin 0pt
%\evensidemargin 0pt

% 19/01/06: EV
\textwidth 490pt
\textheight 700pt
\oddsidemargin -20pt
\evensidemargin -20pt

% --------------------------------

\makeindex

\newcommand{\eyedb}{\textsc{EyeDB} }
\newcommand{\eyedbX}{\textsc{EyeDB}}
\newcommand{\sysra}{SYSRA}
\newcommand{\indexentry}[2]{\item #1 #2}
\newcommand{\chap}[1]{\chapter{#1} \index{#1}}
\newcommand{\sect}[1]{\section{#1} \index{#1}}
\newcommand{\subsect}[1]{\subsection{#1} \index{#1}}
\newcommand{\subsubsect}[1]{\subsubsection{#1} \index{#1}}
%% WARNING \Box does not work with latex2html
%%\newcommand{\iii}{\item[{$\Box$}]}
\newcommand{\rrr}{\item[{$\Box$}]}
\newcommand{\iii}{\item}
\newcommand{\jjj}{\item[-]}
\newcommand{\bi}{\begin{itemize}}
\newcommand{\ei}{\end{itemize}}
\newcommand{\be}{\begin{enumerate}}
\newcommand{\ee}{\end{enumerate}}
\newcommand{\kkk}{\item[ ]}
\newcommand{\sh}[1]{\bf #1}
\newcommand{\ttt}{$\tilde{ }$}
\newcommand{\ttv}[1]{\texttt{#1}}
\newcommand{\ttu}[1]{{\verbsize\texttt{#1}}}
\newcommand{\verbsizex}{\small}
\newcommand{\verbsizey}{\fontsize{8}{\baselineskip}\selectfont}
\newcommand{\verbsize}{\normalsize}
\newcommand{\bugreport}{bug-report@eyedb.org }
\newcommand{\idt}{\mbox{ } - }
\newcommand{\ident}{{\bf \texttt{<}identifier\texttt{>}}}
\newcommand{\grindent}{xxxxxxxxxxxxxxx \= : \kill}
\newcommand{\arr}{\texttt{-$>$}}
\newcommand{\ul}[1]{\underline{#1}}
\newcommand{\uul}[1]{\textit{\texttt{#1}}}
\newcommand{\ixx}{\makebox[3mm]{ }}
\newcommand{\ixy}{\makebox[10mm]{ }}
\newcommand{\ixz}{\\\ixy {- }}

\renewcommand{\normalsize}{\fontsize{9}{11pt}\selectfont}

%\renewcommand{\normalsize}{\fontsize{8}{10pt}\selectfont}

%% The following commands comes from corbaref.tex

\newcommand{\bt}{\begin{tabular}{|p{7cm}|p{7cm}|}}
\newcommand{\et}{\hline \end{tabular} \\ {\vspace{0.6cm}} \\ }

\newcommand{\mt}[1]{\hline \multicolumn{2}{|c|}{\centerline{{\bf interface #1}
   \emph{mapped from} {\bf class #1}}}}

\newcommand{\Desc}[1]{\hline \multicolumn{2}{|l|}{\emph{#1}}}

\newcommand{\mTT}[2]{\hline \multicolumn{2}{|c|}
   {\centerline{{\bf interface #1} \emph{extends} {\bf #2}
   \emph{mapped from} {\bf #1}}}}


\newcommand{\mT}[2]{\hline \multicolumn{2}{|c|}{\centerline{{\bf #1}
\emph{mapped from} {\bf #1}}}}

\newcommand{\att}{\hline \hline {\bf Attribute} & {\bf Description}}
\newcommand{\mtt}{\hline \hline {\bf Method} & {\bf Description}}
\newcommand{\mf}{\emph{mapped from }}

\newcommand{\satt}[1]{\hline \multicolumn{2}{|c|}
   {\centerline{\emph{attributes inherited from} {\bf #1}}}}

\newcommand{\smtt}[1]{\hline \multicolumn{2}{|c|}{
   \centerline{\emph{methods inherited from} {\bf #1}}}}

\newcommand{\sets}[1]{sets the #1 \emph{value} in the Object \emph{o} at the \emph{ind}}

\newcommand{\Gets}[1]{gets the #1 value in the Object \emph{o} at the \emph{ind}}

%\setcounter{secnumdepth}{-1}
%\addtocounter{secnumdepth}{2}

\renewcommand{\thesection}{\arabic{section}}
\renewcommand{\thesubsection}{\arabic{section}.\arabic{subsection}}
\renewcommand{\thechapter}{}

\begin{document}
\normalsize


\input{version}
\newcommand{\mantitle}{\textsc{C++ Binding} }

\thispagestyle{empty}

\mbox{ }
\\
\vspace{3 cm}
\\
%{\bf{\Huge E{\LARGE YE}DB} \hspace{0.cm} {\LARGE Programming Manual}}
{\bf{\Huge E{\LARGE YE}DB} \hspace{0.cm} {\Huge \mantitle}}
\\
%\newcommand{\rulewidth}{12.2cm}
\newcommand{\rulewidth}{15.4cm}
\rule{\rulewidth}{1.6mm}
\begin{flushright}
{\large Version \eyedbversion}
\end{flushright}
\mbox{ }
\\
\vspace{11 cm}
\\
\begin{flushright}
{\large January 2006}
\end{flushright}

\newpage

\thispagestyle{empty}

\mbox{ }
\\
\vspace{1cm}
\\
Copyright {\copyright} 2001-2006 SYSRA
\\
\vspace{1cm}
\\
Published by \sysra
\\
30, avenue G\'en\'eral Leclerc
\\
91330 Yerres - France
\\
\\
home page: http://www.eyedb.org
\\
\\
bug report: \bugreport


\tableofcontents

\chapter*{The EyeDB C++ Binding}

%%\eyedb provides currently two language bindings: C++ and Java.
%%\\
%%The C++ binding is the native language binding as \eyedb is fully written
%%in C++.
The C++ binding maps the \eyedb object model into C++ by introducing
a generic API
%%The C++ binding is expressed as a generic API
%%including about one hundred of classes,
and a tool to generate a speficic C++ API from a given schema,
built upon the generic API.
\\
\\
The generic C++ API is made up of about one hundred of classes such
as some abstract classes as the \texttt{object} and \texttt{class} classes
and some more concrete classes such as the \texttt{database} and \texttt{image}
classes.
\\
\\
Each type in the \eyedb object model is implemented as a C++ class whithin
the C++ API: there is a one for one mapping between the object model and
the C++ API.
\\
This mapping follows a very simple naming scheme: each C++ class mapped from
a type has the name of this type prefixed by the namespace \texttt{eyedb}.
\\
For instance, the \texttt{object} type in the \eyedb object model is
mapped to the \texttt{eyedb::Object} C++ class and the \texttt{agregat}
type is mapped to the \texttt{eyedb::Agregat} C++ class.
\\
\\
To avoid writing each time the full qualified type name
(i.e. \texttt{eyedb::}\emph{type}), you may use the C++ instruction
\texttt{using namespace eyedb}.
\\
\\
We are going to introduce the main classes and methods through some simple
examples.
\sect{The Generic C++ API}
\subsect{Initialization}
The minimal \eyedb C++ program is as follows:
\verbsize
\begin{verbatim}
#include <eyedb/eyedb.h>

int
main(int argc, char *argv[])
{
  eyedb::init(argc, argv);
  // ...
  eyedb::release();
  return 0;
}
\end{verbatim}
\normalsize
A few remarks about this code:
\be
\item the file \texttt{eyedb/eyedb.h} contains the whole \eyedb C++ API; except
for some specific administration or hacker tasks, it is not necessary
to include any other eyedb files.
\item the \eyedb C++ layer must be initialized using one of the static method
\texttt{init} of the namespace \texttt{eyedb}:
\be
\item static void eyedb::init()
\item static void eyedb::init(int \&argc, char *argv[])
\ee
The first method only initializes the \eyedb layer while the second
one performs also some command line option processing.
For instance, the option \texttt{--port=<port>} allows you to use
the port \texttt{<port>} as the default connection port to the \eyedb
server, while the option \texttt{-logdev=stderr} displays log information
on the standard error.
\\
\\
The option \texttt{--help-eyedb-options} displays a brief usage for each standard options:
\verbsize
\begin{verbatim}
  -U <user>|@, --user=<user>|@       User name
  -P [<passwd>], --passwd[=<passwd>] Password
  --host=<host>                      eyedbd host
  --port=<port>                      eyedbd port
  --inet                             Use the tcp_port variable if port is not set
  --dbm=<dbmfile>                    EYEDBDBM database file
  --conf=<conffile>                  Configuration file
  --logdev=<logfile>                 Output log file
  --logmask=<mask>                   Output log mask
  --logdate=on|off                   Control date display in output log
  --logtimer=on|off                  Control timer display in output log
  --logpid=on|off                    Control pid display in output log
  --logprog=on|off                   Control progname display in output log
  --error-policy=<value>             Control error policy: status|exception|abort|stop|echo
  --trans-def-mag=<magorder>         Default transaction magnitude order
  --arch                             Display the client architecture
  -v, --version                      Display the version
  --help-eyedb-options               Display this message
\end{verbatim}
\normalsize
Note that all the standard command line options recognized in the
\texttt{argc/argv} array are suppressed from this array by
\texttt{eyedb::init(int \&argc, char *argv[])}.
\item the last statement \texttt{eyedb::release()} allows you to release all
the \eyedb allocated resources and to close opened databases and connections.
Note that this statement is optionnal as all \eyedb allocated resources,
opened databases and connections will be automatically released or closed
in the \texttt{exit()} function.
\ee
\subsect{Connection Setting-up}
To manage objects within a database we need to open this database.
But before opening any database we need to establish a connection with
the \eyedb server.
\\
The connection to the \eyedb server is realized through the
\texttt{eyedb::Connection} class as follows:
\verbsize
\begin{verbatim}
  eyedb::Connection conn;
  conn.open();
\end{verbatim}
\normalsize
A few remarks about this code:
\be
\item the construction of an \texttt{eyedb::Connection} instance (first line of code)
does not perform any actual actions: it only constructs a runtime instance.
\item to establish the connection, one needs to use the
\texttt{eyedb::Connection::open(const char *host=0, const char *port=0)} method.
This method has two optionnal arguments: \texttt{host} and \texttt{port}.
\\
If these arguments are not specified, their values are taken from the
configuration or
from the command line options \texttt{--host=<host>} and
\texttt{--port=<port>} if specified.
\item in case of an error happened during the connection setting-up,
a status is returned or an exception is raised depending on the
chosen error policy. The default error policy is the \texttt{status error
policy} which means that each \eyedb method returns a status
implemented by the \texttt{eyedb::Status} class. The special status
\texttt{eyedb::Success} (in fact a null pointer) means that the operation
has been performed successfully:
\bi
\item
\verbsize
\begin{verbatim}
  eyedb::Status s;
  eyedb::Connection conn;
  s = conn.open();
  if (s) {
    cerr << status;
    return 1;
  }
\end{verbatim}
\normalsize
\ei
The \texttt{exception error policy} means that each \eyedb method throws
an exception, implemented by the class \texttt{eyedb::Exception}, when an error
happened:
\bi
\item
\verbsize
\begin{verbatim}
  try {
    eyedb::Connection conn;
    conn.open();
  }
  catch(eyedb::Exception &e) {
    cerr << e;
    return 1;
  }
\end{verbatim}
\normalsize
\ei
Note that \texttt{eyedb::Status} is an alias for \texttt{const eyedb::Exception *}.
To use the \texttt{exception error policy}, one needs to call the following
method before any operation:
\verbsize
\begin{verbatim}
  eyedb::Exception::setMode(eyedb::Exception::ExceptionMode);
\end{verbatim}
\normalsize
Although the exception error policy is not currently the default one in
\eyedbX, we recommend to use it: it makes code clearer and safer.
\\
In the following examples we use the \texttt{exception error policy}
to avoid any error management noise in the introduced C++ code.
\ee
\subsect{Database Opening}
To open a database one uses the \texttt{eyedb::Database} class as follows:
\verbsize
\begin{verbatim}
  const char *dbname = argv[1];
  eyedb::Database db(dbname);
  db.open(&conn, eyedb::Database::DBRW);
\end{verbatim}
\normalsize
\be
\item as the \texttt{eyedb::Connection} constructor, the \texttt{eyedb::Database}
constructor does not perform any actual operation: it constructs
a runtime instance.
\item to open a database one uses the \texttt{eyedb::Database::open} methods
which takes the following arguments:
\be
\item a pointer to an opened \texttt{eyedb::Connection} instance.
\item the opening flag which can be either \texttt{eyedb::Database::DBRead} for
read-only opening or \texttt{eyedb::Database::DBRW} for read-write opening.
\\
Note that there are a dozen of opening modes that are introduced in
the reference manual.
\item the user authentication
\item the password authentication
\ee
The two last arguments are optionnal: if not specified, their values
are taken from the configuration file or from the command line options
\texttt{--user=<user>} and \texttt{--passwd=<passwd>}, or from
the standard input when using \texttt{--passwd} without given value.
\ee
Note that an \eyedb client can manage several connections and
several databases on each connection, for instance:
\verbsize
\begin{verbatim}
  eyedb::Connection conn_local;
  conn_local.open();
  eyedb::Connection conn_remote;
  conn_remote.open("arzal.zoo.com", 7620);

  eyedb::Database db_1("foo");
  db_1.open(&conn_local, eyedb::Database::DBRW);

  eyedb::Database db_2("EYEDBDBM");
  db_2.open(&conn_local, eyedb::Database::DBRead, "guest", "guest");

  eyedb::Database db_3("droopy");
  db_2.open(&conn_remote, eyedb::Database::DBRW, "droopy", "xyztu");
\end{verbatim}
\normalsize
\subsect{Transaction Management}
Any object operation - storing or loading for instance -
within a database must be done in the scope of a transaction.
\\
\\
A transaction is an unit with atomicity, coherency and integrity.
\be
\item {\bf Atomicity} means that the transaction modifications are either
realized (commit) or not realized at all (rollback or abort).
\item {\bf Coherency} means that a transaction starts from a coherent database
state, and leaves the database in a coherent state.
\item {\bf Integrity} means that a transaction modification is not lost, even
in case of a process, operating system or hardware failure.
\ee
A transaction scope is composed of a starting point, \texttt{transactionBegin},
and an ending point, \texttt{transactionCommit} or \texttt{transactionAbort}:
\verbsize
\begin{verbatim}
  eyedb::Database db(dbname);
  db.open(&conn, eyedb::Database::DBRW);

  db.transactionBegin();
  // ... object operations
  db.transactionCommit();
\end{verbatim}
\normalsize
A call to \texttt{eyedb::Database::transactionCommit()} means that all the
operations performed in the transaction scope will be stored in
the database, while a call to \texttt{eyedb::Database::transactionAbort()} means
that all the operations will be forgotten.
\\
\\
Currently, \eyedb does not support nested transactions but it allows you
to write code such as:
\verbsize
\begin{verbatim}
  db.transactionBegin();     // level 0 begin
  // ... object operations
     db.transactionBegin();  // level 1 begin
     // ... object operations
     db.transactionAbort();  // level 1 abort
  // ... object operations
  db.transactionCommit();    // level 0 commit
\end{verbatim}
\normalsize
But the abort at level 1 is without effect: it will
not be performed; only the commit at level 0 will be performed.
\\
\\
One can give parameters to the transaction that one begins by setting an
optional argument of type \texttt{eyedb::TransactionParams} to the
\texttt{transactionBegin} method. The \texttt{TransactionParams} type is
composed of the following public attributes :
\be
\item the \texttt{trsmode} argument controls the transaction mode,
\item the \texttt{lockmode} argument controls the object lock policy,
\item the \texttt{recovmode} argument controls the recovery mode,
\item the \texttt{magorder} argument controls the size of the allocated
tables for the transaction,
\item the \texttt{ratioalrt} argument controls the  error returned if
ratioalrt != 0 and trans object number > ratioalrt * magorder
\item the \texttt{wait\_timeout} argument controls 
wait timeout value.
\ee
For instance :
\begin{verbatim}
 TransactionParams params; // create params with default values
 params.lockmode = eyedb::ReadNWriteX; // objects are not locked for reading
                                       // and locked exclusive for writing
 params.magorder = 100000000; // transaction can deal with about
                              // 100 millions of objects
 db.transactionBegin(params);
\end{verbatim}
Refer to the reference manual to get more information about these arguments.

\subsect{Schema and Class Manipulation}
The \eyedb C++ API provides runtime facilities to manipulate the \eyedb
classes. In fact, as the class \texttt{class} inherits from the
class \texttt{object}, \eyedb classes can be manipulated as objects.
\\
\\
A class is composed of a list of attributes, constraints,
variables, methods, triggers and indexes.
\\
\\
The classes are gathered through a schema instance tied to
each database.
\\
\\
A class can be a system class, for instance the class \texttt{class}, the
class \texttt{object}, the class \texttt{agregat} or a user class, for
instance the class \texttt{Person}, the class \texttt{Employee}.
\\
\\
To illustrate this object model, we are going to show how to display the
user classes of a given database:
\verbsize
\begin{verbatim}
  eyedb::Database db(dbname);
  db.open(&conn, eyedb::Database::DBRW);

  db.transactionBegin();
  eyedb::LinkedListCursor c(db.getSchema()->getClassList());
  eyedb::Class *cls;
  while (c.getNext((void*&)cls))
    if (!cls->isSystem())
      cout << cls;
  db.transactionCommit();
\end{verbatim}
\normalsize
As shown here, this code is very simple:
\be
\item database opening as we have seen before.
\item linked list cursor creation on the database schema class list.
\item display of each class in the list which is not a system class.
\ee
For instance, to display all the classes of type \texttt{struct} which contains
an attribute named \texttt{age}:
\verbsize
\begin{verbatim}
  eyedb::LinkedListCursor c(db.getSchema->getClassList());
  eyedb::Class *cls;
  while (c.getNext((void*&)cls))
    if (cls->asStructClass()) {
      int attr_cnt;
      const eyedb::Attribute **attrs = cls->getAttributes(attr_cnt);
      for (int i = 0; i < attr_cnt; i++)
        if (!strcmp(attrs[i]->getName(), "age")) {
          cout << cls;
          break;
       }
     }
\end{verbatim}
\normalsize
\subsect{Object Manipulation}
There are two types of objects: runtime objects and database objects.
\\
Runtime objects are the OML (Object Manipulation Language) objects,
for instance C++ or Java objects, while the database objects are
the objects stored in a database.
\\
\\
There are two types of runtime objects: persistent runtime objects and
transient runtime objects.
\\
A runtime object is persistent if it is tied to a database object.
Otherwise, it is  transient.
\\
\\
Contrary to some other OODBMS, \eyedb does not
provide an automatic synchronisation between persistent runtime objects
and database objects.
\\
When setting values on a persistent runtime object, we do not modify
the tied database object.
We must call the \texttt{store} method on the persistent runtime object
to update the tied database object.
\\
\\
Note that any persistent runtime object manipulation must be done
in the scope of a transaction.
\\
\\
To illustrate object manipulations, we introduce a simple concrete
example.
This example will be used in the whole continuation of this chapter.
\\
\\
The example is as follows:
\verbsize
\begin{verbatim}
//
// person.odl
//

enum CivilState {
  Lady = 0x10,
  Sir  = 0x20,
  Miss = 0x40
};

class Address {
  attribute string street;
  attribute string<32> town;
  attribute string country;
};

class Person {
  attribute string name;
  attribute int age;
  attribute Address addr;
  attribute Address other_addrs[];
  attribute CivilState cstate;
  attribute Person * spouse inverse Person::spouse;
  attribute set<Car *> cars inverse owner;
  attribute array<Person *> children;

  int change_address(in string street, in string town,
                     out string oldstreet, out string oldtown);

  static int getPersonCount();
  index on name;
};

class Car {
  attribute string brand;
  attribute int num;
  Person *owner inverse cars;
};

class Employee extends Person {
  attribute long salary;
};
\end{verbatim}
\normalsize
This file is located at \emph{prefix}/\texttt{share/doc/eyedb/examples/C++Binding/schema-oriented/share/schema.odl}.
\subsect{Creating Runtime Objects}
Using the C++ API, we cannot create directly a database object.
We must create first a runtime object.
\\
To create a runtime object we invoke the \texttt{newObj} method
of the object class.
\\
For instance, to create a runtime \texttt{Person} object, we need
to invoke the \texttt{newObj} method of the \texttt{Person} runtime class
as follows:
\verbsize
\begin{verbatim}
  eyedb::Class *cls = db.getSchema()->getClass("Person");

  eyedb::Object *p = cls->newObj(&db);
\end{verbatim}
\normalsize
The \texttt{eyedb::Class::newObj(eyedb::Database * = 0)} is the class instantiation
method for both persistent and transient object.
\\
A transient object is created using the \texttt{newObj} without any argument,
while a persistent object is created using the same method with
a valid database runtime pointer.
\\
\\
Note that as long as the \texttt{store} method has not been called,
the persistent runtime object is not yet tied to a database object.
\\
So, if we follow strictly the definition of runtime objects,
it is not yet a persistent runtime object; but as soon as a runtime
object is created using the \texttt{newObj} method with a valid database
pointer, we will say that it is persistent.
\subsect{Synchronizing Runtime Objects to Database Objects}
When a persistent object is stored in the database using the
\texttt{store} method, an unique object identifier \texttt{OID} is allocated
to this object.
\\
This \texttt{OID} can be acceded using the method \texttt{eyedb::Object::getOid()},
for instance to display the allocated \texttt{OID}:
\verbsize
\begin{verbatim}
  eyedb::Object *p = cls->newObj(&db);
  cout << "before storing: " << p->getOid() << endl;
  p->store();
  cout << "after storing: " << p->getOid() << endl;
\end{verbatim}
\normalsize
The output displayed by the previous code is something as follows:
\verbsize
\begin{verbatim}
before storing: NULL
after storing: 1456.3.38475637:oid
\end{verbatim}
\normalsize
As shown here, before the first call of the \texttt{store} method, the \texttt{OID} is
not set; a \texttt{NULL} is displayed.
\\
The created \texttt{OID} is composed of three fields:
\be
\item the object number : \texttt{1456}
\item the database identifier : \texttt{3}
\item a magic number : \texttt{38475637}
\ee
The database identifier designates, in an unique way, a database while
the object number designates, in an unique way, an object within a database.
\\
The magic number, which is a random generated number, ensures more security
in the object identification process.
\subsect{Setting Attribute Values to a Runtime Object}
Assume that we want to set a name and a age values to a \texttt{Person} instance.
Here is a way to do so:
\verbsize
\begin{verbatim}
  eyedb::Class *cls = db.getSchema()->getClass("Person");
  eyedb::Object *p = cls->newObj(&db);

  // getting attributes from class
  const eyedb::Attribute *attr_name = cls->getAttribute("name");
  const eyedb::Attribute *attr_age = cls->getAttribute("age");

  // setting name attribute value
  attr_name->setSize(p, strlen("john")+1);
  attr_name->setValue(p, (eyedb::Data)"john", strlen("john")+1, 0);

  // setting age attribute value
  eyedb::_int32 age = 27;
  attr_age->setValue(p, (eyedb::Data)&age, 1, 0);
\end{verbatim}
\normalsize
We need to do a few remarks about this code:
\be
\item to get specific named attribute within a class,
we use the method \texttt{eyedb::Class::getAttribute(const char *)}.
\\
This method returns a pointer to an \texttt{eyedb::Attribute} which contains
a complete description of this attribute: type, name, size, position and
so on.
\item to set an attribute value for the instance \texttt{p}, we
use the method \texttt{eyedb::Attribute::setValue(eyedb::Object *o, eyedb::Data data,
int nb, int from)} whose arguments are as follows:
\be
\item \texttt{eyedb::Object *o}: the runtime object pointer to modify.
\item \texttt{eyedb::Data data}: the pointer to the attribute value to set.
\item \texttt{int nb}: for an array, the number of values to set.
\item \texttt{int from}: for an array, the starting index of the values to set.
\ee
\item the \texttt{eyedb::Attribute::setSize(eyedb::Object *, eyedb::Size)} method is
used for the attribute \texttt{name} because this attribute is of variable
size (remember the schema description : \texttt{string name}).
\\
So, before setting the attribute value, we must set the size of this
attribute value.
\item remember that the database object tied to this persistent object has not
been changed in the database: only the transient values have been changed.
\\
To change the database object, one needs to use the method
\texttt{eyedb::Object::store()} as follows:
\verbsize
\begin{verbatim}
  p->store();
\end{verbatim}
\normalsize
The \texttt{store} method allows you to synchonize the transient values
of a persistent object with the database.
\\
\\
To avoid all this class and attribute manipulation and to deal with
direct access attribute methods, one needs to use the \texttt{eyedbodl}
tool which allows you to generate specific C++ code from a specific database
schema.
\\
\\
For instance, using this tool, the previous code becomes:
\verbsize
\begin{verbatim}
  Person *p = new Person(&db);

  p->setName("john");
  p->setAge(27);

  p->store();
\end{verbatim}
\normalsize
The class \texttt{Person}, the methods \texttt{setName} and \texttt{setAge}
have been generated by the \texttt{eyedbodl} tool in a very simple way.
\\
Refer to the second part of this chapter \texttt{the Schema-Oriented
Generated C++ API}.
\ee
\subsect{Loading Database Objects}
To load an object from a database, one needs to give its \texttt{OID} to
the \texttt{eyedb::Database::loadObject} method, for instance:
\verbsize
\begin{verbatim}
  eyedb::Oid oid("1456.3.38475637:oid");
  eyedb::Object *o;
  db.loadObject(oid, o);
  cout << "object " << oid << " is of class " << o->getClass()->getName()
       << endl;
  cout << o;
\end{verbatim}
\normalsize
The previous code loads the object from the database, displays its
oid and class name and displays the whole object.
\subsect{Getting Attribute Values from a Runtime Object}
The process to get attribute values from a runtime object is very similar
to the process to set attribute values.
\\
For instance to get the \texttt{name} and \texttt{age} attribute values of
the previous loaded object, one can do as follows:
\verbsize
\begin{verbatim}
  eyedb::Oid oid("1456.3.38475637:oid");
  eyedb::Object *o;
  db.loadObject(oid, o);

  // getting attributes from class
  const eyedb::Attribute *attr_name = cls->getAttribute("name");
  const eyedb::Attribute *attr_age = cls->getAttribute("age");

  // getting name attribute size
  eyedb::Size name_length;
  attr_name->getSize(o, name_length);

  // getting name attribute value
  char *name = new char[name_length];
  attr_name->getValue(o, (eyedb::Data *)name, name_length, 0);
  cout << "name is : " << name << endl;
  delete [] name;

  // getting age attribute value
  eyedb::_int32 age;
  attr_age->getValue(o, (eyedb::Data *)&age, 1, 0);
  cout << "age is : " << age << endl;
\end{verbatim}
\normalsize
To get an attribute value we use the method
\texttt{eyedb::Attribute::getValue(const eyedb::Object *o, eyedb::Data *data,
int nb, int from, eyedb::Bool * isnull = 0)} whose arguments are as follows:
\be
\item \texttt{eyedb::Object *o}: the runtime object pointer.
\item \texttt{eyedb::Data data}: the pointer to the attribute value to get: this
pointer must be allocated correctly according to the returned value type.
It is why we get first the size of the \texttt{name} attribute value to allocate
the returned buffer with a valid size.
\item \texttt{int nb}: for an array, the number of values to get.
\item \texttt{int from}: for an array, the starting index of the values to get.
\item \texttt{eyedb::Bool *isnull}: an optionnal boolean to check if the attribute
value is null (i.e. not initialized).
\ee
If we want to get the \texttt{spouse} value of the loaded person,
we must do something a little bit more complicated:
\verbsize
\begin{verbatim}
  eyedb::Oid oid("1456.3.38475637:oid");
  eyedb::Object *o;
  db.loadObject(oid, o);

  // getting spouse attribute from class
  const eyedb::Attribute *attr_spouse = cls->getAttribute("spouse");

  eyedb::Oid spouse_oid;
  attr_spouse->getOid(o, &spouse_oid);

  if (spouse_oid.isValid()) {
    eyedb::Object *spouse;
    db.loadObject(spouse_oid, spouse);
    cout << "spouse is: " << spouse;
  }
\end{verbatim}
\normalsize
To get the \texttt{spouse} attribute value, we need to get first the
\texttt{spouse OID} using the \texttt{eyedb::Attribute::getOid} method on
the \texttt{spouse} attribute.
\\
Then, if the \texttt{OID} is valid, we load the \texttt{spouse} from the
found \texttt{OID}.
\\
\\
Once again, using the \texttt{eyedbodl} tool, all the previous code
becomes very simple:
\verbsize
\begin{verbatim}
  eyedb::Oid oid("1456.3.38475637:oid");
  eyedb::Object *o;
  db.loadObject(oid, o);
  Person *p = Person_c(o);

  cout << "name is : " << p->getName() << endl;
  cout << "age is : " << p->getAge() << endl;

  cout << "spouse is: " << p->getSpouse();
\end{verbatim}
\normalsize
\subsect{Loading Database Objects using OQL}
We have seen in the previous section how to load a database object from its
\texttt{OID}. The problem is that the \texttt{OID} is a rather
hidden concept and there are very few chances to know an object \texttt{OID}
before having loaded it.
\\
\\
To load database objects it seems more reasonnable to use a query language
such as OQL.
\\
The \eyedb C++ API allows you to perform any OQL queries using the class
\texttt{eyedb::OQL}.
\\
For instance to get all Person whose age is less than a given value:
\verbsize
\begin{verbatim}
  db.transactionBegin();
  eyedb::OQL q(&db, "select Person.age < %d", given_age);

  eyedb::ObjectArray obj_arr(eyedb::True);
  q.execute(obj_arr);
  for (int i = 0; i < obj_arr.getCount(); i++)
    cout << obj_arr[i];
\end{verbatim}
\normalsize
A few remarks about this code:
\be
\item remember that any persistent runtime object manipulation must be done
in the scope of a transaction: it is why the first statement is a call
to the \texttt{transactionBegin} method. In most of the previous code examples,
we volontary omit this call.
\item the class \texttt{eyedb::OQL} is used to perform
any OQL query. The main constructor \texttt{eyedb::OQL(eyedb::Database *db, const char *fmt, ...)} allows you to make an OQL query in a simple way.
The arguments are as follows:
\be
\item the database pointer within which to perform the query.
\item the format of the query in a \texttt{sprintf} style.
\item the other arguments are the arguments related to the previous
format.
\ee
\item to get all the objects returned by the query, we use the
\texttt{eyedb::OQL::execute(eyedb::ObjectArray \&)} method. This method filled
the object array reference given as input parameter.
\item the method \texttt{eyedb::ObjectArray::getCount()} returned the number
of objects contained in an object array.
\item the \texttt{[] eyedb::ObjectArray} operator has been overloaded to
allow you to perform direct access to the contained objects: \texttt{obj\_arr[i]} is
the object \#\texttt{i} within the object array.
\item the argument \texttt{eyedb::True} to the \texttt{eyedb::ObjectArray} constructor
means that we want that all the contained objects to be deleted
when this object array will be deleted.
\ee
Sometimes we want to perform a query to get only a part of some objects.
\\
For instance, to get the name of all persons whose age is less than
a given value, there are two ways:
\be
\item the first one is to get all the persons whose age is less
than the given value using an OQL query, and then get their name value
as follows:
\verbsize
\begin{verbatim}
  eyedb::OQL q(&db, "select Person.age < %d", given_age);
  eyedb::ObjectArray obj_arr(eyedb::True);
  q.execute(obj_arr);

  const eyedb::Attribute *attr_name = cls->getAttribute("name");
  for (int i = 0; i < obj_arr.getCount(); i++) {
     // getting name attribute size
    eyedb::Size name_length;
    attr_name->getSize(obj_arr[i], name_length);

    // getting name attribute value
    char *name = new char[name_length];
    attr_name->getValue(obj_arr[i], (eyedb::Data *)name, name_length, 0);
    cout << "name of #" << i << " is : " << name << endl;
    delete [] name;
  }  
\end{verbatim}
\normalsize
\item the second one is to perform directly an appropriate query as follows:
\verbsize
\begin{verbatim}
  eyedb::OQL q(&db, "(select Person.age < %d).name", given_age);
  eyedb::ValueArray val_arr;

  q.execute(val_arr);
  for (int i = 0; i < val_arr.getCount(); i++)
    cout << "name of #" << i << " is : " << val_arr[i].str << endl;
\end{verbatim}
\normalsize
In this case, the returned value are not object values but string values.
So we cannot use the \texttt{execute(eyedb::ObjectArray\&)} method to get
these values but the more general form \texttt{execute(eyedb::ValueArray\&)}
\\
\\
An \texttt{eyedb::ValueArray} instance is an array of \texttt{eyedb::Value} instances.
The \texttt{eyedb::Value} class is the most general form of an OQL returned value.
It can take the form of a integer, a string, an \texttt{OID}, an object and
so on.
\\
\\
Note that this second way is more efficient as only the person
name are returned from the server and not the full object.
\ee

\subsect{Releasing Runtime Objects}
All the runtime objects which have been allocated by the client code
or by a load or query method must be released by the client code.
\\
\\
To release an \texttt{eyedb::Object} or inherited class instance, we must use
the \texttt{eyedb::Object::release()} method as follows:
\verbsize
\begin{verbatim}
  eyedb::Object *o1 = cls->newObj();
  // ...
  o1->release();

  eyedb::Object *o2;
  db.loadObject(oid, o2);
  // ...
  o2->release();
\end{verbatim}
\normalsize
The C++ \texttt{delete} operator is forbiden: if you try to use this operator
on any \texttt{eyedb::Object} instance, you will get an error message at runtime.
\\
\\
Note that if you release a persistent runtime object you do not release
the tied database object.
\\
\\
Refer to the section \texttt{Memory Management} to understand the whole
memory policy of the C++ API.
\subsect{Removing Database Objects}
To remove a database object, we need to use the
\texttt{eyedb::Object::remove()} method or the \texttt{eyedb::Database::removeObject(const eyedb::Oid \&oid)} method, for instance:
\verbsize
\begin{verbatim}
  db.transactionBegin();

  o->remove();
  o->release();

  db.transactionCommit();
\end{verbatim}
\normalsize
or:
\verbsize
\begin{verbatim}
  db.transactionBegin();

  db.removeObject(oid);

  db.transactionCommit();
\end{verbatim}
\normalsize
When calling one of the previous remove methods, it is not necessary
to call the store method after.

\sect{The Schema-Oriented Generated C++ API}
The generic C++ API allows you to manipulate any object within any
database: this is its force. But, as shown in the previous section,
object manipulation is sometimes very heavy as the provided methods
are too much generic.
\\
\\
To enrich the generic API, one introduces a tool to generate specific
C++ code from a specific ODL schema: the generated API
is call a schema-oriented API.
\\
\\
The schema-oriented API contains mainly:
\be
\item a C++ class for each class defined in the ODL schema.
\item selector and modifier methods in the C++ class
for each attribute defined in the ODL class.
\item user friendly selector and modifier methods for array and
collection attributes.
\item a C++ method mapped on each method defined in the ODL class.
\item a specific C++ database class used to open a database and
check its schema.
\item some utilities such as down-casting funtions.
\ee
The schema-oriented API is designed so that the object manipulation
for this schema is the most comfortable as possible.

\subsect{Generating a Schema-Oriented C++ API}
To generate a schema-oriented C++ API, one needs a well formed ODL
file describing a schema or a reachable database containing this schema
and the \texttt{eyedbodl} tool.
\\
To generate a schema-oriented C++ API, the minimal \texttt{eyedbodl} invocation
is as follows:
\begin{verbatim}
eyedbodl --gencode=C++ <odlfile>
\end{verbatim}
or
\begin{verbatim}
eyedbodl --gencode=C++ --package=<package> -d <database>
\end{verbatim}
For instance, to generate the schema-oriented C++ API for the \texttt{person.odl}
schema:
\begin{verbatim}
eyedbodl --gencode=C++ person.odl
\end{verbatim}
For a given \emph{package}.odl ODL file, the generated files are as follows:
\bi
\item \emph{package}.h, \emph{package}.cc: the generated C++ API to be used in
a client program
\item template\_\emph{package}.cc: an example of a client program using the generated
API
\item Makefile.\emph{package}: an example of Makefile to compile \emph{package}.cc and
template\_\emph{package}.cc: make -f Makefile.\emph{package} will compile
and link the generated API and template files
\item \emph{package}stubsfe.cc, \emph{package}stubsbe.cc: stubs for client and server methods
\item \emph{package}mthfe-skel.cc, \emph{package}mthbe-skel.cc: skeletons for
client and server methods
\ei
The \texttt{eyedbodl} tool contains a lot of command line options to
control the generated code.
\\
\\
There is one mandatory option:\\
\emph{odlfile}\texttt{|}-\texttt{|}\texttt{-d} \emph{dbname}\texttt{|}\texttt{--database=}\emph{dbname} : Input ODL file (or - for standard input) or the database name\\
and some optionnal options:
\begin{tabbing}
\texttt{--gencode-error-policy=exception} \= : \= Exception oriented error policy\kill
\texttt{--package=}\emph{package}          \> : \> Package name\\
\texttt{--output-dir=}\emph{dirname}       \> : \> Output directory for generated files\\
\texttt{--output-file-prefix=}\emph{prefix}\> : \> Ouput file prefix (default is the package name)\\
\texttt{--class-prefix=}\emph{prefix}     \> : \>  Prefix to be put at the begining of each runtime class\\
\texttt{--db-class-prefix=}\emph{prefix}  \> : \>  Prefix to be put at the begining of each database class\\
\texttt{--attr-style=}implicit       \> : \>  Attribute methods have the attribute name\\
\texttt{--attr-style=}explicit       \> : \>  Attribute methods have the attribute name prefixed by get/set (default)\\
\texttt{--schema-name=}\emph{schname}     \> : \>  Schema name (default is \emph{package})\\
\texttt{--export}                    \> : \>  Export class instances in the .h file\\
\texttt{--dynamic-attr}              \> : \>  Uses a dynamic fetch for attributes in the get and set methods\\
\texttt{--down-casting=yes}          \> : \>  Generates the down casting methods (the default)\\
\texttt{--down-casting=no}           \> : \>  Does not generate the down casting methods\\
\texttt{--attr-cache=yes}            \> : \>  Use a second level cache for attribute value\\
\texttt{--attr-cache=no}             \> : \>  Does not use a second level cache for attribute value (the default)\\
\texttt{--namespace=}\emph{namespace}     \> : \>  Define classes with the namespace \emph{namespace}\\
\texttt{--c-suffix=}\emph{suffix}         \> : \>  Use \emph{suffix} as the C file suffix\\
\texttt{--h-suffix=}\emph{suffix}         \> : \>  Use \emph{suffix} as the H file suffix\\
\texttt{--gen-class-stubs}           \> : \>  Generates a file class\_stubs.h for each class\\
\texttt{--class-enums=yes}           \> : \>  Generates enums within a class\\
\texttt{--class-enums=no}            \> : \>  Do not generate enums within a class (default)\\
\texttt{--gencode-error-policy=status}\> : \> Status oriented error policy (the default)\\
\texttt{--gencode-error-policy=exception} \> : \> Exception oriented error policy\\
\texttt{--rootclass=}\emph{rootclass}     \> : \>  Use \emph{rootclass} name for the root class instead of the package name\\
\texttt{--no-rootclass}           \> : \>     Does not use any root class\\
\end{tabbing}
For instance to generate a schema-oriented C++ API in the directory
\texttt{tmp}, prefixing the runtime classes with \texttt{pp},
suffixing C++ files with \texttt{.cpp}, we invoke \texttt{eyedbodl} as follows:
\begin{verbatim}
eyedbodl --gencode C++ --output-dir=tmp --class-prefix=pp \
         --c-suffix=.cpp person.odl
\end{verbatim}
\subsect{The Generated Code}
Seven files are generated:
\be
\item the header C++ file: \emph{package}\texttt{.h}
 (for instance \texttt{person.h})
\item the core C++ file: \emph{package}\texttt{.cc}
(for instance \texttt{person.cc})
\item files for frontend and backend user method support:
\be
\item stubs: \emph{package}\texttt{stubsfe.h}
and  \emph{package}\texttt{stubsbe.h}
\item skeleton: \emph{package}\texttt{mthfe-skel.h}
and  \emph{package}\texttt{mthbe-skel.h}
\item a template Makefile:
\texttt{Makefile.}\emph{package}
\ee
(for instance \texttt{Makefile.person})
\ee
The use of the generated files for the user methods are introduced in the
chapter \texttt{Methods and Triggers}.
\\
\\
The header file contains C++ class declarations and function prototypes.
\\
The following classes are generated:
\be
\item the package class whose name is the package name and which contains
a static init method, a static release method and two methods for
schema update within a database, for instance:
\verbsize
\begin{verbatim}
class person {
 public:
  static void init();
  static void release();
  static eyedb::Status updateSchema(eyedb::Database *db);
  static eyedb::Status updateSchema(eyedb::Schema *m);
};
\end{verbatim}
\normalsize
\be
\item the \texttt{person::init()} method must be called before any use of
the schema-oriented API.
\item the \texttt{person::release()} should be called after any use
of this API, but this call is not mandatory as this method only
release allocated runtime memory.
\item the \texttt{person::updateSchema()} methods are not generally called
directly by client code.
\ee
\item the database class whose name is \emph{package}Database inherited
from the generic \texttt{eyedb::Database} class.
\\
This class overloads two inherited methods: the \texttt{open} and the
\texttt{loadObject\_realize} methods.
\\
The overloaded \texttt{open} method has two purposes:
\bi
\item database opening.
\item schema checking: it checks that the opened database schema is strictly
identical to the runtime schema.
\ei
The \texttt{loadObject\_realize} method has one purpose:
\bi
\item runtime object construction: for any object loaded from the database
whose class is one of the generated classes
(for instance \texttt{Person}, \texttt{Car}), it call the generated
class constructor. For instance if an object loaded is of class \texttt{Person}
it will perform a \texttt{new Person(db)} to construct correctly
the loaded object.
\ei
Note that to use the generated schema-oriented API it is not mandatory
to use the generated database class: you can use the genetic \texttt{eyedb::Database}
class; there is a lot of cases where you will get no trouble.
But to avoid any potential trouble, it is strongly recommended to use the
generated database class.
\item a root class which is the superclass of
all generated classes, except the package and the database classes.
\\
This class is used to facilitate the down casting process.
\\
If the command line option \texttt{-no-rootclass} is specified, the root
class is not generated.
\\
Unless its name is given using the command line option \texttt{-rootclass name},
its name is \texttt{Root}.
\item for each ODL class, a C++ class is generated with the same name
possibly prefixed by a string if specified by the command line option
\texttt{-class-prefix}.
This class is inherited from the root class.
\\
\\
The generated class contains the following method families:
\be
\item constructors.
\item down casting methods.
\item selector attribute methods.
\item modifier attribute methods.
\item methods mapped from ODL backend or frontend methods.
\item client stubs.
\item the destructor.
\ee
\ee
\subsect{Constructors and Copy Operator}
For each C++ class, two constructors and the assignment operator are
generated:
\verbsize
\begin{verbatim}
  Person(eyedb::Database * = 0);
  Person(const Person& x);

  Person& operator=(const Person& x);
\end{verbatim}
\normalsize
\bi
\item The first constructor is used to instantiate transient or persistent objects.
\\
The following code:
\verbsize
\begin{verbatim}
  Person *p = new Person(&db);
\end{verbatim}
\normalsize
does nearly the same things as:
\verbsize
\begin{verbatim}
  eyedb::Class *cls = db.getSchema()->getClass("Person");
  eyedb::Object *o = cls->newObj(&db);
\end{verbatim}
\normalsize
The major difference is that in the second case, an \texttt{eyedb::Object} instance
(in fact an \texttt{eyedb::Struct} instance) is created while in the first case
an \texttt{Person} (which inherits from \texttt{eyedb::Struct}) instance is created.
\\
But in both cases, you can use the instantiated object to set and get
\texttt{Person} attribute values and to synchronize the runtime object with
the database.
\\
To set or get attribute values in the second case, you need to use
the \texttt{eyedb::Attribute::setValue} or \texttt{eyedb::Attribute::getValue} methods
while in the first case, you may use the generated selector and modifier
methods such as \texttt{Person::setName} or \texttt{Person::getAge}.
\item The second constructor is the copy constructor. For instance:
\verbsize
\begin{verbatim}
  Person *p1 = new Person(&db);

  Person p2 = *p1;
\end{verbatim}
\normalsize
\item At last, the assignment operator can be used as follows:
\verbsize
\begin{verbatim}
  Person *p1 = new Person(&db);
  Person *p2 = new Person(&db);

  *p2 = *p1;
\end{verbatim}
\normalsize
\ei

\subsect{Down Casting Methods and Functions}
Unless the command line option \texttt{-down-casting no} has been used,
down casting methods and functions have been generated.
\\
For instance, the following methods have been generated for the \texttt{Person} class:
\verbsize
\begin{verbatim}
class Person : public Root {
  // ...
  virtual Person *asPerson() {return this;}
  virtual const Person *asPerson() const {return this;}
  virtual Employee *asEmployee() {return (Employee *)0;}
  virtual const Employee *asEmployee() const {return (const Employee *)0;}
  // ...
};
\end{verbatim}
\normalsize
These methods are very useful to process safe down casting.
The down casting may be used in several cases.
For instance, if you instantiate an \texttt{Employee} object as follows:
\verbsize
\begin{verbatim}
  extern void display(Person *);

  Employee *empl = new Employee(&db);
  display(empl);
\end{verbatim}
\normalsize
the display function expects a \texttt{Person} instance: when calling
it with an \texttt{Employee} instance, we do not make any mistake as
the \texttt{Employee} class inherits from the \texttt{Person} class.
\\
Assume now, that the display function displays the name and the age
of the \texttt{Person} instance and its salary if the instance is an
employee. Using the down casting method \texttt{Person::asEmployee()},
one can do as follows:
\verbsize
\begin{verbatim}
void display(Person *p)
{
  cout << "name : " << p->getName() << endl;
  cout << "age : " << p->getAge() << endl;
  if (p->asEmployee())
    cout << "salary : " << p->asEmployee()->getSalary() << endl;
\end{verbatim}
\normalsize
Note that the call to this down casting method cost nearly nothing.
Without the help of the down casting method, the previous code becomes:
\verbsize
\begin{verbatim}
void display(Person *p)
{
  cout << "name : " << p->getName() << endl;
  cout << "age : " << p->getAge() << endl;
  if (!strcmp(p->getClass()->getName(), "Employee"))
    cout << "salary : " << ((Employee *)p)->getSalary() << endl;
\end{verbatim}
\normalsize
which is rather less efficient and less elegant.
\\
\\
There is another case to use down casting methods and functions is when
loading a database object.
\\
When loading a database object (for instance a \texttt{Person} database object)
using the \texttt{eyedb::Database::loadObject}, we get a generic \texttt{eyedb::Object}
instance, not a \texttt{Person} instance nor a \texttt{Employee} instance.
\\
Nevertheless, in the case of a \texttt{Person} database object has been
loaded, a \texttt{Person} persistent runtime object has been correctly
constructed by the generated API.
\\
So, it is legitimate to down cast the loaded \texttt{eyedb::Object} instance
to a \texttt{Person} instance as follows:
\verbsize
\begin{verbatim}
  eyedb::OQL q(&db, "select Person.age < %d", given_age);

  eyedb::ObjectArray obj_arr(eyedb::True);
  q.execute(obj_arr);
  for (int i = 0; i < obj_arr.getCount(); i++)
   {
     Person *p = (Person *)obj_arr[i];
     cout << "name: " << p->getName() << endl;
   }
\end{verbatim}
\normalsize
The cast:
\verbsize
\begin{verbatim}
     Person *p = (Person *)obj_arr[i];
\end{verbatim}
\normalsize
is legal according to the context but is not safe because neither static
(i.e. compiler level) check nor runtime check is performed.
\\
Safe down casting functions are generated by \texttt{eyedbodl} as follows:
\verbsize
\begin{verbatim}
inline Person *Person_c(eyedb::Object *o)
{
  Root *x = personDatabase::asRoot(o);
  if (!x) return (Person *)0;
  return x->asPerson();
}
\end{verbatim}
\normalsize
This function allows you to perform compiler and runtime check as follows:
\verbsize
\begin{verbatim}
  for (int i = 0; i < obj_arr.getCount(); i++)
   {
     Person *p = Person_c(obj_arr[i]);
     if (p) 
       cout << "name: " << p->getName() << endl;
   }
\end{verbatim}
\normalsize
in the case of the loaded object is not a real \texttt{Person} instance,
the \texttt{Person\_c} function returns a null pointer.
\\
\\
It is strongly recommended to make use of these safe down casting methods
and functions instead of performing manual down casting.
\subsect{Selector Methods}
For each attribute in the ODL class, \texttt{eyedbodl} generates one or more
selector methods.
\\
The number and the form of the selector methods depends on the type
of the attribute.
\\
An attribute type is a combination of:
\be
\item a primitive type which can take the form of a:
\be
\item basic type: for instance \texttt{int32}, \texttt{char} or \texttt{double}.
\item system type: for instance \texttt{class}, \texttt{object}, \texttt{image}.
\item user type: for instance \texttt{Person}, \texttt{Employee},
\texttt{set<Car *>}.
\item user enum: for instance \texttt{CivilState}.
\ee
\item the literal or object property:
\be
\item the literal property means that the attribute value has no identifier
(i.e. \texttt{OID}).
\item the object property means that the attribute value has an identifier.
\ee
\item an optional array modifier:
\be
\item multi-dimensionnal and variable size array are supported.
\ee
\ee
For instance, the attribute:
\verbsize
\begin{verbatim}
  attribute Address addr;
\end{verbatim}
\normalsize
can be described as
\{primitive type = \texttt{Address}, property = \texttt{literal}, array = \texttt{nil}\}
%\\
%\\
%while the following attribute:
%\verbsize
%\begin{verbatim}
%  attribute array<Person *> children;
%\end{verbatim}
%\normalsize
%can be described as
%\{primitive type = \texttt{Person}, property = \texttt{object}, array = \texttt{[]}\}
\\
\\
The form of the selector methods are designed according to the
following attribute type family:
\begin{tabbing}
\mbox{ }\mbox{ }\mbox{ }\=4. {\bf object basic, user or system type} \= : \texttt{Person *spouse},\kill
\>1. {\bf literal basic or user enum type} \>: \texttt{int32 age}, \texttt{CivilState cstate}.\\
\>2. {\bf literal string} \>: \texttt{string<32> town}, \texttt{string name},
\texttt{string country}.\\
\>3. {\bf literal user type} \>: \texttt{Address addr}\\
\>4. {\bf object basic, user or system type} \>: \texttt{Person *spouse}\\
\>5. {\bf object collection type}\>: \texttt{array<Person *> children}, \texttt{set<Car *>> cars}.\\
\end{tabbing}
All those type families support in an orthogonal way an multi-dimenstion
array modifier.
%%Concerning the form of the selector methods, the array modifier is orthogonal
%%to those family types.
\subsubsect{Literal Basic or User Enum Type}
The selector method is under the form:
\verbsize
\begin{verbatim}
<attribute primitive type> get<attribute name>(eyedb::Bool *isnull = 0,
                                               eyedb::Status *status = 0) const
\end{verbatim}
\normalsize
for instance for the \texttt{age} attribute:
\verbsize
\begin{verbatim}
  eyedb::_int32 getAge(eyedb::Bool *isnull = 0, eyedb::Status *status = 0) const;
\end{verbatim}
\normalsize
Every selector method has the two following optional arguments:
\be
\item \texttt{isnull} : a pointer to a \texttt{eyedb::Bool} value.\\
If this pointer is not null, the selector method assigns it to \texttt{eyedb::False}
if the attribute value is not null, otherwise it assigns it to \texttt{eyedb::True}.
\item \texttt{status} : a pointer to a \texttt{eyedb::Status} value.\\
If this pointer is not null, the selector method assigns to \texttt{eyedb::Success}
is the operation is successul, otherwise is assigns to the error status.
Note that if you are using the exception error policy (the recommended
one), this argument is not useful. If you have generated the schema-oriented
C++ API using the \texttt{-error-policy exception} option, the
\texttt{status} argument will not be generated.
\ee
\subsubsect{Literal String}
The selector methods are under the form:
\verbsize
\begin{verbatim}
const char *get<attribute name>(eyedb::Bool *isnull = 0,
                                eyedb::Status *status = 0) const
char get<attribute name>(unsigned int a0,
                         eyedb::Bool *isnull = 0,
                         eyedb::Status *status = 0) const
\end{verbatim}
\normalsize
for instance for the \texttt{name} attribute:
\verbsize
\begin{verbatim}
  const char *getName(eyedb::Bool *isnull = 0, eyedb::Status *status = 0) const;
  char getName(unsigned int a0, eyedb::Bool *isnull = 0, eyedb::Status *status = 0)  const;
\end{verbatim}
\normalsize
The first selector method is to get the full string value of the attribute
while the second one is to get a specific character within this string value.
The argument \texttt{a0} is the number of the character.
\subsubsect{Literal User Type}
The selector methods are under the form:
\verbsize
\begin{verbatim}
  <attribute primitive type>* get<attribute name>(eyedb::Bool *isnull = 0,
                                                  eyedb::Status *status = 0)
  const <attribute primitive type>* get<attribute name>(eyedb::Bool *isnull = 0,
                                                        eyedb::Status *status = 0) const
\end{verbatim}
\normalsize
for instance for the \texttt{addr} attribute:
\verbsize
\begin{verbatim}
  Address *getAddr(eyedb::Bool *isnull = 0, eyedb::Status *status = 0) ;
  const Address *getAddr(eyedb::Bool *isnull = 0, eyedb::Status *status = 0) const;
\end{verbatim}
\normalsize
Note than the returned value cannot be a null pointer as this is literal
attribute fully included in the instance.
\subsubsect{Object Basic, User or System Type}
The selector methods are under the form:
\verbsize
\begin{verbatim}
  <attribute primitive type>* get<attribute name>(eyedb::Bool *isnull = 0,
                                                  eyedb::Status *status = 0)
  const <attribute primitive type>* get<attribute name>(eyedb::Bool *isnull = 0,
                                                        eyedb::Status *status = 0) const
\end{verbatim}
\normalsize
for instance for the \texttt{spouse} attribute:
\verbsize
\begin{verbatim}
  Person *getSpouse(eyedb::Bool *isnull = 0, eyedb::Status *status = 0) ;
  const Person *getSpouse(eyedb::Bool *isnull = 0, eyedb::Status *status = 0) const;
  eyedb::Oid getSpouseOid(eyedb::Status *status = 0);
\end{verbatim}
\normalsize
Note that:
\be
\item the returned value can be a null pointer as this a is
an object attribute with its own life.
\item for this same reason, there is a method to get the identifier of this
object without loading it.
\item this selector method automatically loads the related object attribute
when called.
\ee
\subsubsect{Object Collection Type}
As introduced in previous chapter, there are two main types of collections:
ordered (or indexed) collections - \texttt{array} and \texttt{list} - and not
ordered collections - \texttt{set} and \texttt{bag}.
The generated methods for these two main types are similar but a
little bit different.
\\
For the not ordered collections, the selector method are as follows:
\verbsize
\begin{verbatim}
  <collection type>* get<attribute name>Coll(eyedb::Bool *isnull = 0,
                                             eyedb::Status *status = 0) ;
  unsigned int get<attribute name>Count(eyedb::Bool *isnull = 0,
                                        eyedb::Status *status = 0) const;
  const <collection type>* get<attribute name>Coll(eyedb::Bool *isnull = 0,
                                                   eyedb::Status *status = 0) const;
  eyedb::Oid get<attribute name>Oid(eyedb::Status *status = 0);
  const <collection object type>* get<attribute name>At(unsigned int ind,
                                                        eyedb::Bool *isnull = 0,
                                                        eyedb::Status *status = 0) const;
  <collection object type>* get<attribute name>At(unsigned int ind,
                                                  eyedb::Bool *isnull = 0,
                                                  eyedb::Status *status = 0);
  eyedb::Oid get<attribute name>OidAt(unsigned int ind,
                                      eyedb::Status *status = 0) const;
\end{verbatim}
\normalsize
where \texttt{<collection type>} can be:
\be
\item \texttt{eyedb::CollSet} for a collection set
\item \texttt{eyedb::CollBag} for a collection bag
%\item \texttt{eyedb::CollArray} for a collection array
\ee
and where \texttt{<collection object type>} is the type which is composing
the collection.
\\
\\
Note that if the collection is not a literal but an object, the following
extra method returning the collection oid is generated:
\verbsize
\begin{verbatim}
  eyedb::Oid get<attribute name>Oid(eyedb::Status *status = 0);
\end{verbatim}
\normalsize
For the \texttt{cars} attribute the following code is generated:
\verbsize
\begin{verbatim}
  eyedb::CollSet *getCarsColl(eyedb::Bool *isnull = 0, eyedb::Status *status = 0) ;
  unsigned int getCarsCount(eyedb::Bool *isnull = 0, eyedb::Status *status = 0) const;
  const eyedb::CollSet *getCarsColl(eyedb::Bool *isnull = 0, eyedb::Status *status = 0) const;
  const Car *getCarsAt(unsigned int ind, eyedb::Bool *isnull = 0,
                       eyedb::Status *status = 0) const;
  Car *getCarsAt(unsigned int ind, eyedb::Bool *isnull = 0, eyedb::Status *status = 0);
  eyedb::Oid getCarsOidAt(unsigned int ind, eyedb::Status *status = 0) const;
\end{verbatim}
\normalsize
Let have a look to each method:
\be
\item
\verbsize
\begin{verbatim}
eyedb::CollSet *getCarsColl(eyedb::Bool *isnull = 0, eyedb::Status *status = 0) ;
const eyedb::CollSet *getCarsColl(eyedb::Bool *isnull = 0, eyedb::Status *status = 0) const;
\end{verbatim}
\normalsize
returns the collection object (const and not const methods).
\item
\verbsize
\begin{verbatim}
unsigned int getCarsCount(eyedb::Bool *isnull = 0, eyedb::Status *status = 0) const;
\end{verbatim}
\normalsize
returns the collection item count.
\verbsize
\item
\verbsize
\begin{verbatim}
const Car *getCarsAt(unsigned int ind, eyedb::Bool *isnull = 0,
                     eyedb::Status *status = 0) const;
Car *getCarsAt(unsigned int ind, eyedb::Bool *isnull = 0, eyedb::Status *status = 0);
\end{verbatim}
\normalsize
returns the \#\texttt{ind} element in the collection. As the collection
is not ordered, the index of the element to get depends on the
load ordering and is not specified. These array-oriented methods
are generated for user convenience because it is somewhat easier to
scan an array that to scan an unordered set.
\item
\verbsize
\begin{verbatim}
eyedb::Oid getCarsOidAt(unsigned int ind, eyedb::Status *status = 0) const;
\end{verbatim}
\normalsize
returns the \#\texttt{ind} oid in the collection.
The remark about the index of the element is the same as above.
\ee
For the ordered collections, the selector method are as follows:
\verbsize
\begin{verbatim}
  <collection type>* get<attribute name>Coll(eyedb::Bool *isnull = 0,
                                             eyedb::Status *status = 0) ;
  unsigned int get<attribute name>Count(eyedb::Bool *isnull = 0,
                                        eyedb::Status *status = 0) const;
  const <collection type>* get<attribute name>Coll(eyedb::Bool *isnull = 0,
                                                   eyedb::Status *status = 0) const;
  eyedb::Oid get<attribute name>Oid(eyedb::Status *status = 0);
  const <collection object type>* retrieve<attribute name>At(unsigned int ind,
                                                             eyedb::Bool *isnull = 0,
                                                             eyedb::Status *status = 0) const;
  <collection object type>* retrieve<attribute name>At(unsigned int ind,
                                                       eyedb::Bool *isnull = 0,
                                                       eyedb::Status *status = 0);
  eyedb::Oid retrieve<attribute name>OidAt(unsigned int ind,
                                           eyedb::Status *status = 0) const;
\end{verbatim}
\normalsize
where \texttt{<collection type>} can be:
\be
\item \texttt{eyedb::CollArray} for a collection array
\item \texttt{eyedb::CollList} for a collection list\\
\emph{Note that the collection list are currently not implemented in
\eyedb}.
\ee
and where \texttt{<collection object type>} is the type which is composing
the collection.
\\
\\
Note that if the collection is not a literal but an object, the following
extra method returning the collection oid is generated:
\verbsize
\begin{verbatim}
  eyedb::Oid get<attribute name>Oid(eyedb::Status *status = 0);
\end{verbatim}
\normalsize
For the \texttt{children} attribute the following code is generated:
\verbsize
\begin{verbatim}
  eyedb::CollArray *getChildrenColl(eyedb::Bool *isnull = 0, eyedb::Status *status = 0) ;
  unsigned int getChildrenCount(eyedb::Bool *isnull = 0, eyedb::Status *status = 0) const;
  const eyedb::CollArray *getChildrenColl(eyedb::Bool *isnull = 0, eyedb::Status *status = 0) const;
  const Person *retrieveChildrenAt(unsigned int ind, eyedb::Bool *isnull = 0,
                                   eyedb::Status *status = 0) const;
  Person *retrieveChildrenAt(unsigned int ind, eyedb::Bool *isnull = 0,
                             eyedb::Status *status = 0);
  eyedb::Oid retrievedChildrenOidAt(unsigned int ind, eyedb::Status *status = 0) const;
\end{verbatim}
\normalsize
Only the last three method templates differ from the corresponding
\texttt{Car} method templates:
\be
\item
\verbsize
\begin{verbatim}
const Person *retrievedChildrenAt(unsigned int ind, eyedb::Bool *isnull = 0,
                                  eyedb::Status *status = 0) const;
Person *retrievedChildrenAt(unsigned int ind, eyedb::Bool *isnull = 0,
                            eyedb::Status *status = 0);
\end{verbatim}
\normalsize
returns the \#\texttt{ind} element in the collection.
As this collection is ordered, the index of the element to get is fully
pertinent.
\item
\verbsize
\begin{verbatim}
eyedb::Oid retrievedChildrenOidAt(unsigned int ind, eyedb::Status *status = 0) const;
\end{verbatim}
\normalsize
returns the \#\texttt{ind} oid in the collection.
As this collection is ordered, the index of the element to get is fully
pertinent.
\ee
\subsubsect{Array Modifier}
When an array modifier is present for an attribute, all the previous
selector methods change in the same way: for each dimension in the
array, an index argument is added at the begining of the selector method.
\\
For instance, for an attribute \texttt{int x[23][12]}, the selector methods
becomes:
\verbsize
\begin{verbatim}
  eyedb::_int32 getX(unsigned int a0, unsigned int a1,
                     eyedb::Bool *isnull = 0, eyedb::Status *status = 0) const;
\end{verbatim}
\normalsize
A call to \texttt{getX(1, 3)} returns the attribute value \texttt{x[1][3]}.
\\
If the left dimension is variable, for instance \texttt{int x[][12]},
the following extra method is generated:
\verbsize
\begin{verbatim}
  unsigned int getXCount(eyedb::Status * = 0) const;
\end{verbatim}
\normalsize
For instance,
for the \texttt{other\_addrs} literal user type attribute, the following
code is generated:
\verbsize
\begin{verbatim}
  Address *getOtherAddrs(unsigned int a0, eyedb::Bool *isnull = 0, eyedb::Status * = 0) ;
  const Address *getOtherAddrs(unsigned int a0, eyedb::Bool *isnull = 0,
                               eyedb::Status * = 0) const;
  unsigned int getOtherAddrsCount(eyedb::Status * = 0) const;
\end{verbatim}
\normalsize
% EXAMPLE OF  other_addrs
%
%For instance, the \texttt{children} attribute has the following selector
%methods:
%\verbsize
%\begin{verbatim}
%  Person *getChildrenAt(unsigned int a0, eyedb::Bool *isnull = 0, eyedb::Status * = 0) ;
%  const Person *getChildrenAt(unsigned int a0, eyedb::Bool *isnull = 0, eyedb::Status * = 0) const;
%  eyedb::Oid getChildrenOid(unsigned int a0, eyedb::Status * = 0);
%  unsigned int getChildrenCount(eyedb::Status * = 0) const;
%\end{verbatim}
%\normalsize
%The two first method returns the \texttt{children} object at position \#\texttt{a0}, while
%the third method returns the \texttt{children} \texttt{OID} at position \#\texttt{a0}.
%\\
%The last one returns the number of children in the variable dimension array.
\subsect{Modidier Methods}
The modifier methods are very similar to the selector methods.
Their forms and their number depends on the same attribute type characteristics
as the modifier methods.

\subsubsect{Literal Basic or User Enum Type}
The modifier method is under the form:
\verbsize
\begin{verbatim}
eyedb::Status set<attribute name>(<attribute primitive type>);
\end{verbatim}
\normalsize
for instance for the \texttt{age} attribute:
\verbsize
\begin{verbatim}
  eyedb::Status setAge(eyedb::_int32);
\end{verbatim}
\normalsize
\subsubsect{Literal String}
The modifier methods are under the form:
\verbsize
\begin{verbatim}
  eyedb::Status set<attribute name>(const char *);
  eyedb::Status set<attribute name>(unsigned int a0, char);
\end{verbatim}
\normalsize
for instance for the \texttt{name} attribute:
\verbsize
\begin{verbatim}
  eyedb::Status setName(const char *);
  eyedb::Status setName(unsigned int a0, char);
\end{verbatim}
\normalsize

\subsubsect{Literal User Type}
The modifier methods are under the form:
\verbsize
\begin{verbatim}
eyedb::Status set<attribute name>(<attribute primitive type>*);
\end{verbatim}
\normalsize
for instance for the \texttt{addr} attribute:
\verbsize
\begin{verbatim}
  eyedb::Status setAddr(Address*);
\end{verbatim}
\normalsize

\subsubsect{Object Basic, User or System Type}
The modifier methods are under the form:
\verbsize
\begin{verbatim}
eyedb::Status set<attribute name>(<attribute primitive type>*);
eyedb::Status set<attribute name>Oid(const eyedb::Oid &);
\end{verbatim}
\normalsize
for instance for the \texttt{spouse} attribute:
\verbsize
\begin{verbatim}
  eyedb::Status setSpouse(Person*);
  eyedb::Status setSpouseOid(const eyedb::Oid &);
\end{verbatim}
\normalsize

\subsubsect{Object Collection Type}
For unordered collection types, the modifier methods are under the form:
\verbsize
\begin{verbatim}
  eyedb::Status set<attribute name>Coll(<collection type>*);
  eyedb::Status set<attribute name>Oid(const eyedb::Oid &);
  eyedb::Status addTo<attribute name>Coll(<collection object type>*,
                                          unsigned int magorder = 0);
  eyedb::Status rmvFrom<attribute name>Coll(<collection object type>*);
  eyedb::Status addTo<attribute name>Coll(const eyedb::Oid &,
                                          unsigned int magorder = 0);
  eyedb::Status rmvFrom<attribute name>Coll(const eyedb::Oid &);
\end{verbatim}
\normalsize
where \texttt{<collection type>} can be:
\be
\item \texttt{eyedb::CollSet} for a collection set
\item \texttt{eyedb::CollBag} for a collection bag
\ee
and where \texttt{<collection object type>} is the type which composing
the collection.
\\
\\
Note that if the collection is not a literal but an object, the following
extra method setting the collection oid is generated:
\verbsize
\begin{verbatim}
  eyedb::Status set<attribute name>Oid(const eyedb::Oid &);
\end{verbatim}
\normalsize
For the \texttt{cars} attribute, the following code is generated:
\verbsize
\begin{verbatim}
  eyedb::Status setCarsColl(eyedb::CollSet*);
  eyedb::Status addToCarsColl(Car*, unsigned int magorder = 0);
  eyedb::Status addToCarsColl(const eyedb::Oid &, unsigned int magorder = 0);
  eyedb::Status rmvFromCarsColl(Car*);
  eyedb::Status rmvFromCarsColl(const eyedb::Oid &);
\end{verbatim}
\normalsize
Let have a look to each method:
\be
\item
\verbsize
\begin{verbatim}
eyedb::Status setCarsColl(eyedb::CollSet *coll);
\end{verbatim}
\normalsize
sets the \texttt{cars} attribute collection to the input argument
\texttt{coll}.
\item
\verbsize
\begin{verbatim}
eyedb::Status addToCarsColl(Car *car, unsigned int magorder = 0);
\end{verbatim}
\normalsize
adds the \texttt{car} instance to the collection attribute \texttt{cars}.
If the collection is not yet created, this method call will create one
using the \texttt{magorder} argument for its magnitude order value.
\item
\verbsize
\begin{verbatim}
eyedb::Status addToCarsColl(const eyedb::Oid &car_oid, unsigned int magorder = 0);
\end{verbatim}
\normalsize
adds the instance of \texttt{Car} whose oid is \texttt{car\_oid}
to the collection attribute \texttt{cars}.
If the collection is not yet created, this method call will create one
using the \texttt{magorder} argument for its magnitude order value.
\item
\verbsize
\begin{verbatim}
eyedb::Status rmvFromCarsColl(Car *car);
\end{verbatim}
\normalsize
removes the \texttt{car} instance from the collection attribute \texttt{cars}.
If the instance is not found, an error is raised.
\item
\verbsize
\begin{verbatim}
eyedb::Status rmvFromCarsColl(const eyedb::Oid &car_oid);
\end{verbatim}
\normalsize
removes the instance of \texttt{Car} whose oid is \texttt{car\_oid}
from the collection attribute \texttt{cars}.
If the instance is not found, an error is raised.
\ee
For ordered collection types, the modifier methods are under the form:
\verbsize
\begin{verbatim}
  eyedb::Status set<attribute name>Coll(<collection type>*);
  eyedb::Status set<attribute name>Oid(const eyedb::Oid &);
  eyedb::Status setIn<attribute name>CollAt(int where, <collection object type>*,
                                            unsigned int magorder = 0);
  eyedb::Status setIn<attribute name>CollAt(int where, const eyedb::Oid &,
                                            unsigned int magorder = 0);
  eyedb::Status unsetIn<attribute name>CollAt(int where);
\end{verbatim}
\normalsize
where \texttt{<collection type>} can be:
\be
\item \texttt{eyedb::CollArray} for a collection array
\item \texttt{eyedb::CollList} for a collection list\\
\emph{Note that the collection list are currently not implemented in
\eyedb}.
\ee
and where \texttt{<collection object type>} is the type which composing
the collection.
\\
\\
Note that if the collection is not a literal but an object, the following
extra method setting the collection oid is generated:
\verbsize
\begin{verbatim}
  eyedb::Status set<attribute name>Oid(const eyedb::Oid &);
\end{verbatim}
\normalsize
For the \texttt{children} attribute, the following code is generated:
\verbsize
\begin{verbatim}
  eyedb::Status setChildrenColl(eyedb::CollArray*);
  eyedb::Status setChildrenOid(const eyedb::Oid &);
  eyedb::Status setInChildrenCollAt(int where, Person*, unsigned int magorder = 0);
  eyedb::Status unsetInChildrenCollAt(int where);
  eyedb::Status setInChildrenCollAt(int where, const eyedb::Oid &,
                                    unsigned int magorder = 0);
\end{verbatim}
\normalsize
Let have a look to each method:
\be
\item
\verbsize
\begin{verbatim}
eyedb::Status setChildrenColl(eyedb::CollSet *coll);
\end{verbatim}
\normalsize
sets the \texttt{children} attribute collection to the input argument
\texttt{coll}.
\item
\verbsize
\begin{verbatim}
eyedb::Status setInChildrenCollAt(int where, Person *person,
                                  unsigned int magorder = 0);
\end{verbatim}
\normalsize
adds the \texttt{person} instance to the collection attribute \texttt{children} at position \texttt{where}.
If the collection is not yet created, this method call will create one
using the \texttt{magorder} argument for its magnitude order value.
\item
\verbsize
\begin{verbatim}
eyedb::Status setInChildrenCollAt(int where, const eyedb::Oid &person_oid,
                                  unsigned int magorder = 0);
\end{verbatim}
\normalsize
adds the instance of \texttt{Person} whose oid is \texttt{person\_oid}
to the collection attribute \texttt{children} at position \texttt{where}..
If the collection is not yet created, this method call will create one
using the \texttt{magorder} argument for its magnitude order value.
\item
\verbsize
\begin{verbatim}
eyedb::Status unsetInChildrenCollAt(int where);
\end{verbatim}
\normalsize
removes the instance found at position \texttt{where}
from the collection attribute \texttt{children}.
\ee
\subsubsect{Array Modifier}
When an array modifier is present for an attribute, all the previous
modifier methods change in the same way: for each dimension in the
array, an index argument is added at the begining of the selector method.
\\
For instance, for an attribute \texttt{int x[32][64]}, the modifier methods
becomes:
\verbsize
\begin{verbatim}
  eyedb::Status setX(unsigned int a0, unsigned int a1, eyedb::_int32);
\end{verbatim}
\normalsize
A call to \texttt{setX(2, 24)} sets the attribute value \texttt{x[2][24]}.
\\
\\
For instance,
for the \texttt{other\_addrs} literal user type attribute, the following
code is generated:
\verbsize
\begin{verbatim}
  eyedb::Status setOtherAddrs(unsigned int a0, Address *);
  eyedb::Status setOtherAddrsCount(unsigned int count);
\end{verbatim}
\normalsize
\subsubsect{Methods mapped from ODL methods}
For each ODL class method, there is a generated C++ method with
the same name and the corresponding type.
\\
The generated methods in our example is as follows:
\verbsize
\begin{verbatim}
  virtual eyedb::Status change_address(const char * street, const char * town,
                                       char * &oldstreet, char * &oldtown,
                                       eyedb::_int32 &retarg);

  static eyedb::Status getPersonCount(eyedb::Database *db, eyedb::_int32 &retarg);
\end{verbatim}
\normalsize

%%\subsubsect{Client Stubs}

\subsect{Initialization}
The minimal \eyedb C++ program using a generated schema-oriented API
is as follows (using our example):
\verbsize
\begin{verbatim}
#include "person.h"

int
main(int argc, char *argv[])
{
  eyedb::init(argc, argv);
  person::init();
  // ...
  person::release();
  eyedb::release();
  return 0;
}
\end{verbatim}
\normalsize
A few remarks about this code:
\be
\item the file \texttt{person.h} contains the whole generated C++ API and
includes the generic \eyedb API.
\item the \eyedb C++ layer must be initialized using one of the static method
\texttt{init} method of the class \texttt{EyeDB}.
\item the generated C++ layer must be initialized using the static method
\texttt{init} of the class \emph{package}.
\item the last statements \texttt{person::release()} and \texttt{eyedb::release()}
allow you to release all the allocated resources and to close opened databases
and connections.
\\
Note that this statement is optionnal as all allocated resources,
opened databases and connections will be automatically released or closed
in the \texttt{exit()} function.
\ee

\subsect{Database Opening}
As shown in a previous section, it is recommended to use the generated
C++ database class to open a database with the template schema.
\\
For instance:
\verbsize
\begin{verbatim}
  eyedb::Connection conn;
  conn.open();

  const char *dbname = argv[1];

  person::Database db(dbname);
  db.open(&conn, eyedb::Database::DBRW);
\end{verbatim}
\normalsize
%% Topics to introduce:
%%  - the class hierarchy and the underlying object model
%%     Object/Instance/Agregat/Struct/Enum/Basic/Collection/CollXxxx
%%     Object/Class/AgregatClass/StructClass/EnumClass/BasicClass/CollectionClass
%%     Database
%%     Schema
%%     -> briefly the main methods for instantiation and schema browsing
%%     -> the main methods for each class family
%%  - transactions
%%  - creating and updating objects
%%     -> eyedb::RecMode
%%     -> eyedb::Database::storeOnCommit(), xxxRegister(), storeObjects()
%%  - loading objects
%%     -> eyedb::RecMode
%%  - removing objects
%%     -> eyedb::RecMode
%%  - cache management
%%     -> first level cache => mechanisms + eyedb::Database::xxxx
%%     -> schema-oriented second level 
%%  - schema-oriented calling methods
%%  - collections: all the methods
%%  - complex queries: eyedb::Value
%%  - schema && class manipulation: eyedb::Class, eyedb::Attribute, eyedb::Schema
%%  - schema generation: eyedb::Schema::genXXX()
%%  - memory management
%%  - error management

%%\subsect{Object Manipulation}

%%\subsect{Creating Runtime Objects}

%%\subsect{Object Loading}

%%\subsect{Calling Methods}

%%\sect{Error Management}

%%\sect{Cache Management}

%%\sect{Memory Management}

%%\sect{The Class Hierarchy}

\sect{Examples}
This section introduces a few complete simple examples that can be found
in the directory
\emph{prefix}\texttt{/share/doc/eyedb/examples}.
The \texttt{README} file describes the way to compile and run these examples.
\\
The first two programs listed here introduce the generic C++ API of
\eyedb while the two following programs presents the
generated schema-oriented C++ API through the simple schema example introduced
in this chapter. The last example shows \texttt{EYEDBDBM} instance
manipulation.
\subsect{Generic Query Example}
This example introduces a simple query program which takes
two arguments: the database name and an OQL construct. It executes
the OQL construct and displays on its standard output the returned
atoms.
\verbsize
\begin{verbatim}
// examples/C++Binding/generic/query/query.cc

#include <eyedb/eyedb.h>

using namespace std;

int
main(int argc, char *argv[])
{
  eyedb::init(argc, argv);

  if (argc != 3) {
    fprintf(stderr, "usage: %s <dbname> <query>\n", argv[0]);
    return 1;
  }

  eyedb::Exception::setMode(eyedb::Exception::ExceptionMode);

  try {
    eyedb::Connection conn;
    // connecting to the eyedb server
    conn.open();

    eyedb::Database db(argv[1]);

    // opening database argv[1]
    db.open(&conn, eyedb::Database::DBRW);

    // beginning a transaction
    db.transactionBegin();

    // performing the OQL query argv[2] using the eyedb::OQL interface
    eyedb::OQL q(&db, argv[2]);

    eyedb::ValueArray arr;
    q.execute(arr);

    cout << "###### Performing the OQL query " << argv[2] <<
      " using the eyedb::OQL interface" << endl;

    // for each value returned in the query, display it:
    for (int i = 0; i < arr.getCount(); i++) {
      // in case of the returned value is an oid, load it first:
      if (arr[i].type == eyedb::Value::OID) {
        eyedb::Object *o;
        db.loadObject(arr[i].oid, &o);
        cout << o;
        o->release();
      }
      else
        cout << arr[i] << endl;
    }

    // performing the same query using eyedb::OQLIterator interface
    // [1]: getting all returned values

    cout << "\n###### Performing the same query using eyedb::OQLIterator "
      "interface: getting all returned values" << endl;

    eyedb::OQLIterator iter(&db, argv[2]);
    eyedb::Value val;

    while (iter.next(val)) {
      // in case of the returned value is an oid, load it first:
      if (val.getType() == eyedb::Value::OID) {
        eyedb::Object *o;
        db.loadObject(val.oid, &o);
        cout << o;

        // in case of the returned object is a collection, display its
        // contents
        if (o->asCollection()) {
          eyedb::CollectionIterator citer(o->asCollection());
          cout << "contents BEGIN\n";
          eyedb::Object *co;
          while(citer.next(co)) {
            cout << co;
            co->release();
          }
          cout << "contents END\n\n";
        }
        // in case of the returned object is a class, display its
        // extent
        else if (o->asClass()) {
          eyedb::ClassIterator citer(o->asClass());
          cout << "extent BEGIN\n";
          eyedb::Object *co;
          while(citer.next(co)) {
            cout << co;
            co->release();
          }
          cout << "extent END\n\n";
        }

        o->release();
      }
      else
        cout << val << endl;
    }

    // [2]: getting only returned objects

    cout << "\n###### Performing the same query using eyedb::OQLIterator "
      "interface: getting only returned objects" << endl;

    eyedb::OQLIterator iter2(&db, argv[2]);
    eyedb::Object *o;

    while (iter2.next(o)) {
      cout << o;
      o->release();
    }

    // committing the transaction
    db.transactionCommit();
  }

  catch(eyedb::Exception &e) {
    cerr << argv[0] << ": " << e;
    eyedb::release();
    return 1;
  }

  eyedb::release();

  return 0;
}
\end{verbatim}
For instance:
\begin{verbatim}
./query person "select Person"
./query EYEDBDBM "select class"
\end{verbatim}
\normalsize
\subsect{Generic Storing Example}
This example introduces a simple store program which takes
three arguments: the database name, a person name and a person age. It
creates a new instance of person using the given name and age.
\verbsize
\begin{verbatim}
// examples/C++Binding/generic/store/store.cc

#include <eyedb/eyedb.h>

using namespace std;

int
main(int argc, char *argv[])
{
  eyedb::init(argc, argv);

  if (argc != 4) {
    fprintf(stderr, "usage: %s <dbname> <person_name> <person_age>\n",
            argv[0]);
    return 1;
  }

  eyedb::Exception::setMode(eyedb::Exception::ExceptionMode);

  try {
    eyedb::Connection conn;
    // connecting to the eyedb server
    conn.open();

    eyedb::Database db(argv[1]);

    // opening database argv[1]
    db.open(&conn, eyedb::Database::DBRW);

    // beginning a transaction
    db.transactionBegin();

    // looking for class 'Person'
    eyedb::Class *personClass = db.getSchema()->getClass("Person");

    // looking for the attribute 'name' and 'age' in the class 'Person'
    const eyedb::Attribute *name_attr = personClass->getAttribute("name");
    if (!name_attr) {
      fprintf(stderr, "cannot find name attribute in database\n");
      return 1;
    }

    const eyedb::Attribute *age_attr = personClass->getAttribute("age");

    if (!age_attr) {
      fprintf(stderr, "cannot find age attribute in database\n");
      return 1;
    }

    // instanciating a 'Person' object
    eyedb::Object *p = personClass->newObj(&db);

    // setting the name argv[2] to the new Person instance
    name_attr->setSize(p, strlen(argv[2])+1);
    name_attr->setValue(p, (eyedb::Data)argv[2], strlen(argv[2])+1, 0);

    // setting the age argv[3] to the new Person instance
    int age = atoi(argv[3]);
    age_attr->setValue(p, (eyedb::Data)&age, 1, 0);
    p->store();

    // committing the transaction
    db.transactionCommit();
  }

  catch(eyedb::Exception &e) {
    cerr << e;
    eyedb::release();
    return 1;
  }

  eyedb::release();

  return 0;
}
\end{verbatim}
For instance:
\begin{verbatim}
./store person john 32
./store person mary 28
\end{verbatim}

\normalsize
\subsect{Schema-Oriented Query Example}
This example introduces a simple schema-oriented query program which takes
two arguments: the database name and an OQL construct. It executes
the OQL construct and displays on its standard output the returned
atoms.
\verbsize
\begin{verbatim}
// examples/C++Binding/schema-oriented/query/query.cc

#include "person.h"

using namespace std;

int
main(int argc, char *argv[])
{
  eyedb::init(argc, argv);
  person::init();

  if (argc != 3) {
    fprintf(stderr, "usage: %s <dbname> <query>\n", argv[0]);
    return 1;
  }

  eyedb::Exception::setMode(eyedb::Exception::ExceptionMode);

  try {
    eyedb::Connection conn;
    // connecting to the eyedb server
    conn.open();

    // opening database argv[1] using 'personDatabase' class
    personDatabase db(argv[1]);
    db.open(&conn, eyedb::Database::DBRW);

    // beginning a transaction
    db.transactionBegin();

    // performing the OQL query argv[2]
    eyedb::OQL q(&db, argv[2]);

    eyedb::ObjectArray arr;
    q.execute(arr);

    // for each Person returned in the query, display its name and age,
    // its address, its spouse name and age and its cars
    for (int i = 0; i < arr.getCount(); i++) {
      Person *p = Person_c(arr[i]);
      if (p) {
        cout << "name:    " << p->getName() << endl;
        cout << "age:     " << p->getAge() << endl;

        if (p->getAddr()->getStreet().size())
          cout << "street:  " << p->getAddr()->getStreet() << endl;

        if (p->getAddr()->getTown().size())
          cout << "town:    " << p->getAddr()->getTown() << endl;

        if (p->getSpouse()) {
          cout << "spouse_name: " << p->getSpouse()->getName() << endl;
          cout << "spouse_age:  " << p->getSpouse()->getAge() << endl;
        }

        eyedb::CollectionIterator iter(p->getCarsColl());
        Car *car;
        while (iter.next((eyedb::Object *&)car)) {
          cout << "car: #" << i << ": " <<
            car->getBrand() << ";" <<
            car->getNum() << endl;
        }
      }
    }

    // committing the transaction
    db.transactionCommit();
  }

  catch(eyedb::Exception &e) {
    cerr << argv[0] << ": " << e;
    eyedb::release();
    return 1;
  }

  eyedb::release();

  return 0;
}
\end{verbatim}
For instance:
\begin{verbatim}
./query person "select Person"
\end{verbatim}
\normalsize
\subsect{Schema-Oriented Storing Example}
This example introduces a simple schema-oriented store program which takes
four arguments: the database name, a person name, a person age and the
name of its spouse. It
creates a new instance of person using the given name and age and mary
this person to the spouse whose name is given.
\verbsize
\begin{verbatim}
// examples/C++Binding/schema-oriented/store/store.cc

#include "person.h"

int
main(int argc, char *argv[])
{
  // initializing the EyeDB layer
  eyedb::init(argc, argv);

  // initializing the person package
  person::init();

  if (argc != 5) {
    fprintf(stderr, "usage: %s <dbname> <person name> <person age> "
            "<spouse name>\n", argv[0]);
    return 1;
  }

  const char *dbname = argv[1];
  const char *name = argv[2];
  int age = atoi(argv[3]);
  const char *spouse_name = argv[4];

  eyedb::Exception::setMode(eyedb::Exception::ExceptionMode);

  try {
    eyedb::Connection conn;

    // connecting to the EyeDB server
    conn.open();

    // opening database dbname using 'personDatabase' class
    personDatabase db(dbname);
    db.open(&conn, eyedb::Database::DBRW);

    // beginning a transaction
    db.transactionBegin();

    // first looking for spouse
    eyedb::OQL q(&db, "select Person.name = \"%s\"", spouse_name);

    eyedb::ObjectArray arr;
    q.execute(arr);

    // if not found, returns an error
    if (!arr.getCount()) {
      fprintf(stderr, "cannot find spouse '%s'\n", spouse_name);
      return 1;
    }

    // (safe!) casting returned object
    Person *spouse = Person_c(arr[0]);

    // creating a Person
    Person *p = new Person(&db);

    p->setCstate(Sir);
    p->setName(name);
    p->setAge(age);

    p->setSpouse(spouse);

    // spouse is no more necessary: releasing it
    spouse->release();

    p->getAddr()->setStreet("voltaire");
    p->getAddr()->setTown("paris");

    // creating two cars
    Car *car1 = new Car(&db);
    car1->setBrand("renault");
    car1->setNum(18374);

    Car *car2 = new Car(&db);
    car2->setBrand("ford");
    car2->setNum(233491);

    // adding the cars to the created person
    p->addToCarsColl(car1);
    p->addToCarsColl(car2);

    // car pointers are no more necessary: releasing them
    car1->release();
    car2->release();

    // creating ten children
    for (int i = 0; i < 10; i++) {
      Person *c = new Person(&db);
      char tmp[64];

      c->setAge(i);
      sprintf( tmp, "%d", i);
      c->setName( (std::string(name) + std::string("_") + std::string(tmp)).c_str() );
      p->setInChildrenCollAt(i, c);
      c->release();
    }

    // storing all in database
    p->store(eyedb::RecMode::FullRecurs);

    // committing the transaction
    db.transactionCommit();

    // releasing p
    p->release();
  }

  catch(eyedb::Exception &e) {
    std::cerr << argv[0] << ": " << e;
    eyedb::release();
    return 1;
  }

  // releasing the EyeDB layer: this is not mandatory
  eyedb::release();

  return 0;
}
\end{verbatim}
\begin{verbatim}
./store person wayne 34
./store person poppins 51
\end{verbatim}
\normalsize

\subsect{Simple Administration Example}
This simple example introduces the way to manipulate objects in the
\texttt{EYEDBDBM} database. This program:
\be
\item displays the schema of the \texttt{EYEDBDBM} database,
\item displays the \eyedb user names,
\item for each database, it displays the name, the database file and
the user access information.
\ee
\verbsize
\begin{verbatim}
// examples/C++Binding/schema-oriented/admin/admin.cc

#include <eyedb/eyedb.h>

using namespace std;

static const char *
get_string_mode(eyedb::DBAccessMode mode)
{
  if (mode == eyedb::NoDBAccessMode)
    return "eyedb::NoDBAccessMode";
  if (mode == eyedb::ReadDBAccessMode)
    return "eyedb::ReadDBAccessMode";
  if (mode == eyedb::WriteDBAccessMode)
    return "eyedb::WriteDBAccessMode";
  if (mode == eyedb::ExecDBAccessMode)
    return "eyedb::ExecDBAccessMode";
  if (mode == eyedb::ReadWriteDBAccessMode)
    return "eyedb::ReadWriteDBAccessMode";
  if (mode == eyedb::ReadExecDBAccessMode)
    return "eyedb::ReadExecDBAccessMode";
  if (mode == eyedb::ReadWriteExecDBAccessMode)
    return "eyedb::ReadWriteExecDBAccessMode";
  if (mode == eyedb::AdminDBAccessMode)
    return "eyedb::AdminDBAccessMode";

  return NULL;
}

int
main(int argc, char *argv[])
{
  // initializing the eyedb layer
  eyedb::init(argc, argv);

  eyedb::Exception::setMode(eyedb::Exception::ExceptionMode);

  try {
    eyedb::Connection conn;

    // connecting to the eyedb server
    conn.open();

    // opening the database EYEDBDBM using 'dbmDataBase' class
    eyedb::DBMDatabase db("EYEDBDBM");
    db.open(&conn, eyedb::Database::DBRead);

    // beginning a transaction
    db.transactionBegin();

    // display the scheme on stdout
    cout << db.getSchema() << endl;

    // looking for all user
    eyedb::OQL q_user(&db, "select User");

    eyedb::ObjectArray user_arr;
    q_user.execute(user_arr);

    cout << "User List {" << endl;
    for (int i = 0; i < user_arr.getCount(); i++) {
      eyedb::UserEntry *user = (eyedb::UserEntry *)user_arr[i];
      cout << "\t" << user->name() << endl;
    }
    cout << "}\n" << endl;

    // looking for all database entry
    eyedb::OQL q_db(&db, "select eyedb::DBEntry");

    eyedb::ObjectArray db_arr;
    q_db.execute(db_arr);

    cout << "Database List {" << endl;

    for (int i = 0; i < db_arr.getCount(); i++) {
      eyedb::DBEntry *dbentry = (eyedb::DBEntry *)db_arr[i];
      cout << "\t" << dbentry->dbname() << " -> " << dbentry->dbfile() << endl;
      // looking for all user which has any permission on this
      // database
      eyedb::OQL q_useraccess(&db,
                              "select eyedb::DBUserAccess->dbentry->dbname = \"%s\"",
                              dbentry->dbname().c_str());
      eyedb::ObjectArray useraccess_arr;
      q_useraccess.execute(useraccess_arr);
      if (useraccess_arr.getCount()) {
        cout << "\tUser Access {" << endl;
        for (int j = 0; j < useraccess_arr.getCount(); j++) {
          eyedb::DBUserAccess *ua = (eyedb::DBUserAccess *)useraccess_arr[j];
          cout << "\t\t" << ua->user()->name() << " -> " <<
            get_string_mode(ua->mode()) << endl;
        }
        cout << "\t}" << endl;
      }
      cout << endl;
      useraccess_arr.garbage();
    }

    cout << "}" << endl;

    // releasing runtime pointers
    db_arr.garbage();
    user_arr.garbage();
  }

  catch(eyedb::Exception &e) {
    cerr << argv[0] << ": " << e;
    eyedb::release();
    return 1;
  }

  // releasing the eyedb layer: this is not mandatory
  eyedb::release();

  return 0;
}
\end{verbatim}
\normalsize

%%\subsect{Method Call}

\end{document}
