\documentclass[dvips,10pt]{book}
%\documentclass[dvips,10pt]{report}

\usepackage{latexsym}
%\usepackage{amstex}
%\usepackage{amssymb}
%\usepackage{amsfonts}
\usepackage{amstext}
\usepackage{amsbsy}
\usepackage{amscd}
\usepackage{makeidx}
\usepackage{graphicx}
%\usepackage{calc}
\usepackage{longtable}

% -- to reduce the margin pages --
\topmargin 2pt
\headsep 20pt
\headheight 2pt

%\textwidth 440pt
%\textheight 690pt
%\oddsidemargin 0pt
%\evensidemargin 0pt

% 19/01/06: EV
\textwidth 490pt
\textheight 700pt
\oddsidemargin -20pt
\evensidemargin -20pt

% --------------------------------

\makeindex

\newcommand{\eyedb}{\textsc{EyeDB} }
\newcommand{\eyedbX}{\textsc{EyeDB}}
\newcommand{\sysra}{SYSRA}
\newcommand{\indexentry}[2]{\item #1 #2}
\newcommand{\chap}[1]{\chapter{#1} \index{#1}}
\newcommand{\sect}[1]{\section{#1} \index{#1}}
\newcommand{\subsect}[1]{\subsection{#1} \index{#1}}
\newcommand{\subsubsect}[1]{\subsubsection{#1} \index{#1}}
%% WARNING \Box does not work with latex2html
%%\newcommand{\iii}{\item[{$\Box$}]}
\newcommand{\rrr}{\item[{$\Box$}]}
\newcommand{\iii}{\item}
\newcommand{\jjj}{\item[-]}
\newcommand{\bi}{\begin{itemize}}
\newcommand{\ei}{\end{itemize}}
\newcommand{\be}{\begin{enumerate}}
\newcommand{\ee}{\end{enumerate}}
\newcommand{\kkk}{\item[ ]}
\newcommand{\sh}[1]{\bf #1}
\newcommand{\ttt}{$\tilde{ }$}
\newcommand{\ttv}[1]{\texttt{#1}}
\newcommand{\ttu}[1]{{\verbsize\texttt{#1}}}
\newcommand{\verbsizex}{\small}
\newcommand{\verbsizey}{\fontsize{8}{\baselineskip}\selectfont}
\newcommand{\verbsize}{\normalsize}
\newcommand{\bugreport}{bug-report@eyedb.org }
\newcommand{\idt}{\mbox{ } - }
\newcommand{\ident}{{\bf \texttt{<}identifier\texttt{>}}}
\newcommand{\grindent}{xxxxxxxxxxxxxxx \= : \kill}
\newcommand{\arr}{\texttt{-$>$}}
\newcommand{\ul}[1]{\underline{#1}}
\newcommand{\uul}[1]{\textit{\texttt{#1}}}
\newcommand{\ixx}{\makebox[3mm]{ }}
\newcommand{\ixy}{\makebox[10mm]{ }}
\newcommand{\ixz}{\\\ixy {- }}

\renewcommand{\normalsize}{\fontsize{9}{11pt}\selectfont}

%\renewcommand{\normalsize}{\fontsize{8}{10pt}\selectfont}

%% The following commands comes from corbaref.tex

\newcommand{\bt}{\begin{tabular}{|p{7cm}|p{7cm}|}}
\newcommand{\et}{\hline \end{tabular} \\ {\vspace{0.6cm}} \\ }

\newcommand{\mt}[1]{\hline \multicolumn{2}{|c|}{\centerline{{\bf interface #1}
   \emph{mapped from} {\bf class #1}}}}

\newcommand{\Desc}[1]{\hline \multicolumn{2}{|l|}{\emph{#1}}}

\newcommand{\mTT}[2]{\hline \multicolumn{2}{|c|}
   {\centerline{{\bf interface #1} \emph{extends} {\bf #2}
   \emph{mapped from} {\bf #1}}}}


\newcommand{\mT}[2]{\hline \multicolumn{2}{|c|}{\centerline{{\bf #1}
\emph{mapped from} {\bf #1}}}}

\newcommand{\att}{\hline \hline {\bf Attribute} & {\bf Description}}
\newcommand{\mtt}{\hline \hline {\bf Method} & {\bf Description}}
\newcommand{\mf}{\emph{mapped from }}

\newcommand{\satt}[1]{\hline \multicolumn{2}{|c|}
   {\centerline{\emph{attributes inherited from} {\bf #1}}}}

\newcommand{\smtt}[1]{\hline \multicolumn{2}{|c|}{
   \centerline{\emph{methods inherited from} {\bf #1}}}}

\newcommand{\sets}[1]{sets the #1 \emph{value} in the Object \emph{o} at the \emph{ind}}

\newcommand{\Gets}[1]{gets the #1 value in the Object \emph{o} at the \emph{ind}}

%\setcounter{secnumdepth}{-1}
%\addtocounter{secnumdepth}{2}

\renewcommand{\thesection}{\arabic{section}}
\renewcommand{\thesubsection}{\arabic{section}.\arabic{subsection}}
\renewcommand{\thechapter}{}

\begin{document}
\normalsize


\input{version}
\newcommand{\mantitle}{\textsc{Object Definition Language}}

\thispagestyle{empty}

\mbox{ }
\\
\vspace{3 cm}
\\
%{\bf{\Huge E{\LARGE YE}DB} \hspace{0.cm} {\LARGE Programming Manual}}
{\bf{\Huge E{\LARGE YE}DB} \hspace{0.cm} {\Huge \mantitle}}
\\
%\newcommand{\rulewidth}{12.2cm}
\newcommand{\rulewidth}{15.4cm}
\rule{\rulewidth}{1.6mm}
\begin{flushright}
{\large Version \eyedbversion}
\end{flushright}
\mbox{ }
\\
\vspace{11 cm}
\\
\begin{flushright}
{\large January 2006}
\end{flushright}

\newpage

\thispagestyle{empty}

\mbox{ }
\\
\vspace{1cm}
\\
Copyright {\copyright} 2001-2006 SYSRA
\\
\vspace{1cm}
\\
Published by \sysra
\\
30, avenue G\'en\'eral Leclerc
\\
91330 Yerres - France
\\
\\
home page: http://www.eyedb.org
\\
\\
bug report: \bugreport


\tableofcontents

\chapter{The Object Definition Language}

The \eyedb Object Definition Language (ODL) is a specification language
to define the specifications of object types based on the ODMG ODL.\\
ODL \footnote{ODL is used for shortness to denote \eyedb ODL} 
is not intented to be a full programming language. It is
a definition language for object specifications. Database management
systems traditionaly provide facilities that support
data definition (using a Data Definition Language (DDL)). The DDL allows
users to define their data types and interfaces while the
Data Manipulation Languages (DML) allows to create, delete, read update
instances of those data types.\\
ODL is a DDL for objects types.
If defines the characteristics of types, including their properties
and operations.
ODL defines only the signatures of operations and does not address
definitions of the methods that implements those operations.
\\
ODL is intented to define object types that can be implemented
in a variety of programming languages. Therefore, ODL is not tied
to the syntax of a particular programming language.\\\\
\eyedb ODL differs from ODMG ODL from several points:
\bi
\item ODMG ODL defines class attributes, relationships, method signatures and keys.
It supports nested classes, typedef constructs, constant definitions and
exception hints.
\item \eyedb ODL defines class attributes, relationships, method signatures,
attribute constraints (notnull, unique, collection cardinality), index
specifications and trigger declarations. It does not support
nested classes, typedef constructs, constant definitions and
exception hints.
\item in \eyedb ODL, any type instance can be both a literal or
an object. In ODMG ODL, this property is tied to the type: all
basic types and user defined \emph{struct} are literal while \emph{interface}s
and \emph{class}es are objects.
In \eyedb ODL, any type instance can be an object, even the basic types.
\item at last, \eyedb ODL allows to specify whether a method is backend or frontend,
and whether it is a class or instance method.
\ei

\sect{The Language Specifications}

To specify ODL we use the BNF notation with the following typographic
conventions:
\begin{list}{-}{{\itemsep 0mm}} %%{\parsep 0mm}}
\item non terminal are in standard font and in lowcase.
\item token are in {\bf bold} font, and surrounded with {\bf $<$$>$} if they
denote a set of tokens.
\item optional grammar symbols are surrounded with open and close brackets.
\end{list}

\sect{The Top Level ODL}
The top-level BNF is a sequence of type specifications:
\begin{tabbing}
\grindent
odl\_construct \> : odl\_construct type\_spec\\
\> $|$ \emph{empty token}
\end{tabbing}

\sect{Type Specification}

A type is defined by specifying its class, union or enum in ODL.\\
The characteristics of the type itself appears first, followed by
lists that defined attributes, relationships and operations of its
class.
\begin{tabbing}
\grindent
type\_spec \> : type\_qualifier class\_spec {\bf \{} [class\_body] {\bf ;} {\bf \}} {\bf ;}\\
type\_spec \> : {\bf enum} \ident [alias\_name] {\bf \{} enum\_body [{\bf ,}] {\bf \}}
\end{tabbing}

\subsect{The Type Qualifier}

\begin{tabbing}
\grindent
type\_qualifier \> : {\bf class}\\
\> $|$ {\bf struct}\\
\> $|$ {\bf union}\\
\> $|$ {\bf interface}\\
\> $|$ {\bf superclass}\\
\> $|$ {\bf superstruct}
\end{tabbing}
A \emph{type\_spec} is denoted by a \emph{type\_qualifier} which must be one
of the following tokens:\\
\idt {\bf class} (equivalent to {\bf interface} and {\bf struct})\\
\idt {\bf union} (\emph{not implemented in version \eyedbversion})\\
\idt {\bf superclass} (equivalent to {\bf superstruct})

\subsect{The Class Specification}

\begin{tabbing}
\grindent
class\_spec \> : \ident [class\_impl] [alias\_name] [extends \ident]
\end{tabbing}

\subsect{The Alias Name}
\begin{tabbing}
\grindent
alias\_name \> : {\bf [} \ident {\bf ]}
\end{tabbing}

\subsect{The Class Implementation}
\begin{tabbing}
\grindent
class\_impl \> : {\bf ( implementation $<$} index\_spec {\bf $>$)}
\end{tabbing}

\subsect{Examples}

\begin{tabbing}
{\bf cla}\={\bf ss} Person {\bf \{}\\
\>\emph{[class\_body]}\\
{\bf \}};\\
\\
{\bf class} Employee {\bf extends} Person {\bf \{}\\
\>\emph{[class\_body]}\\
{\bf \}};\\
\\
{\bf class} Employee {\bf :} Person {\bf \{}\\
\>\emph{[class\_body]}\\
{\bf \}};\\
\\
{\bf class} Slave [pp\_Slave] {\bf extends} Employee {\bf \{}\\
\>\emph{[class\_body]}\\
{\bf \}};\\
\\
{\bf class} Slave [slave\_class] [\~{ }100K] {\bf extends} Employee {\bf \{}\\
\>\emph{[class\_body]}\\
{\bf \}};\\
\\
{\bf enum} CivilState {\bf \{}\\
\>\emph{[enum\_body]}\\
{\bf \}};\\
\\
{\bf enum} CivilState [civil\_state] {\bf \{}\\
\>\emph{[enum\_body]}\\
{\bf \}};
\end{tabbing}

\sect{Class Body Specification}
\begin{tabbing}
\grindent
class\_body \> : class\_body item\_dcl\\
\> $|$ item\_dcl\\
\\
item\_dcl \> : attr\_dcl\\
\> $|$ {\bf attribute} attr\_dcl\\
\> $|$ {\bf relship} attr\_dcl\\
\> $|$ attr\_hint\\
\> $|$ op\_dcl
\end{tabbing}

\sect{Attribute Specification}

\begin{tabbing}
\grindent
attr\_dcl \> : \ident [ref] \ident [array] [inverse] {\bf ;}\\
\> $|$ coll\_spec [ref] \ident [array] [inverse] {\bf ;}
\end{tabbing}

\subsect{Reference Qualifier}
\begin{tabbing}
\grindent
ref \> : {\bf *}\\
\> $|$ {\bf \&}
\end{tabbing}
The reference type qualifier denoted by {\bf *} or {\bf \&}, indicates that the
attribute refers to an object in the database of the indicated type.
\\
\\
For instance the attribute field \emph{Person *spouse}, means that spouse
is an object of type Person, while \emph{Address addr} means that addr
is an attribute of type Address but not an object of the database: it is
fully included in the structure Person.
\\
\\
The attribute \emph{spouse} is called an \underline{indirect attribute},
while \emph{addr} is called an \underline{direct attribute}.
\\
\\
Do not confuse the {\bf *} ODL meaning and the {\bf *} C/C++ meaning.
\\
\\
In C/C++, the {\bf *} type modifier denotes an address to an area of the
indicated type instances: it is a pointer to an address. This pointer
can be incremented and decremented to change its location in the area.
\\
In ODL, the {\bf *} denotes a reference to one and only one object, it is why
the {\bf \&} token is also accepted, although the meaning of this token
is a little bit different in C++.
\\
\\
So, in ODL the construct \emph{Person **x} makes no sense, in the same
manner that the construct \emph{Person} \&\&\emph{x} makes no sense in C++.


\subsect{Array Qualifier}
\begin{tabbing}
\grindent
array \> : {\bf [} [expr\_int] {\bf ]}\\
\> $|$ array {\bf [} [expr\_int] {\bf ]}
\end{tabbing}

\subsect{Collection Type Specification}
\begin{tabbing}
\grindent
coll\_spec \> : \ident {\bf $<$} \ident {\bf [} expr\_int {\bf ]} {\bf $>$}\\
\> $|$ \ident {\bf $<$} \ident [ref] {\bf $>$}\\
\> $|$ \ident {\bf $<$} coll\_spec [ref] {\bf $>$}\\
\> $|$ \ident {\bf $<$} coll\_spec {\bf [} expr\_int {\bf ]} {\bf $>$}
\end{tabbing}

\subsect{Attribute Hint Specification}
\begin{tabbing}
\grindent
attr\_hint \> : index\\
\> $|$ notnull\\
\> $|$ implementation\\
\> $|$ card
\end{tabbing}

\subsect{Index Specification}
\begin{tabbing}
\grindent
index \> : {\bf index on} attr\_path\\
\> $|$ {\bf index} {\bf $<$} index\_spec {\bf $>$} {\bf on} attr\_path\\
\> $|$ {\bf index} {\bf $<$} propagate {\bf $>$} {\bf on} attr\_path\\
\> $|$ {\bf index} {\bf $<$} index\_spec {\bf ,} propagate{\bf $>$} {\bf on} attr\_path\\
\\
index\_spec \> : {\bf btree} \\
\> $|$ {\bf hash}\\
\> $|$ {\bf type = btree}\\
\> $|$ {\bf type = hash}\\
\> $|$ {\bf hints =} hints\_spec\\
\\
hints\_spec \> : [{\bf key\_count = $<$integer$>$;}]\\
\>[{\bf initial\_size = $<$integer$>$;}]\\
\>[{\bf initial\_object\_count = $<$integer$>$;}]\\
\>[{\bf extend\_coef = $<$integer$>$;}]\\
\>[{\bf size\_max = $<$integer$>$;}]\\
\>[{\bf key\_function = $<$class$>$::$<$method$>$;}]\\
\>[{\bf dataspace = }\ident{\bf ;}]
\end{tabbing}

\subsect{Unique Specification}
\begin{tabbing}
\grindent
unique \> : {\bf constraint $<$ unique $>$}
\end{tabbing}

\subsect{Notnull Specification}
\begin{tabbing}
\grindent
notnull \> : {\bf constraint $<$ notnull $>$}
\end{tabbing}

\subsect{Implementation Specification}
\begin{tabbing}
\grindent
implementation \> : {\bf implementation} {\bf $<$} index\_spec {\bf $>$} {\bf on} attr\_path\\
\>$|$ {\bf implementation} {\bf $<$} propagate {\bf $>$} {\bf on} attr\_path\\
\>$|$ {\bf implementation} {\bf $<$} index\_spec {\bf, }propagate {\bf $>$} {\bf on} attr\_path
\end{tabbing}

\sect{Propagate Specification}
\begin{tabbing}
\grindent
propagate \> : {\bf propagate} {\bf = on}\\
\>$|$ {\bf propagate} {\bf = off}
\end{tabbing}

\sect{Attribute Path Specification}
\begin{tabbing}
\grindent
attr\_path \> : \ident\\
\> $|$ attr\_path {\bf .} \ident\\
\> $|$ \ident {\bf ::} \ident
\end{tabbing}

\sect{Inverse Specification}
\begin{tabbing}
\grindent
inverse \> : {\bf inverse} \ident {\bf ::} \ident\\
\> $|$ {\bf inverse} \ident
\end{tabbing}

\sect{Operation Specification}
\begin{tabbing}
\grindent
op\_dcl \> : trigger\_dcl {\bf ;}\\
\> $|$ method\_dcl {\bf ;}
\end{tabbing}

\sect{Method Specification}
\begin{tabbing}
\grindent
method\_hints \> : {\bf method}\\
\> $|$ {\bf instmethod}\\
\> $|$ {\bf classmethod}\\
\> $|$ {\bf static}\\
\\
method\_dcl \> : method\_hints client\_server method\_dcl\_base {\bf extref} {\bf :} string\\
\> $|$ method\_hints client\_server method\_dcl\_base\\
\> $|$ method\_dcl\_base\\
\\
method\_dcl\_base \> : \ident ref \ident {\bf (} arg\_list {\bf )}\\
\> $|$ \ident \ident {\bf (} arg\_list {\bf )}\\
\> $|$ \ident {\bf [} {\bf ]} \ident {\bf (} arg\_list {\bf )}\\
\> $|$ \ident ref {\bf [} {\bf ]} \ident {\bf (} arg\_list {\bf )}\\
\\
client\_server \> : {\bf $<$} {\bf client} {\bf $>$}\\
\> $|$ {\bf $<$} {\bf server} {\bf $>$}\\
\> $|$\\
\\
arg\_list \> : arg\_list {\bf ,} in\_out arg [\ident]\\
\> $|$ in\_out arg [\ident]\\
\> $|$\\
\\
in\_out \> : {\bf in}\\
\> $|$ {\bf out}\\
\> $|$ {\bf inout}\\
\\
arg \> : \ident\\
\> $|$ \ident ref\\
\> $|$ \ident {\bf [} {\bf ]}\\
\> $|$ \ident ref {\bf [} {\bf ]}
\end{tabbing}

\sect{Trigger Specification}
\begin{tabbing}
\grindent
trigger\_dcl \> : {\bf trigger} {\bf $<$} \ident {\bf $>$} \ident {\bf (} {\bf )}\\
\> $|$ {\bf trigger} \ident {\bf $<$} \ident {\bf $>$} {\bf (} {\bf )}
\end{tabbing}

\sect{Integer Expression}
\begin{tabbing}
\grindent
expr\_int \> : expr\_int {\bf +} expr\_int\\
\> $|$ expr\_int {\bf -} expr\_int\\
\> $|$ expr\_int {\bf *} expr\_int\\
\> $|$ expr\_int {\bf /} expr\_int\\
\> $|$ expr\_int {\bf $<$$<$} expr\_int\\
\> $|$ expr\_int {\bf $>$$>$} expr\_int\\
\> $|$ expr\_int {\bf \%} expr\_int\\
\> $|$ expr\_int {\bf $|$} expr\_int\\
\> $|$ expr\_int {\bf \^{ }} expr\_int\\
\> $|$ expr\_int {\bf \&} expr\_int\\
\> $|$ {\bf \~{ }} expr\_int\\
\> $|$ {\bf !} expr\_int\\
\> $|$ {\bf -} expr\_int\\
\> $|$ {\bf (} expr\_int {\bf )}\\
\> $|$ {\bf $<$integer$>$}\\
\> $|$ {\bf $<$character$>$}
\end{tabbing}

\sect{String Specification}
\begin{tabbing}
\grindent
string \> : string {\bf $<$string$>$}\\
\> $|$ {\bf $<$string$>$}
\end{tabbing}

\sect{Code Inclusion}
\begin{tabbing}
\grindent
type\_spec \> : {\bf $<$quoted\_seq$>$}\\
\\
attr\_dcl \> : $|$ {\bf $<$quoted\_seq$>$}
\end{tabbing}
\subsect{Examples}
\begin{tabbing}
cla\=ss \= Per\=son \{\\
\>attribute int age;\\
\>attribute char firstname[];\\
\\
\>void perform(in string, out float);\\
\\
\>{\bf \%C++\{}\\
\>\>virtual int f(int a, float b);\\
\>\>static int list\_count;\\
\>\%{\bf \}}\\
\\
\>{\bf \%Java\{}\\
\>\>int f(int a, float b) \{\\
\>\>\>return a+(int)b;\\
\>\>\}\\
\>\>static int list\_count = 3;\\
\>\%{\bf \}}\\
\};
\end{tabbing}
\sect{The eyedbodl compiler}

\begin{tabbing}
eyedbodl \={\bf -gencode} C++ {\bf -package-name} \emph{package}\\
         \>[{\bf -output-dir} \emph{dirname}] [{\bf -output-file-prefix} \emph{prefix}]\\
         \>[{\bf -schema-name} \emph{schname}]\\
         \>[{\bf -class-prefix} \emph{prefix}] [{\bf -db-class-prefix} \emph{dbprefix}\\
         \>[{\bf -attr-style} implicit$|$explicit]\\
         \>[{\bf -gen-class-stubs]} [{\bf -class-enums} yes$|$no]\\
         \>[{\bf -c-suffix} \emph{suffix}] [{\bf -h-suffix} \emph{suffix}]\\
         \>[{\bf -down-casting} yes$|$no] [{\bf -error-policy} status$|$exception]\\
         \>[{\bf -attr-cache} yes$|$no] [{\bf -rootclass} \emph{rootclass}] [{\bf -no-rootclass}]\\
         \>[{\bf -cpp} \emph{cpp}] [{\bf -cpp-flags} \emph{flags}] [{\bf -no-cpp]} \emph{odlfile}$|${\bf -$|$-db} \emph{dbname}\\
\\
eyedbodl \>{\bf -gencode} Java {\bf -package-name} \emph{package}\\
         \>[{\bf -output-dir} \emph{dirname}] [{\bf -output-file-prefix} \emph{prefix}]\\
         \>[{\bf -schema-name} \emph{schname}]\\
         \>[{\bf -class-prefix} \emph{prefix}] [{\bf -db-class-prefix} \emph{dbprefix}\\
         \>[{\bf -attr-style} implicit$|$explicit]\\
         \>[{\bf -down-casting} yes$|$no] [{\bf -error-policy} status$|$exception]\\
         \>[{\bf -cpp} \emph{cpp}] [{\bf -cpp-flags} \emph{flags}] [{\bf -no-cpp]} \emph{odlfile}$|${\bf -$|$-db} \emph{dbname}\\
\\
eyedbodl \>{\bf -gencode} CORBA$|$IDL {\bf -package-name} \emph{package} {\bf -idl-module} \emph{module}\\
         \>[{\bf -imdl} \emph{imdlfile}] [{\bf -output-dir} \emph{dirname}] [{\bf -no-generic-idl}]\\
         \>[{\bf -generic-idl} \emph{idlfile}] [{\bf -class-prefix} \emph{prefix}]\\
         \>[{\bf -no-factory]} [{\bf -sync} yes$|$no] [{\bf -comments} yes$|$no]\\
         \>[{\bf -cpp} \emph{cpp}] [{\bf -cpp-flags} \emph{flags}] [{\bf -no-cpp]} \emph{odlfile}$|${\bf -$|$-db} \emph{dbname}\\
\\
eyedbodl \>{\bf -gencode} ODL {\bf -db} \emph{dbname} [{\bf -o} \emph{odlfile}]\\
\\
eyedbodl \>{\bf -diff} {\bf -db} \emph{dbname}\\
         \>[{\bf -cpp} \emph{cpp}] [{\bf -cpp-flags} \emph{flags}] [{\bf -no-cpp]} \emph{odlfile}$|${\bf -}\\
\\
eyedbodl \>{\bf -update} {\bf -db} \emph{dbname} {\bf -package-name} \emph{package}\\
         \>[{\bf -db-class-prefix} \emph{dbprefix}] [{\bf -admin}]\\
         \>[{\bf -schema-name} \emph{schname}] \emph{odlfile}$|${\bf -}\\
\\
eyedbodl \>{\bf -checkfile} \emph{odlfile}$|${\bf -}\\
\\
eyedbodl \>{\bf -help}
\end{tabbing}
\begin{tabbing}
%%{\bf -gencode} C++xxxxxxxxxxxxxxxxx\= : \=generates C++ code\kill
one must specify one and only one\= of \=the following major options:\\
{\bf -gencode} C++                 \>: generates C++ code\\
{\bf -gencode} Java                \>: generates Java code\\
{\bf -gencode} CORBA               \>: generates CORBA code (IDL and C++ stubs)\\
{\bf -gencode} IDL                 \>: generates IDL\\
{\bf -gencode} ODL                 \>: generates ODL\\
{\bf -update}                      \>: updates schema in database \emph{dbname}\\
{\bf -diff}                        \>: displays the differences between\\
                             \>\> a database schema and an odl file\\
{\bf -checkfile}                   \>: check input ODL file\\
{\bf -help}                        \>: displays the current information\\
\\
the following options must be added to the {\bf -gencode} C++ or Java option:\\
{\bf -package-name} \emph{package}      \>: package name\\
\emph{odlfile}$|${\bf -$|$-db} \emph{dbname} \>: input ODL file ({\bf -} is the standard input) or\\
                             \>\> the database which contains the schema\\
\\
the following options can be added to the {\bf -gencode} C++ or Java option:\\
{\bf -output-dir} \emph{dirname}        \>: output directory for generated files\\
{\bf -output-file-prefix} \emph{prefix} \>: ouput file prefix (default is the package name)\\
{\bf -class-prefix} \emph{prefix}       \>: prefix to be put at the begining of each runtime class\\
{\bf -db-class-prefix} \emph{prefix}    \>: prefix to be put at the begining of each database class\\
{\bf -attr-style} implicit         \>: attribute methods have the attribute name\\
{\bf -attr-style} explicit         \>: attribute methods have the attribute name\\
                             \>\> prefixed by get/set (default)\\
{\bf -schema-name} \emph{schname}   \>: schema name (default is \emph{package})\\
{\bf -down-casting} yes            \>: generates the down casting methods (the default)\\
{\bf -down-casting} no             \>: does not generate the down casting methods\\
{\bf -attr-cache} yes              \>: generates attribute cache instance variable\\
{\bf -attr-cache} no               \>: does not generate attribute cache instance variable\\
\>\> (the default)\\
\\
for the {\bf -gencode} C++ option only:\\
{\bf -c-suffix} \emph{suffix}           \>: use \emph{suffix} as the C file suffix\\
{\bf -h-suffix} \emph{suffix}           \>: use \emph{suffix} as the H file suffix\\
{\bf -gen-class-stubs}             \>: generates a file class\_stubs.h for each class\\
{\bf -class-enums} yes             \>: generates enums within a class\\
{\bf -class-enums} no              \>: do not generate enums within a class (default)\\
{\bf -error-policy} status         \>: status oriented error policy (the default)\\
{\bf -error-policy} exception      \>: exception oriented error policy\\
{\bf -rootclass} \emph{rootclass}    \>: use \emph{rootclass} name for the root class instead\\
                             \>\>  of the package name.\\
{\bf -no-rootclass}                \>: does not use any root class\\
\\
the following options must be added to the {\bf -gencode} CORBA or IDL option:\\
{\bf -package-name} \emph{package}      \>: package name\\
{\bf -idl-module} \emph{module}         \>: the IDL module name\\
\\
one of the following options must be added to the {\bf -gencode} CORBA option:\\
\emph{odlfile}$|${\bf -$|$-db} \emph{dbname} \>: input ODL file ({\bf -} is the standard input) or\\
                             \>\>the database which contains the schema\\
\\
the following options can be added to the {\bf -gencode} CORBA or IDL option:\\
{\bf -output-dir} \emph{dirname}        \>: output directory for generated files\\
{\bf -imdl} \emph{imdlfile}$|${\bf -}           \>: IMDL file ({\bf -} is the standard input)\\
{\bf -no-generic-idl}              \>: the EyeDB generic IDL file eyedb.idl will not\\
                             \>\>  be automatically included in the IDML file\\
{\bf -generic-idl} \emph{idlfile}       \>: give the location of the EyeDB generic IDL file eyedb.idl\\
{\bf -no-factory}                  \>: the factory interface will not be generated\\
{\bf -sync} yes$|$no                 \>: the backend runtime objects will be synchronized\\
                             \>\>  (yes: the default), or not synchronized (no)\\
                             \>\>  with the database.\\
{\bf -comments} yes$|$no             \>: does (yes) or does not (no) generate mapping\\
                              \>\>comments in the IDL\\
\\
the following option must be added to the {\bf -update} option:\\
{\bf -db} \emph{dbname}                 \>: the database for which operation is performed\\
\emph{odlfile}$|${\bf -}                  \>: input ODL file ({\bf -} is the standard input)\\
\\
the following options can be added to the {\bf -update} option:\\
{\bf -schema-name} \emph{schname}       \>: schema name (default is package)\\
{\bf -db-class-prefix} \emph{prefix}    \>: prefix to be put at the begining of each database class\\
{\bf -admin}                       \>: opens the database in admin mode.\\
\\
the following options must be added to the {\bf -diff} option:\\
{\bf -db} \emph{dbname}                 \>: the database for which the schema\\
                             \>\>  difference is performed\\
\emph{odlfile}                    \>: the input ODL file for which the schema\\
                               difference is performed\\
\\
the following option must be added to the {\bf -checkfile} option:\\
\emph{odlfile}$|${\bf -}                  \>: input ODL file ({\bf -} is the standard input)\\
\\
the following options can be added when an \emph{odlfile} is set:\\
{\bf -cpp} \emph{cpp}                   \>: uses \emph{cpp} preprocessor instead of\\
                             \>\> the default one\\
{\bf -cpp-flags} \emph{cpp{\bf -flags}}       \>: adds \emph{cpp{\bf -flags}} to the preprocessing command\\
{\bf -no-cpp}                      \>: does not use any preprocessor.
\end{tabbing}


\end{document}
