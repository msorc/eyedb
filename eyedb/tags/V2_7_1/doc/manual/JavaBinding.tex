\documentclass[dvips,10pt]{book}
%\documentclass[dvips,10pt]{report}

\usepackage{latexsym}
%\usepackage{amstex}
%\usepackage{amssymb}
%\usepackage{amsfonts}
\usepackage{amstext}
\usepackage{amsbsy}
\usepackage{amscd}
\usepackage{makeidx}
\usepackage{graphicx}
%\usepackage{calc}
\usepackage{longtable}

% -- to reduce the margin pages --
\topmargin 2pt
\headsep 20pt
\headheight 2pt

%\textwidth 440pt
%\textheight 690pt
%\oddsidemargin 0pt
%\evensidemargin 0pt

% 19/01/06: EV
\textwidth 490pt
\textheight 700pt
\oddsidemargin -20pt
\evensidemargin -20pt

% --------------------------------

\makeindex

\newcommand{\eyedb}{\textsc{EyeDB} }
\newcommand{\eyedbX}{\textsc{EyeDB}}
\newcommand{\sysra}{SYSRA}
\newcommand{\indexentry}[2]{\item #1 #2}
\newcommand{\chap}[1]{\chapter{#1} \index{#1}}
\newcommand{\sect}[1]{\section{#1} \index{#1}}
\newcommand{\subsect}[1]{\subsection{#1} \index{#1}}
\newcommand{\subsubsect}[1]{\subsubsection{#1} \index{#1}}
%% WARNING \Box does not work with latex2html
%%\newcommand{\iii}{\item[{$\Box$}]}
\newcommand{\rrr}{\item[{$\Box$}]}
\newcommand{\iii}{\item}
\newcommand{\jjj}{\item[-]}
\newcommand{\bi}{\begin{itemize}}
\newcommand{\ei}{\end{itemize}}
\newcommand{\be}{\begin{enumerate}}
\newcommand{\ee}{\end{enumerate}}
\newcommand{\kkk}{\item[ ]}
\newcommand{\sh}[1]{\bf #1}
\newcommand{\ttt}{$\tilde{ }$}
\newcommand{\ttv}[1]{\texttt{#1}}
\newcommand{\ttu}[1]{{\verbsize\texttt{#1}}}
\newcommand{\verbsizex}{\small}
\newcommand{\verbsizey}{\fontsize{8}{\baselineskip}\selectfont}
\newcommand{\verbsize}{\normalsize}
\newcommand{\bugreport}{bug-report@eyedb.org }
\newcommand{\idt}{\mbox{ } - }
\newcommand{\ident}{{\bf \texttt{<}identifier\texttt{>}}}
\newcommand{\grindent}{xxxxxxxxxxxxxxx \= : \kill}
\newcommand{\arr}{\texttt{-$>$}}
\newcommand{\ul}[1]{\underline{#1}}
\newcommand{\uul}[1]{\textit{\texttt{#1}}}
\newcommand{\ixx}{\makebox[3mm]{ }}
\newcommand{\ixy}{\makebox[10mm]{ }}
\newcommand{\ixz}{\\\ixy {- }}

\renewcommand{\normalsize}{\fontsize{9}{11pt}\selectfont}

%\renewcommand{\normalsize}{\fontsize{8}{10pt}\selectfont}

%% The following commands comes from corbaref.tex

\newcommand{\bt}{\begin{tabular}{|p{7cm}|p{7cm}|}}
\newcommand{\et}{\hline \end{tabular} \\ {\vspace{0.6cm}} \\ }

\newcommand{\mt}[1]{\hline \multicolumn{2}{|c|}{\centerline{{\bf interface #1}
   \emph{mapped from} {\bf class #1}}}}

\newcommand{\Desc}[1]{\hline \multicolumn{2}{|l|}{\emph{#1}}}

\newcommand{\mTT}[2]{\hline \multicolumn{2}{|c|}
   {\centerline{{\bf interface #1} \emph{extends} {\bf #2}
   \emph{mapped from} {\bf #1}}}}


\newcommand{\mT}[2]{\hline \multicolumn{2}{|c|}{\centerline{{\bf #1}
\emph{mapped from} {\bf #1}}}}

\newcommand{\att}{\hline \hline {\bf Attribute} & {\bf Description}}
\newcommand{\mtt}{\hline \hline {\bf Method} & {\bf Description}}
\newcommand{\mf}{\emph{mapped from }}

\newcommand{\satt}[1]{\hline \multicolumn{2}{|c|}
   {\centerline{\emph{attributes inherited from} {\bf #1}}}}

\newcommand{\smtt}[1]{\hline \multicolumn{2}{|c|}{
   \centerline{\emph{methods inherited from} {\bf #1}}}}

\newcommand{\sets}[1]{sets the #1 \emph{value} in the Object \emph{o} at the \emph{ind}}

\newcommand{\Gets}[1]{gets the #1 value in the Object \emph{o} at the \emph{ind}}

%\setcounter{secnumdepth}{-1}
%\addtocounter{secnumdepth}{2}

\renewcommand{\thesection}{\arabic{section}}
\renewcommand{\thesubsection}{\arabic{section}.\arabic{subsection}}
\renewcommand{\thechapter}{}

\begin{document}
\normalsize


\input{version}
\newcommand{\mantitle}{\textsc{Java Binding}}

\thispagestyle{empty}

\mbox{ }
\\
\vspace{3 cm}
\\
%{\bf{\Huge E{\LARGE YE}DB} \hspace{0.cm} {\LARGE Programming Manual}}
{\bf{\Huge E{\LARGE YE}DB} \hspace{0.cm} {\Huge \mantitle}}
\\
%\newcommand{\rulewidth}{12.2cm}
\newcommand{\rulewidth}{15.4cm}
\rule{\rulewidth}{1.6mm}
\begin{flushright}
{\large Version \eyedbversion}
\end{flushright}
\mbox{ }
\\
\vspace{11 cm}
\\
\begin{flushright}
{\large January 2006}
\end{flushright}

\newpage

\thispagestyle{empty}

\mbox{ }
\\
\vspace{1cm}
\\
Copyright {\copyright} 2001-2006 SYSRA
\\
\vspace{1cm}
\\
Published by \sysra
\\
30, avenue G\'en\'eral Leclerc
\\
91330 Yerres - France
\\
\\
home page: http://www.eyedb.org
\\
\\
bug report: \bugreport


\tableofcontents

\chapter{The Java Binding}

%\sect{Basic Concepts of the Java Binding}

The Java binding maps the \eyedb object model into Java by introducing
a generic API
%%The Java binding is expressed as a generic API
%%including about one hundred and fifty classes,
and a tool to generate a speficic Java API from a given schema,
built upon the generic API.
\\
\\
The generic Java API is made up of about one hundred and fifty classes such
as some abstract classes as the \texttt{object} and \texttt{class} classes
and some more concrete classes such as the \texttt{database} and \texttt{image}
classes.
\\
\\
Each type in the \eyedb object model is implemented as a Java class whithin
the Java API: there is a one for one mapping between the object model and
the Java API.
\\
This mapping follows a very simple naming schema: each Java class mapped from
a type has the name of this type prefixed by \texttt{COM.sysra.eyedb.idb}.
\\
For instance, the \texttt{object} type in the \eyedb object model is
mapped on the \texttt{COM.sysra.eyedb.idbObject} Java class.
\\
\\
%We are going to introduce the main classes and methods through some simple
%examples.
%As we have seen previously, \eyedb provides currently two language bindings: C++
%and Java.
%\\
%\\
%The C++ binding has been fully described in a previous chapter.
%\\
%\\
The use of the Java language for an \eyedb binding has been motivated by
several reasons:
\bi
\item Java is architecture independent.
\item Java is indeed valuable for distributed network environment.
\item Java has a very rich builtin library.
\item Java is secure.
\item Java is easier to program than C++.
\ei
%Contrary to C++, Java provides run-time facilities to manipulate classes
%(class {\bf Class} and the {\bf reflect} package), but it seems
%more coherent to keep the soul of the C++ binding.
%\\
%\\
%Therefore, the
The Java binding is very close from the C++ binding: the
class interfaces are identical, the functionalities
are the same; only the language is slightly different.
\sect{Getting Started}
We will introduce the Java binding through a simple example of
defining an ODL schema, generating the java stubs and then writing
a client program.
We will explain only what is necessary to understand the example
to avoid begin swamped by unnecessary details.
\\
\\
The example that we will develop in this section is the same that
has been developped in the C++ binding chapter, a couple of classes
\texttt{Person} and \texttt{Car}.
\subsect{Programming Steps}
The following programming steps are typically required to write a Java
client program dealing with the given schema:
\begin{enumerate}
\item Define the schema, using the \eyedb Object Definition Language.
\item Generate the Java stubs, using the {\bf eyedbodl} tool.
\item Write a client program using the generated Java package.
\end{enumerate}
We will illustrate these steps in the remainder of this chapter.
\subsect{The ODL Schema}
The ODL schema for our application can be described as follows:

{\verbsize \begin{verbatim}
/* person.odl */

struct Person {
  string name;
  int age;
  Person *spouse inverse Person::spouse;
  Person *father;
  set<Person *> children inverse Person::father;
  set<Car *> cars inverse Car::owner;

  index on name;
  index on age;
  constraint<notnull> on name;
  constraint<unique> on name;
};

struct Car {
  string brand;
  int num;
  Person *owner inverse Person::cars;

  index on brand;
  index on num;
};
\end{verbatim}
}
This schema describes two classes \texttt{Person} and \texttt{Car}, one 1 to 1
relationship (attribute \texttt{spouse} in class \texttt{Person}) and two 1 to many
relationships (from \texttt{Person::father} to \texttt{Person::children}
and from \texttt{Car::owner} to \texttt{Person::cars}).

\subsect{Compiling the ODL Schema}
The ODL schema must be compiled, both to check the schema and to bind
it into some Java code so that it can be used in a client program.
\\
\\
The \texttt{person.odl} file can be compiled using the following command:
{\verbsize \begin{verbatim}
% mkdir person
% eyedbodl -gencode JAVA -package person -output-dir person person.odl
\end{verbatim}
}
\subsect{The Java produced}
The ODL compiler will then produce the following Java files in the \texttt{person} directory:
\bi
\item Person.java
\item Car.java
\item Database.java
\item set\_class\_Car\_ref.java
\item set\_class\_Person\_ref.java
\ei
The Person.java file produced by the ODL compiler contains the following
(only the method interfaces are shown, not the body):
{\verbsize \begin{verbatim}
//
// class Person
//
// package person
//
// Automatically Generated by eyedbodl at ...
//

package person;

import COM.sysra.eyedb.*;

public class Person extends idbStruct {

  public Person(idbDatabase db);
  public Person(idbStruct x, boolean share);
  public Person(Person x, boolean share);

  public void setName(String _name);
  public void setName(int a0, char _name) throws idbException;
  public String getName();

  public void setAge(int _age) throws idbException;
  public int getAge()  throws idbException;

  public void setSpouse(Person _spouse) throws idbException;
  public Person getSpouse() throws idbException;
  public void setSpouse_oid(idbOid _oid) throws idbException;
  public idbOid getSpouse_oid() throws idbException;
 
  public idbCollSet getChildrenColl() throws idbException;
  public int getChildrenCount() throws idbException;
  public Person getChildrenAt(int ind) throws idbException;
  public void setChildrenColl(idbCollSet _children) throws idbException;
  public void addToChildrenColl(Person _children) throws idbException;
  public void addToChildrenColl(idbOid _oid) throws idbException;
  public void rmvFromChildrenColl(Person _children) throws idbException;nn
  public void rmvFromChildrenColl(idbOid _oid) throws idbException;nn

  // and so on. father and cars set and get methods.
  // ...

  // protected and private parts
  // ...
};
\end{verbatim}
}
eyedbodl has generated three constructors (including two copy constructors),
and get and set methods for each attributes.

\subsect{Compiling the Java stubs}
The generated code must be compiled using the \texttt{javac} compiler.
\\
The Java CLASSPATH must include both the \eyedb Java class path and
the current directory, as follows:
{\verbsize \begin{verbatim}
% CLASSPATH=$EYEDBCLASSPATH:. javac -depend -d. person/Database.java 
\end{verbatim}
}

\subsect{Writing a Client}
Here is a simple example of a Java client which opens a database,
creates 2 person instances and mary them:
{\verbsize \begin{verbatim}
//
// class TestP.java
//

import COM.sysra.eyedb.*;
import person.*;

class TestP {
  public static void main(String args[]) {

    // Initialize the eyedb package and parse the default eyedb options
    // on the command line
    String[] outargs = idbRoot.init("TestP", args);
     
    // Check that a database name is given on the command line
    int argc = outargs.length;
    if (argc != 1)
      {
        System.err.println("usage: java TestP dbname");
        System.exit(1);
      }

    try {
      // Initialize the person package
      person.Database.init();

      // Open the connection with the backend
      idbConnection conn = new idbConnection();

      // Open the database named outargs[0]
      person.Database db = new person.Database(outargs[0]);
      db.open(conn, idbDatabase.DBRW);

      db.transactionBegin();
      // Create two persons john and mary
      Person john = new Person(db);
      john.setName("john");
      john.setAge(32);
     
      Person mary = new Person(db);
      mary.setName("mary");
      mary.setAge(30);
     
      // Mary them :-)
      john.setSpouse(mary);

      // Store john and mary in the database
      john.store(idbRecMode.FullRecurs);

      db.transactionCommit();
    }
    catch(idbException e) { // Catch any eyedb exception
       e.print();
       System.exit(1);
    }
  }
}
\end{verbatim}
}
The client contains the following lines at the beginning of all
its modules:
{\verbsize \begin{verbatim}
import COM.sysra.eyedb.*;
import person.*;
\end{verbatim}
}
Those lines means that you are imported the standard \eyedb package
and the generated \texttt{person} package.
\\
\\
NOTE : those package importations are not essential, as you can refer to the
generated classes using the \texttt{person.} prefix; and that you can refer
to the standard \eyedb package using the \texttt{COM.sysra.eyedb.} prefix.
\\
\\
Before any \eyedb method call, you need to initialize the \eyedb
package by calling the \texttt{idbRoot.init} method, as follows:
{\verbsize \begin{verbatim}
      String[] outargs = idbRoot.init("TestP", args);
\end{verbatim}
}
This method will take out from \texttt{args} all the \eyedb options
such as \texttt{-eyedbhost $<$host$>$, -eyedbport $<$port$>$}
(refer to the environment chapter).
\\
\\
The returned array \texttt{outargs} will contain the command line arguments
except those that have been recognized as \eyedb options.
\\
\\
If you do not call the \texttt{idbRoot.init} method, the \eyedb java binding
will not work.
\\
\\
Then, you need to initialiaze the person generated package, as follows:
{\verbsize \begin{verbatim}
      person.Database.init()
\end{verbatim}
}
Once again, if you do not call this method, the \eyedb java binding will
not work properly.
\\
\\
Then, you need to open a connection with the \eyedb backend
{\verbsize \begin{verbatim}
      idbConnection conn = new idbConnection();
\end{verbatim}
}
The constructor \texttt{idbConnection()} will try to connect to the backend
using the host and port given on the command line.
\\
In case of failure, an \texttt{idbException} is thrown.
\\
\\
To open the database whose name is \texttt{outargs[0]}, you must first
create a \texttt{person.Database} object, then call the \texttt{open}
method on this object as follows:
{\verbsize \begin{verbatim}
      person.Database db = new person.Database(outargs[0]);
      db.open(conn, idbDatabase.DBRW);
\end{verbatim}
}
The \texttt{idbDatabase.DBRW} flag indicates that we wants to open
the database in the read/write mode.
\\
\\
Once that the database has been opened, we begin a transaction as follows:
{\verbsize \begin{verbatim}
      db.transactionBegin();
\end{verbatim}
}
This method call is necessary for any database access in read or write
mode.
\\
\\
To create two persons whose name are \texttt{john} and \texttt{mary} and
to mary them, you need to use the generated constructors and methods
as follows:
{\verbsize \begin{verbatim}
      Person john = new Person(db);
      john.setName("john");
      john.setAge(32);
     
      Person mary = new Person(db);
      mary.setName("mary");
      mary.setAge(30);
     
      john.setSpouse(mary);
\end{verbatim}
}
Note that at this step, the persons \texttt{john} and \texttt{mary} have
{\bf not} been stored in the database.
\\
Those person references are runtime references, not persistant references.
\\
\\
To store them permanently in the database:
{\verbsize \begin{verbatim}
      john.store(idbRecMode.FullRecurs);
\end{verbatim}
}
The \texttt{idbRecMode.FullRecurs} argument means that all the object
references included in the \texttt{john} object will be stored too ; so
the \texttt{mary} reference which is the \texttt{john} spouse, will be stored too.
\\
\\
At this step, if you exit the program, the current transaction will
be automatically aborted (rolled back).
\\
So the \texttt{john} and \texttt{mary} persons will not be effectively stored
in the database.
\\
\\
To commit the transaction, you must do the following:
{\verbsize \begin{verbatim}
      db.transactionCommit();
\end{verbatim}
}
Note that all the constructors and methods previously called, may
throw an \texttt{idbException} in case of failure.
\\
The following code catches the exceptions, print them and exit the program:
{\verbsize \begin{verbatim}
    catch(idbException e) {
       e.print();
       System.exit(1);
    }
\end{verbatim}
}

\subsect{Compiling the client}
The client must be compiled as follows:
{\verbsize \begin{verbatim}
% CLASSPATH=$EYEDBCLASSPATH:. javac -d . TestP.java
\end{verbatim}
}
\subsect{Updating the schema}
Before running the client for the first time, we need to create a test database
and to update its schema according to the ODL description.
\\
\\
Note that to create a database, the \eyedb user under which you are working,
must has the \texttt{dbcreate} system mode (refer to the administration
section).
\\
\\
To create a database \texttt{foo}:
\verbsize \begin{verbatim}
% eyedbdbcreate foo
user authentication : <user name>
password authentication : <user passwd>
\end{verbatim}
\normalsize
To update the schema person.odl in the database \texttt{foo}:
{\verbsize \begin{verbatim}
% eyedbodl -dbinit foo person.odl
Updating <unnamed> Schema in database foo...
Done
\end{verbatim}
}

\subsect{Running the client}
To run properly, an \eyedb program needs some environment information such
as the \eyedb host and port, the \eyedb user and password.
\\
\\
When we run a C++ program, the \eyedb runtime takes this environment
from the UNIX environment variables.
\\
\\
There is not such mechanisms for a Java program: the \texttt{getenv()} function
does not exist in Java.
\\
\\
So, we need to give all the environment using the command line arguments
as follows:
\verbsize \begin{verbatim}
% eyedbjrun foo
\end{verbatim}
\normalsize
After the program has run, let's verify that the objects have really been
created in the \texttt{foo} database, using the \texttt{eyedboql} tool:
{\verbsize \begin{verbatim}
% eyedboql
Welcome to eyedboql. Type `!help' to display the command list.
? !open foo
? select Person;
47886.20.803967:oid, 47887.20.2361599:oid
? !print full
47886.20.803967:oid Person = { 
        name = "john";
        age = 32;
        *spouse 47887.20.2361599:oid Person = { 
                name = "mary";
                age = 30;
                *spouse 47886.20.803967:oid Person = { 
                        <trace cycle>
                };
                *father NULL;
                *children NULL;
                *cars NULL;
        };
        *father NULL;
        *children NULL;
        *cars NULL;
};
47887.20.2361599:oid Person = { 
        name = "mary";
        age = 30;
        *spouse 47886.20.803967:oid Person = { 
                name = "john";
                age = 32;
                *spouse 47887.20.2361599:oid Person = { 
                        <trace cycle>
                };
                *father NULL;
                *children NULL;
                *cars NULL;
        };
        *father NULL;
        *children NULL;
        *cars NULL;
};
? !quit
\end{verbatim}
}
As there is a unique constraint on the Person \texttt{name} attribute,
if you run the program again, you will catch the following exception:
{\verbsize \begin{verbatim}
% TestP.run foo
COM.sysra.eyedb.idbStoreException: unique[] constraint error : attribute
`name' in the agregat class `Person'
\end{verbatim}
}
\subsect{An advanced example}
You can use \eyedb OQL withing the Java Binding, using the \texttt{idbOQL}
class.
\\
\\
For instance, the following program looks for the \texttt{Person} instance whose
name is given on the command line.
\\
Then, it looks for the \texttt{Person} instances whose age is
less than 3 years old, adds them to the children collection of the previous
\texttt{Person} instance, and increments their age.
{\verbsize \begin{verbatim}
//
// class TestPC.java
//

import COM.sysra.eyedb.*;
import person.*;

class TestPC {
  public static void main(String args[]) {

    // Initialize the eyedb package and parse the default eyedb options
    // on the command line
    String[] outargs = idbRoot.init("TestPC", args);
     
    // Check that a database name is given on the command line
    int argc = outargs.length;
    if (argc != 2)
      {
        System.err.println("usage: java TestPC dbname person-name");
        System.exit(1);
      }

    try {
      // Initialize the person package
      person.Database.init();

      // Open the connection with the backend
      idbConnection conn = new idbConnection();

      // Open the database named outargs[0]
      person.Database db = new person.Database(outargs[0]);
      db.open(conn, idbDatabase.DBRW);

      db.transactionBegin();
      // Looks for the Person john

      String pname = outargs[1];
      idbOQL q = new idbOQL(db, "select Person.name = \"" + pname + "\"");

      idbObjectArray obj_array = new idbObjectArray();

      q.execute(obj_array);

      if (obj_array.getCount() == 0)
        {
          System.err.println("TestPC: cannot find person `" + pname + "'");
          System.exit(1);
        }

      Person john = (Person)obj_array.getObjects()[0];

      // Looks for Person whose age is less than 3
      // and add them to the john children collection

      q = new idbOQL(db, "select Person.age < 3");

      obj_array = new idbObjectArray();

      q.execute(obj_array);

      for (int i = 0; i < obj_array.getCount(); i++)
        {
          Person child = (Person)obj_array.getObjects()[i];
          child.setAge(child.getAge() + 1);
          john.addToChildrenColl(child);
        }

      // Update john and its children in the database
      john.store(idbRecMode.FullRecurs);

      db.transactionCommit();
    }
    catch(idbException e) { // Catch any eyedb exception
       e.print();
       System.exit(1);
    }
  }
}
\end{verbatim}
}
The following steps (and code) are identical to the previous example:
\bi
\item importing packages \texttt{person} and \texttt{COM.sysra.eyedb}
\item initializing \eyedb package and \texttt{person} packages
\item opening the connection
\item opening the database
\item beginning a transaction
\ei
The first difference comes using the \texttt{idbOQL} constructor call:
{\verbsize \begin{verbatim}
      String pname = outargs[1];
      idbOQL q = new idbOQL(db, "select Person.name = \"" + pname + "\"");
\end{verbatim}
}
This constructor makes a \eyedb OQL query in the opened database.
\\
\\
Nothing is returned from this query, but an exception is thrown if the query
is invalid.
\\
\\
To get back the result of this query (i.e. the Person whose \texttt{name}
is \texttt{pname}), one needs to scan the query.
\\
\\
Several \texttt{idbOQL} methods allows us to scan the query:
{\verbsize \begin{verbatim}
  public void scan(idbValueArray value_array)
    throws idbTransactionException;

  public void scan(idbOidArray oid_array)
    throws idbTransactionException;

  public void scan(idbObjectArray obj_array)
    throws idbTransactionException, idbLoadObjectException;

  public void scan(idbObjectArray obj_array, idbRecMode rcm)
    throws idbTransactionException, idbLoadObjectException;

  public void scan(idbValueArray value_array, int count, int start)
    throws idbTransactionException;

  public void scan(idbOidArray oid_array, int count, int start)
    throws idbTransactionException;

  public void scan(idbObjectArray obj_array, int count, int start)
    throws idbTransactionException, idbLoadObjectException;

  public void scan(idbObjectArray obj_array, int count, int start,
                   idbRecMode rcm)
    throws idbTransactionException, idbLoadObjectException;
\end{verbatim}
}
The \texttt{scan} methods which deals with a \texttt{idbValueArray} are
the most general.
\\
\\
These methods allows us to get back anything: integer, float, string, char,
oid.
\\
For instance, if you make a query such as:
{\verbsize \begin{verbatim}
     idbOQL q = new idbOQL(db, \"list(1, 2, 3, "hello world",
                               3.45, 'c', (select Person.age < 3));
\end{verbatim}
}
the returned values will be:
\bi
\item 3 integers
\item 1 string
\item 1 float
\item 1 character
\item some Person oids
\ei
In this case, it seems raisonnable to use a \texttt{scan} method which
deals with a \texttt{idbValueArray}, as follows:
{\verbsize \begin{verbatim}
     idbValueArray valarr = new idbValueArray();
     q.execute(valarr);   

     for (int i = 0; i < valarr.getCount(); i++)
      {
         idbValue value = valarr.getValues()[i];
         if (value.type == idbValue.INT)
           // ...
         else if (value.type == idbValue.FLOAT)
           // ...
         else if (value.type == idbValue.OID)
           // ...
         else if (value.type == idbValue.CHAR)
           // ...
      }
\end{verbatim}
}
In the \texttt{TestPC.java} example, as we know that only objects could
be returned, a more simple \texttt{scan} method may be used, as follows:
{\verbsize \begin{verbatim}
      idbOQL q = new idbOQL(db, "select Person.name = \"" + pname + "\"");

      idbObjectArray obj_array = new idbObjectArray();

      q.execute(obj_array);

      if (obj_array.getCount() == 0)
        {
          System.err.println("TestPC: cannot find person `" + pname + "'");
          System.exit(1);
        }
\end{verbatim}
}
In the case of no objects has been found, we display an error message and
then leaves the program.
\\
\\
In the case of, at least, one object has been found, we cast the \texttt{idbObject} instance in a Person instance, as follows:
{\verbsize \begin{verbatim}
      Person john = (Person)obj_array.getObjects()[0];
\end{verbatim}
}
This cast is a reasonnable cast for several reasons:
\begin{enumerate}
\item As the query is \texttt{"Person.name = \\"" + pname + "\\""}, the expected
returned values are \texttt{Person} instances.
\item As the database has been opened using an instance
of the \texttt{person.Database} class, a \texttt{Person} instance has been
actually constructed using the generated stubs.
\\
If we had used an instance of the generic \texttt{idbDatabase} class, this
cast would has been illegal, as the \texttt{Person} class is unknown from
the generic \eyedb package!
\item Contrary to C++, Java performs secure dynamic casting, so if the
returned object is not a \texttt{Person} instance, an invalid cast exception
will be thrown.
\end{enumerate}
Then:
\bi
\item we look for the \texttt{Person} instances whose \texttt{age} is less
than \texttt{3}:
{\verbsize \begin{verbatim}
      q = new idbOQL(db, "select Person.age < 3");

      obj_array = new idbObjectArray();

      q.execute(obj_array);
\end{verbatim}
}
\item one increments their \texttt{age}.
{\verbsize \begin{verbatim}
      for (int i = 0; i < obj_array.getCount(); i++)
        {
          Person child = (Person)obj_array.getObjects()[i];
          child.setAge(child.getAge() + 1);
\end{verbatim}
}
\item one adds them to the \texttt{children} collection of the previous
\texttt{Person} instance found.
{\verbsize \begin{verbatim}
          john.addToChildrenColl(child);
        }
\end{verbatim}
}
\item one stores all in the database.
{\verbsize \begin{verbatim}
      john.store(idbRecMode.FullRecurs);
\end{verbatim}
}
\item one commits the transaction.
{\verbsize \begin{verbatim}
      db.transactionCommit();
\end{verbatim}
}
\ei
\sect{The \eyedb Java API Reference Manual}
The \eyedb Java API Reference Manual is available online in the file:
\\
\$EYEDBROOT/java/doc/packages.html


\end{document}
