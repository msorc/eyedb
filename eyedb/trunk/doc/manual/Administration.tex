\documentclass[dvips,10pt]{book}
%\documentclass[dvips,10pt]{report}

\usepackage{latexsym}
%\usepackage{amstex}
%\usepackage{amssymb}
%\usepackage{amsfonts}
\usepackage{amstext}
\usepackage{amsbsy}
\usepackage{amscd}
\usepackage{makeidx}
\usepackage{graphicx}
%\usepackage{calc}
\usepackage{longtable}

% -- to reduce the margin pages --
\topmargin 2pt
\headsep 20pt
\headheight 2pt

%\textwidth 440pt
%\textheight 690pt
%\oddsidemargin 0pt
%\evensidemargin 0pt

% 19/01/06: EV
\textwidth 490pt
\textheight 700pt
\oddsidemargin -20pt
\evensidemargin -20pt

% --------------------------------

\makeindex

\newcommand{\eyedb}{\textsc{EyeDB} }
\newcommand{\eyedbX}{\textsc{EyeDB}}
\newcommand{\sysra}{SYSRA}
\newcommand{\indexentry}[2]{\item #1 #2}
\newcommand{\chap}[1]{\chapter{#1} \index{#1}}
\newcommand{\sect}[1]{\section{#1} \index{#1}}
\newcommand{\subsect}[1]{\subsection{#1} \index{#1}}
\newcommand{\subsubsect}[1]{\subsubsection{#1} \index{#1}}
%% WARNING \Box does not work with latex2html
%%\newcommand{\iii}{\item[{$\Box$}]}
\newcommand{\rrr}{\item[{$\Box$}]}
\newcommand{\iii}{\item}
\newcommand{\jjj}{\item[-]}
\newcommand{\bi}{\begin{itemize}}
\newcommand{\ei}{\end{itemize}}
\newcommand{\be}{\begin{enumerate}}
\newcommand{\ee}{\end{enumerate}}
\newcommand{\kkk}{\item[ ]}
\newcommand{\sh}[1]{\bf #1}
\newcommand{\ttt}{$\tilde{ }$}
\newcommand{\ttv}[1]{\texttt{#1}}
\newcommand{\ttu}[1]{{\verbsize\texttt{#1}}}
\newcommand{\verbsizex}{\small}
\newcommand{\verbsizey}{\fontsize{8}{\baselineskip}\selectfont}
\newcommand{\verbsize}{\normalsize}
\newcommand{\bugreport}{bug-report@eyedb.org }
\newcommand{\idt}{\mbox{ } - }
\newcommand{\ident}{{\bf \texttt{<}identifier\texttt{>}}}
\newcommand{\grindent}{xxxxxxxxxxxxxxx \= : \kill}
\newcommand{\arr}{\texttt{-$>$}}
\newcommand{\ul}[1]{\underline{#1}}
\newcommand{\uul}[1]{\textit{\texttt{#1}}}
\newcommand{\ixx}{\makebox[3mm]{ }}
\newcommand{\ixy}{\makebox[10mm]{ }}
\newcommand{\ixz}{\\\ixy {- }}

\renewcommand{\normalsize}{\fontsize{9}{11pt}\selectfont}

%\renewcommand{\normalsize}{\fontsize{8}{10pt}\selectfont}

%% The following commands comes from corbaref.tex

\newcommand{\bt}{\begin{tabular}{|p{7cm}|p{7cm}|}}
\newcommand{\et}{\hline \end{tabular} \\ {\vspace{0.6cm}} \\ }

\newcommand{\mt}[1]{\hline \multicolumn{2}{|c|}{\centerline{{\bf interface #1}
   \emph{mapped from} {\bf class #1}}}}

\newcommand{\Desc}[1]{\hline \multicolumn{2}{|l|}{\emph{#1}}}

\newcommand{\mTT}[2]{\hline \multicolumn{2}{|c|}
   {\centerline{{\bf interface #1} \emph{extends} {\bf #2}
   \emph{mapped from} {\bf #1}}}}


\newcommand{\mT}[2]{\hline \multicolumn{2}{|c|}{\centerline{{\bf #1}
\emph{mapped from} {\bf #1}}}}

\newcommand{\att}{\hline \hline {\bf Attribute} & {\bf Description}}
\newcommand{\mtt}{\hline \hline {\bf Method} & {\bf Description}}
\newcommand{\mf}{\emph{mapped from }}

\newcommand{\satt}[1]{\hline \multicolumn{2}{|c|}
   {\centerline{\emph{attributes inherited from} {\bf #1}}}}

\newcommand{\smtt}[1]{\hline \multicolumn{2}{|c|}{
   \centerline{\emph{methods inherited from} {\bf #1}}}}

\newcommand{\sets}[1]{sets the #1 \emph{value} in the Object \emph{o} at the \emph{ind}}

\newcommand{\Gets}[1]{gets the #1 value in the Object \emph{o} at the \emph{ind}}

%\setcounter{secnumdepth}{-1}
%\addtocounter{secnumdepth}{2}

\renewcommand{\thesection}{\arabic{section}}
\renewcommand{\thesubsection}{\arabic{section}.\arabic{subsection}}
\renewcommand{\thechapter}{}

\begin{document}
\normalsize


\input{version}
\newcommand{\mantitle}{\textsc{Administration}}

\thispagestyle{empty}

\mbox{ }
\\
\vspace{3 cm}
\\
%{\bf{\Huge E{\LARGE YE}DB} \hspace{0.cm} {\LARGE Programming Manual}}
{\bf{\Huge E{\LARGE YE}DB} \hspace{0.cm} {\Huge \mantitle}}
\\
%\newcommand{\rulewidth}{12.2cm}
\newcommand{\rulewidth}{15.4cm}
\rule{\rulewidth}{1.6mm}
\begin{flushright}
{\large Version \eyedbversion}
\end{flushright}
\mbox{ }
\\
\vspace{11 cm}
\\
\begin{flushright}
{\large January 2006}
\end{flushright}

\newpage

\thispagestyle{empty}

\mbox{ }
\\
\vspace{1cm}
\\
Copyright {\copyright} 2001-2006 SYSRA
\\
\vspace{1cm}
\\
Published by \sysra
\\
30, avenue G\'en\'eral Leclerc
\\
91330 Yerres - France
\\
\\
home page: http://www.eyedb.org
\\
\\
bug report: \bugreport


\tableofcontents

\chapter*{Administration and Security Guide}

The chapter introduces the way to administrate properly \eyedb under
a UNIX based operating system.
\eyedb administration is simple as only very few commands are necessary to
make it work properly: starting and stopping the server, creating and deleting
a database, adding a user and granting him access to a database.
Those very basic command have already been presented in the
\emph{Getting Started} document.
\\
This chapter introduces all the concepts and operations concerning
the administration and the security of \eyedbX.
The distribution organizing is presented in section 1. Section 2
introduces the configuration of the system.
The server management is exposed in section 3.
Section 4 introduces database concepts and operations:
database files, the EYEDBDBM database, database management.
Section 5 introduces the \eyedb user concepts and management.
Section 6 exposed the way to control system and database access.
Storage manager low level administration is presented in section 7.
The last section introduces common trouble shotting and the
way to recover them.

\newcommand{\tdir}{\emph{topdir}}
\newcommand{\constvar}{The value of this variable cannot be changed}
\newcommand{\modvar}{The value of this variable can be modified}
\newcommand{\esc}[1]{\ttv{$\backslash$#1}}

\section{The Distribution}
\subsection{File Hierarchy}
In this chapter, we denote by {\tdir } the target directory that has been given
during the installation step as follows:\\
\\
\mbox{ }\ttv{eyedb-install} {\tdir } \emph{licensefile}\\
\\
for instance
\\
\\
\mbox{ }\ttv{eyedb-install /usr/local/eyedb license.dat}
\\
\\
Under the directory \tdir, the install shell script has created the
directories and files: \ttv{db}, \ttv{log}, \ttv{eyedb-V\eyedbversion} and \ttv{eyedb}:
\bi
\item \ttv{db}: contains the database files,
\item \ttv{etc}: contains the license file, a startup shell script
and environment files,
\item \ttv{log}: will contains the output log files,
\item \ttv{eyedb-V\eyedbversion}: contains the whole \eyedbversion \eyedb distribution,
\item \ttv{eyedb}: a symbolic link to the current distribution, currently
\ttv{eyedb-V\eyedbversion}.
When you will perform an update to a new version, for instance \eyedbversion,
you will need to change this symbolic link to point to
the directory \ttv{eyedb-V{\eyedbversion}}.
\ei
We present now the contents of each of the previous directories.
\subsubsection{The \tdir/\ttv{db} directory}
The \ttv{db} directory contains the database files. After the installation,
this directory contains the database files of the main \eyedb database,
\textsc{EYEDBDBM}: \ttv{dbmdb.db}, \ttv{dbmdb.log}, \ttv{dbmdb.omp} and
\ttv{dbmdb.vol}. When you create, for instance, the database \ttv{person}
the following files
appears in this directory: \ttv{person.db}, \ttv{person.log},
\ttv{person.omp} and \ttv{person.vol}. Refer to the \emph{Database} section
to get the meaning of those files.

\subsubsection{The \tdir/\ttv{etc} directory}
This directory contains the following files:
\bi
\item \ttv{environ.sh}: if you are using a Bourne shell, you need to execute
this shell script to set the correct environment for using \eyedb, as follows:
\\
. \tdir/\ttv{etc/environ.sh}
\\
The only operation performed by this shell script is the execution of
another shell script \tdir/\ttv{eyedb/etc/environ.sh}.
This allows one to set the environment of the current
installed version of \eyedbX: indeed,
remember that \tdir/\ttv{eyedb} is a symbolic link to the current version
\tdir/\ttv{eyedb-V\eyedbversion}. Furthermore, when you update \eyedbX, you change the
\ttv{eyedb} symbolic link to point to the new version of \eyedbX, for instance
\ttv{eyedb-V{\eyedbversion}}.\\
Therefore, the shell script \tdir/\ttv{etc}/environ.sh will execute
the shell script \tdir/\ttv{eyedb/etc/environ.sh} which is in fact the file\\
\tdir/\ttv{eyedb-V{\eyedbversion}/etc/environ.sh}
\item \ttv{environ.csh}: this file has the same purpose as \ttv{environ.sh}
but must be used if you are using a C-shell:
\\
source \tdir/\ttv{etc/environ.csh}
\item \ttv{license.dat}: this file contains the license information.
\item \ttv{rc.eyedb}: this shell script is used to launched the \eyedb servers.
The only action performed by this shell script is the execution
of another shell script \tdir/\ttv{eyedb/etc/rc.eyedb}.
This allows on to launch the \ttv{rc.eyedb} shell script of the
current installed version. 
\item \ttv{passwd}: this file contains a crypted password necessary
to create a \textsc{EYEDBDBM} database.
\ei
\subsubsection{The \tdir/\ttv{log} directory}
After the installation, this directory is empty. The launcher script
\tdir/\ttv{eyedb/etc/rc.eyedb} redirects the log messages into files
in this directory.
\subsubsection{The \tdir/\ttv{eyedb-V\eyedbversion} directory}
This directory contains the directories and files
\ttv{bin}, \ttv{etc}, \ttv{include}, \ttv{lib}, \ttv{doc},
\ttv{examples}, \ttv{java} and \ttv{VERSION}:
\bi
\item \ttv{bin}: this directory contains all the \eyedb programs
divided into different families:
\be
\item Server management programs:
\bi
\item \ttv{eyedbd}: the \eyedb server,
\item \ttv{eyedbctl}: the \eyedb launcher used to start or stop
the server or to get status information on the opened sessions.
\ei
\item Generic administration program:
\bi
\item \ttv{eyedbadmin}: the generic administration program used to perform
all the database management or user management operations (do not confuse
this program with the storage manager level administration program
\ttv{eyedbdbadmin}).
\ei
\item Database management programs:
\bi
\item \ttv{eyedbdbcreate}: to create a database,
\item \ttv{eyedbdbdelete}: to delete a database,
\item \ttv{eyedbdblist}: to list databases,
\item \ttv{eyedbdbrename}: to rename a database,
\item \ttv{eyedbdbmove}: to move a database,
\item \ttv{eyedbdbcopy}: to copy a database,
\item \ttv{eyedbdbexport}: to export a database to a file,
\item \ttv{eyedbdbimport}: to import a database from a file,
\item \ttv{eyedbdbaccess}: to set the default database access,
\item \ttv{eyedbdbadmin}: to perform storage manager level database operations
(do not confuse this program with the generic administration program \ttv{eyedbadmin}),
\item \ttv{eyedbdbmcreate}: to create an EYEDBDBM database,
\ei
\item User management programs:
\bi
\item \ttv{eyedbuseradd}: to add an \eyedb user,
\item \ttv{eyedbuserdelete}: to delete an \eyedb user,
\item \ttv{eyedbuserlist}: to list the \eyedb users,
\item \ttv{eyedbpasswd}: to set the password of an \eyedb user,
\item \ttv{eyedbuserpasswd}:  to set the password of an \eyedb user,
\item \ttv{eyedbuserdbaccess}: to set a database access for a specific user,
\item \ttv{eyedbsysaccess}: to set system access for a specific user,
\ei
\item OQL monitor:
\bi
\item \ttv{eyedboql}: OQL command monitor,
\ei
\item ODL compiler:
\bi
\item \ttv{eyedbodl}: ODL compiler used to generate C++, Java, IDL or
CORBA stubs,
\ei
\item Generic CORBA servers:
\bi
\item \ttv{eyedborbacusd}: \ttv{Orbacus} generic \eyedb server (since
version \eyedbversion),
\item \ttv{eyedborbixd}: \ttv{Orbix} generic \eyedb server,
\ei
\item WWW programs:
\bi
\item \ttv{eyedbwwwd}: the \eyedb WWW server,
\item \ttv{eyedbcgife}: the \eyedb WWW frontend,
\item \ttv{eyedbw3tool}: a WWW tool for customization,
\item \ttv{eyedbputimage}: to insert an image in a database,
\ei
\item Miscellaneous programs:
\bi
\item \ttv{eyedbgetenv}: to get the value of an \eyedb configuration variable,
\item \ttv{eyedbchklicense}: to check the validity of a license file,
\item \ttv{eyedbprotadmin}: to manage object-level protections,
\ei
\ee
\item \ttv{etc}: this directory contains the enviroment files:
\bi
\item \ttv{environ.sh}: this Bourne shell script set the
\ttv{EYEDBROOT}, \ttv{EYEDBCLASSPATH} and \ttv{PATH} environment variables.
The \ttv{EYEDBROOT} is set to \tdir/\ttv{eyedb-V}\emph{version}, for instance
\tdir/\ttv{eyedb-V\eyedbversion}.
Note that these environment variables are not strictly necessary to use
\eyedb but are recommended. We assume in the rest of the chapter that
these environment variables have been correctly set by sourcing
\ttv{environ.sh}
\item \ttv{environ.csh}: C-shell version of the \ttv{environ.sh} Bourne
shell script.
\item \ttv{rc.eyedb}: template shell script used to launch the \eyedb server,
the \eyedb WWW server and the \eyedb generic CORBA server. This file could
be customized by the \eyedb administrator.
\item \ttv{EyeDB.access}: controls TCP/IP accesses,
\item \ttv{Makefile.eyedb}: a template makefile to be included in your makefile
to compile properly \eyedb clients: C++, Java or CORBA \eyedb clients.
\item \ttv{EyeDB.cfg}: the default configuration file generated at the
installation step,
\item \ttv{so}: the default directory where to put backend methods or
triggers for dynamic loading.
\ei
\item \ttv{include}: this directory contains the C++ include files.
\item \ttv{lib}: this directory contains the static and shared C++ libraries
and the \ttv{classes.jar} file.
\item \ttv{doc}: this directory contains the documentation. Note
that the documentation is not part of the binary package, you need
to install properly the documentation package.
\item \ttv{examples}: this directory contains a set of examples.
\item \ttv{java}: this directory contains the Java API classes (same
as the \ttv{classes.jar} files).
\item \ttv{VERSION}: this file contains the version number as follows:
\begin{verbatim}
Copyright (c) SYSRA 1995-2001
EyeDB Version \eyedbversion, January 2001
\end{verbatim}
\ei
The files and directories in the distribution directory
are created with the following rights:
\bi
\item the files under the \ttv{bin} directories have read and execute
rights for all users,
\item the other files have the read right for all users,
\item the directories have read, write and search rights for the
installing user, and read and search rights for the other users.
\ei
\subsection{Suggested Organization}
To install and run \eyedbX, it is recommended to use a specific
user named \ttv{eyedb}.
Although this is not mandatory for an evaluation installation,
it is strongly recommended for a production installation.
Here are the suggested recommandations:
\be
\item as the UNIX root, create a pseudo UNIX user (i.e. a UNIX
user with no login) named \ttv{eyedb},
\item in the \ttv{eyedb} home directory, create the file \ttv{.rhosts}
which contains the following line:
\\
\ttv{localhost} \emph{DBA-user-list}
\\
for instance:
\\
\ttv{localhost root bill johnny}
\\
This means that the users \ttv{root}, \ttv{bill} and \ttv{johnny} are
granted to log without password under the \ttv{eyedb} user using the
command:
\ttv{rlogin localhost -l eyedb}
\\
As \ttv{eyedb} is a pseudo UNIX user, there are no other way to connect
under this user.
The users \ttv{root}, \ttv{bill} and \ttv{johnny} can act as Database
Administrators (DBA) as they are able to start or stop the \eyedb server,
and to write database files in the directory \tdir/\ttv{db}
after have logged under the \ttv{eyedb} UNIX user.
\ee

\section{Configuration}
At the installation step, the installation shell script creates
a default configuration file \tdir/\ttv{eyedb/etc/EyeDB.cfg}.
This file contains a few configuration variables. Some of them
are used by the \eyedb clients and the server while others
are used only by the server (variables prefixed by \ttv{sv\_}).
Some of them cannot be changed, others are modifiable.
Here is a typical configuration file:
\verbsize
\begin{verbatim}
#
# EyeDB package, version 2.6.6
#

# EyeDB root directory

topdir = /usr/local/eyedb;
root   = /usr/local/eyedb/eyedb-V2.6.6;

# Version Number

version = 2.6.6;
numversion = 205001;

# EyeDB ports

port = %root%/pipes/1;

# EyeDB host

host = bill;

# EYEDBDBM Database

dbm = %topdir%/db/dbmdb.db;

# Java Class Path

classpath = %root%/lib/classes.jar;

sopath = %root%/etc/so;

# Set the following variables for a default authentication

#user   = 
#passwd = 

# Server Parameters

sv_port = %port%;
sv_voldir = %topdir%/db; 
sv_timeout = 7200;
sv_access_file = %root%/etc/EyeDB.access;
sv_tmpdir = %root%/tmp;

sv_passwd_file = %topdir%/etc/passwd;
sv_license_file = %topdir%/etc/license.dat;
\end{verbatim}
\normalsize
{\bf Client and server variables}:
\bi
\item \ttv{topdir} denotes the target directory
given in the installation step: \tdir. \constvar.
If you want to move the \eyedb distribution to another
directory, you need to perform a new partial installation.
\item \ttv{root} denotes the \eyedb root directory,
for instance \ttv{\%root\%/eyedb-V\eyedbversion} where the syntax
construct \ttv{\%xx\%} denotes the value of the variable \ttv{xx}.
\constvar.
\item \ttv{version} denotes the string version of the current \eyedb
installation, for instance \ttv{\eyedbversion}. \constvar.
\item \ttv{numversion} denotes the numeric version of the current \eyedb
installation, for instance \ttv{20406}. \constvar.
\item \ttv{port} denotes the port for connection
with the \eyedb server. A numeric value denotes a TCP/IP port while
a alpha-numeric value denotes a named pipe. Its default value is
equal to \ttv{\%root\%/pipes/1}. \modvar.
\item \ttv{host} denotes the host name or IP address for connection with
the \eyedb server. Its default value is the hostname on which the
installation has been performed. \modvar.
\item \ttv{dbm} denotes the database file of the EYEDBDBM database. Its
default value is \ttv{\%root\%/db/dbmdb.db}.
\modvar.
\item \ttv{classpath} denotes the class path for the \eyedb Java library.
Its default value is \ttv{\%root\%/lib/classes.jar}.
\modvar.
\item \ttv{sopath} denotes the path for the loadbable methods and triggers.
Its default value is \ttv{\%root\%/etc/so}.
\modvar.
\item the variables \ttv{user} and \ttv{passwd} are not set by default.
The value of these variables can be modified.
\ei
{\bf Server variables}:
\bi
\item \ttv{sv\_port} denotes the port on which the \eyedb server is listening.
Its default value is set to \ttv{\%port\%}. \modvar.
\item \ttv{sv\_voldir} denotes the default database file repositery.
Its default value is \ttv{\%topdir\%/db}. \modvar.
\item \ttv{sv\_timeout} denotes the elapsed timeout client inactivity in
seconds. The default is \ttv{7200}. \modvar.
\item \ttv{sv\_access\_file} denotes the TCP/IP control access file.
Its default value is \ttv{\%root\%/etc/EyeDB.access}.
\modvar.
\item \ttv{sv\_passwd\_file} denotes the password file for
EYEDBDBM creation. Its default value is \ttv{\%topdir\%/etc/passwd}.
\item \ttv{sv\_license\_file} denotes the license file path.
Its default value is \ttv{\%topdir\%/etc/license.dat}.
\ei
\subsection{Configuration File Syntax}
The configuration syntax rules are as follows:
\bi
\item comments: all the characters following a \ttv{\#} are ignored.
\item the general form of an assignation is as follows:
\\
\\
\mbox{ } \emph{variable} = \emph{value}\ttv{;}
\item \emph{variable} is under the following form:
\bi
\item \ttv{[a-zA-Z\_][a-zA-Z\_0-9]*} (i.e. a letter or underscore optionnaly
following by letters, digits or underscore).
\ei
\item \emph{value} is under the following form:
\bi
\item \ttv{"([\^{}{\textbackslash}"]$|${\textbackslash}")*"}, where the surrounding double
quotes are optionnal. You need only to surround with double quotes if
value contains one or more spaces.
\item the following escape sequences are interpreted:
\esc{a}, \esc{b}, \esc{f}, \esc{r}, \esc{t}.
\esc{v}, \esc{\textbackslash}
\item a sequence variable delimited by two \ttv{\%} are substitued by the value
of this variable, for instance \ttv{\%root\%} is substitued by the value of the
variable \ttv{root}.
\ei
\item including other configuration file:
\\
\ttv{@include /tmp/EyeDB.cfg}
\\
includes the file \ttv{/tmp/EyeDB.cfg}
\\
\ttv{@include etc/EyeDB.cfg}
\\
includes the file \ttv{etc/EyeDB.cfg} relatively to the root \eyedb
directory.
\ei
For instance, the following constructs are valid:
\begin{verbatim}
xx = %root%/abc/edf;
yy = "%root%  1 2 abc \n \t";
alpha = " == ; \ney;\tedb  \a \"\"root ";
\end{verbatim}
\subsection{The \ttv{eyedbgetenv} Program}
\eyedb provides a very simple but useful utility to display the
value of a configuration variable: \ttv{eyedbgetenv}.
\\
For instance:\\
\mbox{ }\ttv{eyedbgetenv root}\\
displays the value of the \ttv{root} configuration variable, while\\
\mbox{ }\ttv{eyedbgetenv xxx}\\
does not display anything as no configuration variable \ttv{xxx} is
set.
\subsection{Configuration General Mechanisms}
If you want to change some configuration values, there are several
ways:
\be
\item you edit the \tdir/\ttv{eyedb/etc/EyeDB.cfg} file,
\item you create a new configuration file which includes the default
one and you override the variables that you want,
%\item you use environment variables,
\item you use the standard \eyedb command line options which works
with any \eyedb client.
\ee 
\subsection{Editing the default configuration file}
Using a standard text editor, you can easily modify any of the
modifiable configuration variables: \ttv{port}, \ttv{host}, \ttv{dbm},
\ttv{classpath}, \ttv{sopath}, \ttv{user}, \ttv{passwd}, \ttv{sv\_port},
\ttv{sv\_voldir}, \ttv{sv\_timeout}, \ttv{sv\_access\_file},
\ttv{sv\_passwd\_file} and/or \ttv{sv\_license\_file}.
\subsection{Creating and using a new configuration file}
Using a standard text editor, you create a new file, for instance
\ttv{\$HOME/EyeDB.cfg}, in which:
\bi
\item you re-define all the configuration variables,
or
\item you include the default configuration file and you
override a few variables.
\\
For instance:
\verbsize
\begin{verbatim}
# private EyeDB configuration file

@include etc/EyeDB.cfg

user=bill;
passwd="^&acb12";
\end{verbatim}
\normalsize
\ei
To use your private configuration file, you need to set the environment
variable \ttv{EYEDBCONF} to \ttv{\$HOME/EyeDB.cfg} or use the \ttv{-eyedbconf}
command line option.

%\subsection{Using environment variables}

\subsection{Using command line options}
All the configuration variable values can be overriden by a command line
option or a hard coded value: for instance, the \ttv{port} configuration
variable value can be overriden by using the command line option
\ttv{-eyedbport}.
\\
\\
Here is the list of the environment variables that
can be overriden by a command line option:\\
\\
\begin{tabular}{|p{4cm}|p{4cm}|p{3cm}|}
\hline {\bf Configuration Variable} & {\bf Command Line Option} & {\bf Client/Server} \\
\hline \ttv{root} & \ttv{-eyedbroot} & Client \\
\hline \ttv{host} & \ttv{-eyedbhost} & Client \\
\hline \ttv{port} & \ttv{-eyedbport} & Client \\
\hline \ttv{dbm} & \ttv{-eyedbdbm} & Client \\
\hline \ttv{user} & \ttv{-eyedbuser} & Client \\
\hline \ttv{passwd} & \ttv{-eyedbpasswd} & Client \\
\hline \ttv{sv\_port} & \ttv{-sv\_port} & Server \\
\hline \ttv{sv\_access\_file} & \ttv{-sv\_access\_file} & Server \\
\hline \ttv{sv\_vol\_dir} & \ttv{-sv\_vol\_dir} & Server \\
\hline \ttv{sv\_timeout} & \ttv{-sv\_timeout} & Server \\
\hline \ttv{sv\_passwd\_file} & \ttv{-sv\_passwd\_file} & Client \\
\hline \ttv{sv\_license\_file} & \ttv{-sv\_license\_file} & Client \\
\hline
\end{tabular}

\subsubsection{The standard client command line options:}
\bi
\item \ttv{-eyedbroot} \emph{root}: if you get several \eyedb installation, you
can force to use a given one by using this option. Rarely used.
\item \ttv{-eyedbhost} \emph{host}: to set the host name or IP address server,
\item \ttv{-eyedbport} \emph{port}: to set the port TCP/IP or named pipe,
\item \ttv{-eyedbdbm} \emph{dbmfile}: to use another EYEDBDBM than the default
one,
\item \ttv{-eyedbuser} \emph{user}: to set an \eyedb user name,
\item \ttv{-eyedbpasswd} \emph{passwd}: to set an \eyedb password,
\item \ttv{-eyedbuser?}: will ask for an \eyedb user name on
the standard input. Note that if you are using a C-shell, you need to
put a backslash before the \ttv{?} character to avoid the interpretation
of this meta-character by the shell,
as follows: \ttv{-eyedbuser\textbackslash?},
\item \ttv{-eyedbpasswd?}: will ask for an \eyedb password on
the standard input. Refer to previous option if you are using with a C-shell.
\item \ttv{-eyedbauth?}: will ask for an \eyedb user/password
on the standard input.  Refer to previous option if you are using with a C-shell.
\item \ttv{-eyedbconf} \emph{conffile}: will use the given configuration file.
It is the same than using the \ttv{EYEDBCONF} environment variable,
\item \ttv{-eyedblog} \emph{logfile}: log information will be displayed
in the given file,
\item \ttv{-eyedbarch}: displays the client architecture (since version \eyedbversion),
\item \ttv{-eyedbversion}: displays the \eyedb version,
\item \ttv{-eyedbhelp}: displays the help window,
\item \ttv{-eyedb-error-policy} \emph{directive}: uses a specific
error policy according to the value of \emph{directive} which can be:
\ttv{status},
\ttv{exception}, \ttv{abort}, \ttv{stop} or \ttv{echo}.
\ei
For instance to connect the \ttv{eyedboql} monitor to the \eyedb server
on host \ttv{alpha1} on the TCP/IP port \emph{3122} using the \eyedb user
\ttv{bill} and password \ttv{bf123}:
\begin{verbatim}
eyedboql -eyedbhost alpha1 -eyedbport 3122 -eyedbuser bill \
        -eyedbpasswd bf123
\end{verbatim}
\subsubsection{The server command line options}
All the server configuration variable values can be overriden by a command line
option or a hard coded value: for instance, the \ttv{sv\_port} configuration
variable value can be overriden by using a server command line option
as follows:
\ttv{eyedbctl -sv\_port 7863}.
\\
\\
Here is the list of all the standard server options that are recognized
by \ttv{eyedb} and \ttv{eyedbctl}:
\bi
\item \ttv{-sv\_port} \emph{ports}:  listening ports,
\item \ttv{-sv\_access\_file} \emph{accessfile}: access file (default is \tdir/\ttv{eyedb/etc/EyeDB.access}),
\item \ttv{-sv\_voldir} \emph{voldir}: default volume directory,
\item \ttv{-sv\_timeout} \emph{timeout}: timeout in seconds (default is 7200 sec),
\item \ttv{-sv\_passwd\_file} \emph{passwdfile}: use the given password file,
\item \ttv{-sv\_license\_file} \emph{licensefile}: use the given license file,
\item \ttv{-eyedbroot} \emph{root}: use the given \eyedb root directory,
\item \ttv{-eyedbconf} \emph{conffile}: use the given configuration file,
\item \ttv{-eyedblog} \emph{logfile}: output log file,
\item \ttv{-eyedbdbm} \emph{dbmfile}: default EYEDBDBM database file,
\item \ttv{-eyedbversion}: displays the \eyedb version,
\item \ttv{-help}: displays a short help.
\ei
For instance, to launch an \eyedb server listening on the ports \ttv{7362} and
\ttv{/tmp/pipes1}, using the default volume directory \ttv{/export/DB}
and the default EYEDBDBM database file \ttv{/export/DB/dbmdb.db}:
\begin{verbatim}
eyedbctl start -sv_port 7362 /tmp/pipes1 -sv_voldir /export/DB \
              -eyedbdbm /export/DB/dbmdb.db
\end{verbatim}

\section{Server Management}
This section introduces the way to launch or stop the server, and to
get status information on the opening sessions.
\subsection{Launching the server}
There are three ways to launch the server:
\be
\item running directly the program \ttv{eyedbd}: this way is not
recommended.
\item running the program \ttv{eyedbctl} with the {start} option, for
instance: \\
\mbox{ }\ttv{eyedbctl start} \\
\mbox{ }\ttv{eyedbctl start -sv\_port 7763} \\
\mbox{ }\ttv{eyedbctl start -eyedblog `eyegetenv topdir`/log/eyedbd.log}
\item running a \ttv{rc.eyedb} shell script: a template is given
in the distribution (file \tdir/\ttv{eyedb/etc/rc.eyedb}); this way
is recommended as this shell script can launch all the \eyedb servers:
\be
\item the \ttv{eyedbd} daemon,
\item the \ttv{eyedbwwwd} WEB daemon,
\item the \ttv{eyedborbixd} generic Orbix server and/or
\item the \ttv{eyedborbacusd} generic Orbix server.
\ee
\ee
Note that the \ttv{rc.eyedb} uses \ttv{eyedbctl start} to launch the
\eyedb server.
\subsection{Listening ports}
There are two types of listening ports: TCP/IP ports and named pipes:
\bi
\item a TCP/IP port is denoted by a number and is available over
a LAN
or a WAN. By configuring the TCP/IP access file (by default the file
\ttv{EyeDB.access}), you can allow or not allow the TCP/IP connections
from a so-and-so host machine.
\item a named pipe is denoted by a filename and is available for
the clients running on the same computer as the server. On the other
hand, a named pipe is much faster than a TCP/IP connection and it allows
for safe authentication without password transmission. Refer
to the \emph{Users} section for information about that.
\ei
It is a good idea to make the \eyedb server listening both on
a TCP/IP port and on a named pipe, for instance:\\
\ttv{eyedbctl start -sv\_port 8100 /tmp/.eyedb-pipe}\\
launches an \eyedb server listening both on the TCP/IP port \ttv{8100}
and on the named pipe \ttv{/tmp/.eyedb-pipe}.
\\
Clients on the same computer will preferly use the named pipe because
transmission is faster, and clients on the LAN or a WAN will use
the TCP/IP port.
\subsection{Lauching Several Servers}
Note that launching one \eyedb server as follows:\\
\ttv{eyedbctl start -sv\_port 8100 /tmp/.eyedb-pipe}\\
is equivalent to launch two \eyedb servers as follows:\\
\ttv{eyedbctl start -sv\_port 8100}\\
\ttv{eyedbctl start -sv\_port /tmp/.eyedb-pipe}\\
\\
The differences is that in the first case, there is only one server listening
on two ports, while in the second case there are two servers listening on
one port each.\\
If you launch the server without using the \ttv{-sv\_port} option,
the value of the listening port will be taken from the configuration
file, so:\\
\ttv{eyedbctl start}\\
is equivalent to:\\
\ttv{eyedbctl start -sv\_port `eyedbgetenv sv\_port`}
\\
\\
There can be as many servers as you want on the same computer dealing
with the same databases: of course, the servers must listen on different
TCP/IP ports or named pipes: in other words, a database can be shared
safely by any number of servers that you want, running on the same computer.
\subsection{MultiProcess Server Architecture}
The \eyedb server architecture is multi-processus: each time a client
reaches the server for a connection and that this connection is accepted,
the server forks a new server.
\\
Assume that two clients (for instance two
\ttv{eyedboql}) are connected to the \eyedb server, there are three
server processus:
\bi
\item the master server that has been launched using the \ttv{eyedbctl start}
command,
\item two children of the master server, each one coressponding to each
client.
\ei
\subsection{Getting Session Information}
The command \ttv{eyedbctl status} gives all the information about the
opened sessions. For instance the following information could be displayed
after launching the server:
\verbsize\begin{verbatim}
% eyedbctl status
EyeDB Server is up from Thu Apr  1 14:18:05 1999
 
  Version 2.6.6
  Program Pid 3263
  Running Under 'eyedb'
 
  Listening Ports 8100
  Volume Dir '/eyedb/db'
 
  No Clients connected.
\end{verbatim}\normalsize
If two \ttv{eyedboql} clients are connected:
\verbsize\begin{verbatim}
% eyedbctl status
EyeDB Server is up from Thu Apr  1 14:18:05 1999
 
  Version 2.6.6
  Program Pid 3263
  Running Under 'eyedb'
 
  Listening Ports 8100
  Volume Dir '/eyedb/db'
 
  2 Clients connected
 
Client #0
  Connected from Thu Apr  1 14:19:11 1999
  Host Name 'bill'
  User Name 'john'
  Program Name 'eyedboql'
  Client Pid 3282
  EyeDB Server Pid 3283
  Databases Opened 'EYEDBDBM' [mode='read/trsless/local']
                   'person' [mode='read'] [userauth='eyedbroot']
 
Client #1
  Connected from Thu Apr  1 14:20:25 1999
  Host Name 'bill'
  User Name 'john'
  Program Name 'eyedboql'
  Client Pid 3301
  EyeDB Server Pid 3302
  Databases Opened 'EYEDBDBM' [mode='read/trsless/local']
\end{verbatim}\normalsize
If you have launched your server on another port as the default port,
you must use \ttv{eyedbctl status} with the \ttv{-sv\_port} command line option,
as follows:\\
\ttv{eyedbctl status -sv\_port /tmp/.eyedb-pipe}

\subsection{The TCP/IP access file}
\newcommand{\tcpip}{\emph{IP-address-or-mask}}
The server access file controls TCP/IP accesses. Its location
is given by the configuration variable \ttv{sv\_access\_file} which
is set by default to \ttv{\%root\%/etc/EyeDB.access}.
\\
\\
The access file syntax rules are as follows:
\bi
\item comments: all the characters following a \ttv{\#} are ignored.
\item the access file is composed of a set of lines where each
line is under the form:
{\tcpip } \emph{user-list}
\item {\tcpip } can be:
\bi
\item a name designating a computer or a subnet mask,
\item a IP address designating a computer or a subnet mask,
\item the \ttv{+} character designating any computer anywhere.
\ei
\item \emph{user-list} is a list of tokens under the form:
\bi
\item an \eyedb user: means that the given user is authorized for
connection from any computer designated by {\tcpip},
\item a \ttv{!} character followed by an \eyedb user: this
means that the given user is not authorized for
connection from any computer designated by {\tcpip},
\item a \ttv{=} character followed by an \eyedb user: means
that this user will be used for any connection from any computer designated
by {\tcpip } if no authentication is provided.
\item the \ttv{+} character: means that any \eyedb user
is authorized for
connection from any computer designated by {\tcpip}.
\ei
\ei
For instance:
\verbsize\begin{verbatim}
# EyeDB.access file

net.infobiogen.fr       + !eyedbroot
enst.enst.fr            enst
www.ircam.fr            rhgm
enst.enst.fr            =guest hgm
192.168.1.0             eyedbroot =foo stuart
192.168.1.3             +
\end{verbatim}\normalsize
Let's explain some of these access lines:
\bi
\item \ttu{net.infobiogen.fr + !eyedbroot}: means than any \eyedb user except
\ttu{eyedbroot} is allowed for connection from the computer(s) designated
by \ttu{net.infobiogen.fr}.
\item {\verbsize\ttu{192.168.1.0  eyedbroot =foo stuart}}: means that the \eyedb users
\ttu{eyedbroot}, \ttu{foo} and \ttu{stuart} are authorized for connection
from the computers designated by the subnet mask \ttu{192.168.1.0}. If
no user authentication is provided for this connection, the user \ttu{foo}
is used.
\item \ttu{192.168.1.3 +}: means that any \eyedb user are authorized
for connection from the computer designated by 
\ttu{192.168.1.3 +}.
\ei
\subsection{The servers and NFS}
An \eyedb server can manage a database located on a partition mounted by NFS.
As \eyedb uses a lot of disk access, and as NFS is less efficient that
a direct access, this is not a very good idea to do so. But you can.
\\
\\
However, there is a restriction for the use of database mounted via NFS:
a same physical database cannot be shared by two servers running on
different computers. Assume you have the following configuration:
\be
\item the \eyedb volume directory \ttv{VOLDIR} is directly mounted or mounted via NFS
on the computer \ttv{A} and an \eyedb server is running on this computer,
\item the \eyedb volume directory \ttv{VOLDIR} is directly mounted or mounted via NFS
on the computer \ttv{B} and an \eyedb server is running on this computer,
\item a client is using the database \ttv{DB} through the
\eyedb server running on the computer \ttv{A}.
\ee
If all the previous conditions are true, the server running on
the computer \ttv{B} cannot deal with the database \ttv{DB}.
If you try to open \ttv{DB} through the server running on the
computer \ttv{B}, you will get the following message:\\
\ttv{database opening error: database 'DB' is hold by computer A}.

\section{Stopping the servers}
To stop the \eyedb servers, you need to use the \ttv{eyedbctl stop}
command, for instance:
{\verbsize\begin{verbatim}
eyedbctl stop
eyedbctl stop -sv_port 8100
eyedbctl stop -sv_port /tmp/.eyedb-pipe
\end{verbatim}}
\subsection{License Management}
To be able to launch a server, the license file must contain
a valid license. \\
By default, the license file is \tdir/\ttv{etc/license.dat}  and
has one of the following form:
\be
\item evaluation license: \ttv{EyeDB license} \ttv{*eval* 0} \emph{date} \emph{key}
\item host oriented license: \ttv{EyeDB license} \emph{hostname} \emph{hostid} \emph{date} \emph{key}
\ee
For instance, the license file: \\
\ttv{EyeDB license ultra 8081d7bf 1998/1/1 ACE4591178355C215357B3C6C4A27C21}
\\
contains a license for the computer \ttv{ultra} with an hostid \ttv{8081d7bf}
available until the 1st january 1998.
\\
Note that the license file is only use by the server: there is no
runtime license for the clients.
To install a new license file, you only need to replace the
deprecated license file by the new one: it is not necessary to
stop any servers or clients.
\\
\\
To check the validity of a license file, you must use the program
\ttv{eyedbchklicense} as follows:\\
\ttv{eyedbchklicense} \emph{licensefile}\\
If the license is valid no output will be displayed and a status
code of value 0 will be returned.
Otherwise, a clear error message will be displayed and a status of value
1 will be returned.

\section{Databases}

\subsection{Database Structure}
The structure of a database is composed of a logical part and a physical
part:
\bi
\item the \emph{logical part} is composed of:
\be
\item a \emph{database name}: this name identifies logically 
a database in a unique way. This name can be changed at any time.
\ee
\item the \emph{physical part} is composed of:
\be
\item \emph{a database identifier}: this number identifies
a database in an unique way,
\ee
\item \emph{a set of files}:
\be
\item \emph{the database file} (suffixed by \ttv{.db}): contains information
about volume files, root entries and the main storage allocator.
\item \emph{the object map file} (suffixed by \ttv{.omp}): contains
the mapping between the logical object identifiers (OIDs) and their
physical location
\item \emph{the log database file} (suffixed by \ttv{.log}): contains the transaction
tables,
\item \emph{the volumes files} (at least one; the first one is generally suffixed by \ttv{.vol}):
contains object data.
\ee
\ei
Note: do not confuse the log database file with the log information displayed by
the server in the \tdir/\ttv{log} directory.
\\
\\
By default, these files are located in the directory given by the configuration
variable \ttv{sv\_voldir}.
However, at the creation step, you can choose where to put your database
files in the directory you want. Furthermore, you can dynamically add, 
move, resize or suppress (under certain conditions) volume files.
\\
To get information on an existing database, you can use the \ttv{eyedbdblist}
program, for instance:
\verbsize
\begin{verbatim}
% eyedbdblist person
dbname    : "person"
dbid      : 21
dbfile    : "/eyedb/db/person.db"
volumes   : (file : "/eyedb/db/person.vol", maxsize : 2000000 Kb)
defaccess : idbReadDBAccessMode
dbaccess  : (username : "eyedbroot", access : idbAdminDBAccessMode)
\end{verbatim}
\normalsize
As this is shown, the object map file and the log database file are
not displayed by the previous commands: anyhow, these files are
located in the same directory as the database file and are
suffixed respectively by \ttv{.omp} and \ttv{.log}.

\subsection{Database Management}
Using \eyedb administration tools, you can perform the following operation
on databases:
\be
\item create a database,
\item delete a database,
\item list databases,
\item rename a database,
\item move a database,
\item copy a database,
\item export a database to a file,
\item import a database from a file,
\item add a volume file,
\item resize a volume file,
\item move a volume file,
\item suppress a volume file.
\ee
The last four operations are performed by using the storage manager level
administration tool \ttv{eyedbdbadmin}. The other operations can be
performed either by using the \ttv{eyedbadmin} generic administration tool
or a specific tool named \ttv{eyedbdb}\emph{operation} where
\emph{operation} is one of the following: \ttv{create}, \ttv{delete},
\ttv{list}, \ttv{rename}, \ttv{move}, \ttv{copy}, \ttv{export} or \ttv{import}.
\\
All the administration tools, except the storage manager level
\ttv{eyedbdbadmin} program understand the client command line options presented
in a previous section (i.e. \ttv{-eyedbport}, \ttv{-eyedbhost} and so on).
\\
Note that to perform the previous operations, some privileges are needed.
For instance, a user cannot create a database if it has no \ttv{dbcreate}
system access. Refer to the section \emph{Access Control} for information
about that.
\subsubsection{Creating a Database}
To create a database one can use either \ttv{eyedbadmin create} \emph{args}
or \ttv{eyedbdbcreate} \emph{args}.
The command line options for this operation are as follows:
\bi
\item \emph{dbname}: the database name, this argument is mandatory.
\item \ttv{-dbfile} \emph{dbfile}: by default, the database file,
the object map
file and the log database file will be created in the directory designated
by the configuration variable \ttv{sv\_voldir}. If you want to create these
files in another directory, you must indicate the location of the
database file using the \ttv{-dbfile} option. The object map file
and the log database file will be automatically created in the
same directory as the database file.
\item \ttv{-volfiles} \emph{volfile size} [\emph{volfile size}]: by default,
one volume file is created in the directory designated by the
configuration variable \ttv{sv\_voldir} with a default maximum size of
2,000,000 Kb (a bit less than 2 Gb). If you want to create the volume
file in another location or with another maximum size or if you want
to use several volume files you must use this option.
The maximum number of volumes files is 512, each one limited to 2Gb because
of some 32-bit limitation. The 64-bit version of \eyedb will theorically
support 512 volumes files of one tera-bytes each.
Note that volumes files can be added or moved dynamically using the
\ttv{eyedbdbadmin} storage manager level administration tool.
\item \ttv{-sizeslot} \emph{size}: the default slot size in the storage
manager is 16 bytes. If you want to change this size, you must use this option.
Note that this option should be used only by experts.
\item \ttv{-nbobjs} \emph{nbobjs}: the default maximum object number is
100,000,000. If you want to change this maximum number, you must use
this option. This maximum object number can be modified using the
\ttv{eyedbdbadmin} storage manager level administration tool.
\ei
Here are some database creation examples:
\bi
\item default creation:
\verbsize
\begin{verbatim}
eyedbdbcreate foo
\end{verbatim}
\normalsize
\item the following command is strictly equivalent to the previous one:
\verbsize
\begin{verbatim}
eyedbadmin create foo
\end{verbatim}
\normalsize
\item same as previously but creating a database using another server
and another port:
\verbsize
\begin{verbatim}
eyedbdbcreate foo -eyedbhost ultra.sysra.com -eyedbport 8201
\end{verbatim}
\normalsize
\item creating a database with two volumes files:
\verbsize
\begin{verbatim}
eyedbdbcreate foo -volfiles /eyedb/db1/foo.vol1 1000000 \
                            /eyedb/db2/foo.vol2 2000000
\end{verbatim}
\normalsize
\item creating a database with a specific database file,
three volumes files and a specific maximum object number:
\verbsize
\begin{verbatim}
eyedbdbcreate foo -dbfile /usr/local/db/FOO.db \
                  -nobjs 200000000 \
                  -volfiles /usr/local/db/foo.vol1 10000000 \
                            /eyedb/db2/foo.vol2 1000000 \
                            /eyedb/db2/foo.vol3 100000000
\end{verbatim}
\normalsize
\ei
\subsubsection{Deleting a Database}
To delete a database one can use either \ttv{eyedbadmin delete} \emph{dbname}
or \ttv{eyedbdbdelete} \emph{dbname}.
There is one and only argument for this command:
\bi
\item \emph{dbname}: the database name, this argument is mandatory.
\ei
Here are some database deletion examples:
\bi
\item default deletion:
\verbsize
\begin{verbatim}
eyedbdbdelete foo
\end{verbatim}
\normalsize
\item the following command is strictly equivalent to the previous one:
\verbsize
\begin{verbatim}
eyedbadmin delete foo
\end{verbatim}
\normalsize
\item same as previously but deleting a database using another server
and another port:
\verbsize
\begin{verbatim}
eyedbdbdelete foo -eyedbhost ultra.sysra.com -eyedbport 8201
\end{verbatim}
\normalsize
\ei
\subsubsection{Listing Databases}
To list one or more databases one can use either \ttv{eyedbadmin list}
\emph{args}
or \ttv{eyedbdblist} \emph{args}.
Where \emph{args} is as follows:
\bi
\item \emph{empty}: the program list all databases.
\item one or more database names: the program list the indicated databases.
\ei
Here are some database list examples:
\bi
\item list all databases:
\verbsize
\begin{verbatim}
eyedbdblist
\end{verbatim}
\normalsize
\item the following command is strictly equivalent to the previous one:
\verbsize
\begin{verbatim}
eyedbadmin list
\end{verbatim}
\normalsize
\item list databases \ttv{foo} and \ttv{person}:
\verbsize
\begin{verbatim}
eyedbdblist foo person
\end{verbatim}
\normalsize
\item same as previously but listing databases using another server
and another port:
\verbsize
\begin{verbatim}
eyedbdblist foo person -eyedbhost ultra.sysra.com -eyedbport 8201
\end{verbatim}
\normalsize
\ei
\subsubsection{Renaming a Database}
To rename a database, one can use either \ttv{eyedbadmin rename}
\emph{args}
or \ttv{eyedbdbrename} \emph{args}.
The command line options for this operation are as follows:
\bi
\item \emph{dbname}: the database name to be renamed. This argument is
mandatory.
\item \emph{newdbname}: the new database name.
This argument is mandatory.
\ei
Some examples:
\bi
\item renaming the database \ttv{foo} to \ttv{FOO}:
\verbsize
\begin{verbatim}
eyedbdbrename foo FOO
\end{verbatim}
\normalsize
\item renaming the database \ttv{person} to \ttv{test\_person} using
another server and another port:
\verbsize
\begin{verbatim}
eyedbdbrename person test_person -eyedbhost ultra.sysra.com -eyedbport 8201
\end{verbatim}
\normalsize
\ei
\subsubsection{Moving a Database}

To move a database one can use either \ttv{eyedbadmin move} \emph{args}
or \ttv{eyedbdbmove} \emph{args}.
The command line options for this operation are as follows:
\bi
\item \emph{dbname}: the database name. This argument is mandatory.
\item \ttv{-dbfile} \emph{dbfile}:
the target database file. This argument is mandatory.
The object map file
and the log database file will be automatically created in the
same directory as this target database file.
\item \ttv{-volfiles} \emph{volfile size} [\emph{volfile size}] (versions
2.4.0 to 2.4.5):
the target volume files and sizes. This argument is mandatory.
There are some restrictions to the use of this option:
\be
\item the number of the target volume files must be the same as the
number of volume files of the source database.
\item the size of each volume file must be the same as the size
of each source volume file.
\ee
Note that as the size cannot be set by the user, this argument is quite
not useful.
So, it has been suppressed since the version 2.4.6.
\\
If you want to change the size or the volume number of the moved database,
you must use the \ttv{eyedbdbadmin} storage manager level administration tool.
\item \ttv{-volfiles} \emph{volfile} [\emph{volfiles}] (version
2.4.6 and upper):
the target volume files. This argument is mandatory.
There are some restrictions to the use of this option:
\be
\item the number of the target volume files must be the same as the
number of volume files of the source database.
\ee
\ei
Here are some database creation and moving examples:
\bi
\item creation and move using version \eyedbversion:
\verbsize
\begin{verbatim}
eyedbdbcreate foo
eyedbdbmove foo -dbfile /eyedb/db/foo_1.db \
                -volfiles /eyedb/db/foo_1.vol 2000000
\end{verbatim}
\normalsize
\item same moving operation using version 2.4.6 or upper (the difference is that there is no
size arguments):
\verbsize
\begin{verbatim}
eyedbdbmove foo -dbfile /eyedb/db/foo_1.db \
                -volfiles /eyedb/db/foo_1.vol
\end{verbatim}
\normalsize
\item complex creation and move using version \eyedbversion:
\verbsize
\begin{verbatim}
eyedbdbcreate foo -dbfile /usr/local/db/FOO.db \
                  -nobjs 200000000 \
                  -volfiles /usr/local/db/foo.vol1 10000000 \
                            /eyedb/db2/foo.vol2 1000000 \
                            /eyedb/db2/foo.vol3 100000000

eyedbdbmove foo -dbfile /usr/share/db/foo.db \
                -volfiles /usr/local/share/foo.vol1 10000000 \
                          /eyedb/db2/foo.vol8 1000000 \
                          /eyedb/DB7/FOO.vol3 100000000
\end{verbatim}
\normalsize
note that if one the size of the target volume files is wrong, the
operation failed.
\item same moving operation using version 2.4.6 or upper:
\verbsize
\begin{verbatim}
eyedbdbmove foo -dbfile /usr/share/db/foo.db \
                -volfiles /usr/local/share/foo.vol1 \
                          /eyedb/db2/foo.vol8 \
                          /eyedb/DB7/FOO.vol3 \
\end{verbatim}
\normalsize
\ei
\subsubsection{Copying a Database}
To copy a database one can use either \ttv{eyedbadmin copy} \emph{args}
or \ttv{eyedbdbcopy} \emph{args}.
The command line options for this operation are as follows:
\bi
\item \emph{dbname}: the database name. This argument is mandatory.
\item \emph{newdbname}: the new database name. This argument is mandatory.
\item \ttv{-dbfile} \emph{dbfile}:
the target database file. This argument is mandatory.
The object map file
and the log database file will be automatically created in the
same directory as this target database file.
\item \ttv{-volfiles} \emph{volfile size} [\emph{volfile size}] (version
2.4.0 to 2.4.5):
the target volume files and sizes. This argument is mandatory.
The restrictions for the use of this option are the same as for
the \ttv{eyedbdbmove} program.
\item \ttv{-volfiles} \emph{volfile} [\emph{volfiles}] (version
2.4.6 and upper):
the target volume files. This argument is mandatory.
The restrictions for the use of this option are the same as for
the \ttv{eyedbdbmove} program.
\ei
Here are some database creation and copying examples:
\bi
\item creation and copy using version 2.4.4:
\verbsize
\begin{verbatim}
eyedbdbcreate foo
eyedbdbcopy foo FOO -dbfile /eyedb/db/foo_1.db \
                -volfiles /eyedb/db/foo_1.vol 2000000
\end{verbatim}
\normalsize
\item same copying operation using version 2.4.6 or upper (the difference is that there is no
size arguments):
\verbsize
\begin{verbatim}
eyedbdbcopy foo FOO -dbfile /eyedb/db/foo_1.db \
                    -volfiles /eyedb/db/foo_1.vol
\end{verbatim}
\normalsize
\item complex creation and copy using version 2.4.4:
\verbsize
\begin{verbatim}
eyedbdbcreate foo FOO -dbfile /usr/local/db/FOO.db \
                      -nobjs 200000000 \
                      -volfiles /usr/local/db/foo.vol1 10000000 \
                                /eyedb/db2/foo.vol2 1000000 \
                                /eyedb/db2/foo.vol3 100000000

eyedbdbcopy foo FOO -dbfile /usr/share/db/foo.db \
                    -volfiles /usr/local/share/foo.vol1 10000000 \
                              /eyedb/db2/foo.vol8 1000000 \
                              /eyedb/DB7/FOO.vol3 100000000
\end{verbatim}
\normalsize
note that if one the size of the target volume files is wrong, the
operation failed.
\item same copying operation using version 2.4.6 or upper:
\verbsize
\begin{verbatim}
eyedbdbcopy foo -dbfile /usr/share/db/foo.db \
                -volfiles /usr/local/share/foo.vol1 \
                          /eyedb/db2/foo.vol8 \
                          /eyedb/DB7/FOO.vol3 \
\end{verbatim}
\normalsize
\ei

\subsubsection{Exporting a Database}
When you want to copy a database on another computer on a LAN or a WAN,
it is sometimes not possible to use \ttv{eyedbdbcopy}, because the
target computer is not visible as a file system from the source
computer.
\\
In this case, you must export the database to a file or device on the
source computer and import this database from a file or device
on the target computer.
\\
To export a database to a file, one can use either \ttv{eyedbadmin export} \emph{args}
or \ttv{eyedbdbexport} \emph{args}.
The command line options for this operation are as follows:
\bi
\item \emph{dbname}: the database name. This argument is mandatory.
\item \emph{file}: the target file or device (for instance /dev/rmt/1). Note
that if \emph{file} is omitted, the standard output is used.
\ei
For instance:
\bi
\item exporting a database to the standard input and compress output
using \ttv{gzip}:
\verbsize
\begin{verbatim}
eyedbdbexport foo | gzip -c > /tmp/fooDB.gz
\end{verbatim}
\normalsize
\item exporting a database to a tape:
\verbsize
\begin{verbatim}
eyedbdbexport foo /dev/rmt/1
\end{verbatim}
\normalsize
\item same operation but using \ttv{gzip}:
\verbsize
\begin{verbatim}
eyedbdbexport foo | gzip -c > /dev/rmt/1
\end{verbatim}
\normalsize
\ei
Note that the tied method modules are automatically included in the
exported file.
\subsubsection{Importing a Database}
To import a database from a file one can use either \ttv{eyedbadmin import} \emph{args}
or \ttv{eyedbdbimport} \emph{args}.
The command line options for this operation are as follows:
\bi
\item \emph{file}: the file from which importation is done.
This argument is mandatory. To use the standard input \emph{file}, must be
set to \ttv{-}.
\item \emph{-l}: if you want only to list the contents of the file, use
this option. In this case, no importation will be performed.
\item \ttv{-db} \emph{dbname}: the target database name.
This argument is not mandatory.
If you do not use this option, \ttv{eyedbdbimport} will ask for 
the target database name on the standard input (interactive mode).
\item \ttv{-voldir} \emph{voldir}: the target volume directory.
This argument is not mandatory.
If you do not use this option, \ttv{eyedbdbimport} will ask for 
the target volume directory on the standard input (interactive mode).
\item \ttv{-mthdir} \emph{mthdir}: the target method directory.
This argument is not mandatory.
If you do not use this option, \ttv{eyedbdbimport} will ask for 
the target method directory on the standard input (interactive mode).
\ei
Here are some database importation examples:
\bi
\item let us assume that the exportation has been done as follows:
\verbsize
\begin{verbatim}
% eyedbdbexport foo /eyedb/db/FOO.dat
\end{verbatim}
\normalsize
\item to list the contents of the exportation file:
\verbsize
\begin{verbatim}
% eyedbdbimport -l /eyedb/db/FOO.dat
Import File "/eyedb/db/FOO.dat"
 
  Database          : foo
  Dbid              : 21
  Volumes Count     : 1
  Total Size        : 3642468 bytes
  Sizeslot          : 16
 
  Database File     : 162304 bytes {16138095 bytes}
  Object Map File   : 196608 bytes {400007168 bytes}
  Volume File #1    : 3055616 bytes {3061456 bytes < 2000000 Kb}
 
  Method File (CC)  : 'foomthbe-2.4.6.so' {227940 bytes}
\end{verbatim}
\normalsize
\item to realize a non interactive importation by designating the value
of the target database, volume and method directory:
\verbsize
\begin{verbatim}
% eyedbdbimport /eyedb/db/FOO.dat -db FooTest -voldir /eyedb/TDB \
  -mthdir /eyedb/TDB/so
\end{verbatim}
\normalsize
\item to realize an interactive importation:
\verbsize
\begin{verbatim}
% eyedbdbimport /eyedb/db/FOO.dat
database [foo]? FooTest
Directory for Database File and Object Map File [/eyedb/db]? /eyedb/TDB
Directory for Volume File #1 [/eyedb/db]? /eyedb/TDB
Directory for Method File 'foomthbe-2.4.6.so' [/eyedb/db/so]? /eyedb/TDB/so
\end{verbatim}
\normalsize
\ei
{\bf Important note}: to import a database on a computer, the program
\ttv{eyedbdbimport} must run as the same UNIX user as the \eyedb server.
\section{Users}
\subsection{What is an \eyedb User?}
%An \eyedb user is \ul{not} a UNIX user.
An \eyedb user is composed of:
\be
\item a name,
\item a user identifier,
\item a type: \ttv{eyedb}, \ttv{unix} or \ttv{strict unix},
\item an optionnal password,
\ee
At user creation, the administrator chooses for a name, a type and
a password if the type is \ttv{eyedb} or \ttv{unix}. The user identifier
is automatically managed by \eyedb and is for internal use only.
\subsection{User Types}
\eyedb supports three types of \eyedb users:
\be
\item \ttv{eyedb user type}: the \ttv{eyedb user type} is the default type.
Such a user is not mapped on any UNIX user. So, you can create as
many users of this type as you want: for instance, you can create the
user \ttv{zolo} even if no UNIX user of this name exists on your computer.
When you create a user of this type, you need to give a password. Such
a user can be used either on a TCP/IP connection or a named pipe connection.
\item \ttv{unix user type}: when you create a user of this type, a
UNIX user with the same name must exists on the computer on which
the \eyedb server is running. When you create a user of this type,
you need to give a password. Such
a user can be used either on a TCP/IP connection or a named pipe connection.
A \ttv{unix user type} does not need to give a password if a named pipe
connection is used because a safe authentication is provided by
the UNIX kernel: this is the only difference with 
a \ttv{eyedb user type} user.
\item \ttv{strict unix user type}:
when you create a user of this type, a
UNIX user with the same name must exists on the computer on which
the \eyedb server is running. When you create a user of this type,
you does not need to give a password. Such
a user can be used only on a named pipe connection.
A \ttv{strict unix user type} does not need to give a password for connection
on a named pipe  because a safe authentication is provided by
the UNIX kernel.
\ee
\subsection{User Management}

\subsubsection{Adding a User}
To add a user one can use either \ttv{eyedbadmin useradd} \emph{args}
or \ttv{eyedbuseradd} \emph{args}.
The command line options for this operation are as follows:
\bi
\item \ttv{-unix}: if this option is set, the user will be a
\ttv{unix user type} user.
\item \ttv{-strict-unix}:
if this option is set, the user will be a
\ttv{strict unix user type} user.
\item \emph{username}: the user name. This argument is mandatory.
\item \emph{password}: a password is needed if the user is not
a \ttv{strict unix user type} user. If no password is given on
the command line, the program will ask for it on the standard input.
\ei
For instance:
\bi
\item adding a \ttv{eyedb user type} user named \ttv{john}:
\verbsize
\begin{verbatim}
% eyedbuser add john
john passwd:
retype john passwd:
\end{verbatim}
\normalsize
\item adding a \ttv{unix user type} user named \ttv{john}:
\verbsize
\begin{verbatim}
% eyedbuser -unix add john
john passwd:
retype john passwd:
\end{verbatim}
\normalsize
If \ttv{john} is not a UNIX user, then the following message is displayed:
\verbsize
\begin{verbatim}
eyedbuseradd: eyedb error: username 'john' is an unknown unix user.
\end{verbatim}
\normalsize
\item adding a \ttv{strict unix user type} user named \ttv{john} (no password
is asked):
\verbsize
\begin{verbatim}
% eyedbuser -strict-unix add john
\end{verbatim}
\normalsize
\ei
\subsubsection{Deleting a User}
To delete a user one can use either \ttv{eyedbadmin userdelete} \emph{args}
or \ttv{eyedbuserdelete} \emph{args}.
The command line options for this operation are as follows:
\bi
\item \emph{username}: the user name. This argument is mandatory.
\ei
\subsubsection{Listing Users}
To list one or more users one can use either \ttv{eyedbadmin userlist} \emph{args}
or \ttv{eyedbuserlist} \emph{args}.
The command line options for this operation are as follows:
\bi
\item \emph{users}: if this optionnal argument is present, the program
will list the designated users, otherwise all the users will be listed.
\ei
\subsubsection{Setting User Password}
To change a user password one can use either \ttv{eyedbadmin userpasswd} \emph{args}
or \ttv{eyedbuserpasswd} \emph{args}.
The command line options for this operation are as follows:
\bi
\item \emph{username}: the user name. This argument is mandatory
\item \emph{newpasswd}: the new password. If this argument is not given,
the program will ask for a new password on the standard input.
\ei
\section{Access Control}
Each \eyedb user has a set of authorized operations:
system and database operations.
A system (resp. database) operation is controled by a system access
(resp. database):
\be
\item a system access control the authorization to perform one
or more of the following system operations:
\be
\item create any database.
\item add any user.
\item delete any user.
\item set the password of any user.
\ee
\item a database access control the authorization to perform one
or more of the following operations on a given database:
\be
\item read from this database.
\item write to this database.
\item execute any method in this database.
\item delete this database.
\ee
\ee
Note that a system access is not tied to a particular database,
while a database access is tied to a particular database.
Here are the list of the authorized system operations according
to the system access:
\\
\\
\begin{tabular}{|p{4cm}|p{6cm}|}
\hline {\bf System Access} & {\bf Authorized Operations} \\
\hline \ttv{no} & no authorized system operation \\
\hline \ttv{dbcreate} & can create a database \\
\hline \ttv{adduser} & can add any user \\
\hline \ttv{deleteuser} & can delete any user \\
\hline \ttv{setuserpasswd} & can set the password of any user\\
\hline \ttv{admin} & can create a database and add a user\\
\hline \ttv{superuser} & all system accesses \\
\hline
\end{tabular}
\\
\\
Here are the list of the authorized operations on a given
database according to the database access:
\\
\\
\begin{tabular}{|p{4cm}|p{6cm}|}
\hline {\bf Database Access} & {\bf Authorized Operations} \\
\hline \ttv{no} & no database access \\
\hline \ttv{read} & have read access on a given database \\
\hline \ttv{write} & have write access on a given database \\
\hline \ttv{exec} & have execute access on a given database \\
\hline \ttv{admin} & have read/write/execute access on a given database and
can delete this database \\
\hline
\end{tabular}

\subsection{Setting System Access}
To set a system access to an \eyedb user,
one can use either \ttv{eyedbadmin sysaccess}
\emph{args}
or \ttv{eyedbsysaccess} \emph{args}.
The command line options for this operation are as follows:
\bi
\item \emph{username}: the user name. This argument is mandatory.
\item \emph{sysaccess}: the system access. This argument is mandatory.
It is a \ttv{+} combination of one or more of the following:
\ttv{no},
\ttv{dbcreate}, \ttv{adduser}, \ttv{deleteuser}, \ttv{setuserpasswd},
\ttv{admin} or \ttv{superuser}.
\ei
{\bf Important note}: only an \eyedb that has all the system accesses
(i.e. an \eyedb superuser) can set a system access for an \eyedb user.
Remember that at the installation step, a superuser
named \ttv{eyedbroot} has been created: so, you must connect the
\eyedb server as \ttv{eyedbroot} to set a system access to an
\eyedb user.
For instance:
\bi
\item to set the \ttv{dbcreate} system access to the user \ttv{john}:
\verbsize
\begin{verbatim}
% eyedbsysaccess john dbcreate -eyedbuser eyedbroot -eyedbpasswd eyedbroot
\end{verbatim}
\normalsize
\item to set the \ttv{dbcreate} and \ttv{useradd}
system access to the user \ttv{john}:
\verbsize
\begin{verbatim}
% eyedbsysaccess john dbcreate+useradd -eyedbuser eyedbroot \
                 -eyedbpasswd eyedbroot
\end{verbatim}
\normalsize
\item to suppress all system access to the user \ttv{john}:
\verbsize
\begin{verbatim}
% eyedbsysaccess john no -eyedbuser eyedbroot -eyedbpasswd eyedbroot
\end{verbatim}
\normalsize
\item to create a new \eyedb superuser:
\verbsize
\begin{verbatim}
% eyedbuseradd ROOT -eyedbuser eyedbroot -eyedbpasswd eyedbroot
ROOT password:
retype ROOT password:
% eyedbsysaccess ROOT superuser -eyedbuser eyedbroot -eyedbpasswd eyedbroot
\end{verbatim}
\normalsize
\ei

\subsection{Setting Database Access}
To set a database access to an \eyedb user,
one can use either \ttv{eyedbadmin userdbaccess}
\emph{args}
or \ttv{eyedbuserdbaccess} \emph{args}.
The command line options for this operation are as follows:
\bi
\item \emph{username}: the user name. This argument is mandatory.
\item \emph{dbname}: the database name. This argument is mandatory.
\item \emph{database access}: one of the following:
\ttv{r, rw, rx, rwx, admin} or \ttv{no} where \ttv{r} means read,
\ttv{w} means write and \ttv{x} means execute.
\ei
{\bf Important note}: only an \eyedb user that has the \ttv{admin} database
access for a given database
can set a database access on this database for an \eyedb user.
At the database creation step, the creating \eyedb user has automatically
the \ttv{admin} database access.
For instance:
\bi
\item to set the \ttv{r} database access to the user \ttv{john}:
\verbsize
\begin{verbatim}
% eyedbcreate foo
% eyedbuserdbaccess john foo r
\end{verbatim}
\normalsize
\item to suppress all database accesses to the user \ttv{john}:
\verbsize
\begin{verbatim}
% eyedbuserdbaccess john foo no
\end{verbatim}
\normalsize
\ei
\subsection{Setting Default Database Access}
To set a default database access for all \eyedb user,
one can use either \ttv{eyedbadmin dbaccess}
\emph{args}
or \ttv{eyedbdbaccess} \emph{args}.
The command line options for this operation are as follows:
\bi
\item \emph{dbname}: the database name. This argument is mandatory.
\item \emph{database access}: one of the following:
\ttv{r, rw, rx, rwx, admin} or \ttv{no} where \ttv{r} means read,
\ttv{w} means write and \ttv{x} means execute.
\ei
{\bf Important note}: only an \eyedb user that has the \ttv{admin} database
access for a given database
can set a default database access for this database.
At the database creation step, the creating \eyedb user has automatically
the \ttv{admin} database access.
For instance:
\bi
\item to set the \ttv{r} database access for all users:
\verbsize
\begin{verbatim}
% eyedbcreate foo
% eyedbdbaccess foo r
\end{verbatim}
\normalsize
\ei

\section{Storage Manager Level Administration}
The storage manager level administration program, \ttv{eyedbdbadmin}
allows for the following operations:
\be
\item information display: general information or statistics display.
\item log management: display the log size, the log allocation table.
Changes the log size or reformat the log.
reformat the
\item transaction management: display the transaction header or the
transaction contents.
\item mutex management: display the mutex state or reset any mutex.
\item oid display: displaying all the oids, the count of oids or the
invalid oids.
\item volume management: adding, inserting, moving, resizing or
suppress a volume file.
\item object number management: change the maximum object number.
\ee
The general usage of \ttv{eyedbdbadmin} is as follows:
\verbsize
\begin{verbatim}
usage: eyedbdbadmin <directive> <command> [arguments] <dbfile>

where directive is `display' or `configure'

 eyedbdbadmin display info <dbfile>
 eyedbdbadmin display stats <dbfile>
 eyedbdbadmin display transaction header [<server-pid>] <dbfile>
 eyedbdbadmin display transaction [<server-pid>] <dbfile>
 eyedbdbadmin display volume [<number>|<name>] <dbfile>
 eyedbdbadmin display log map <dbfile>
 eyedbdbadmin display log size <dbfile>
 eyedbdbadmin display mutex <dbfile>
 eyedbdbadmin display oid invalid <dbfile>
 eyedbdbadmin display oid count <dbfile>
 eyedbdbadmin display oid map <dbfile>
 eyedbdbadmin display oid list <dbfile>
 
 eyedbdbadmin configure volume add      <volname> <size (Kb)> <dbfile>
 eyedbdbadmin configure volume insert   <pos> <volname> <size (Kb)> <dbfile>
 eyedbdbadmin configure volume move     <volname> <newvolname> <dbfile>
 eyedbdbadmin configure volume resize   <volname> <newsize (Kb)> <dbfile>
 eyedbdbadmin configure volume suppress <volname> <dbfile>
 eyedbdbadmin configure volumes compress <dbfile>
 eyedbdbadmin configure log size        <newsize> <dbfile>
 eyedbdbadmin configure log format <dbfile>
 eyedbdbadmin configure object number <number> <dbfile>
 eyedbdbadmin configure mutex unlock <dbfile>
 eyedbdbadmin configure mutex reset <dbfile>
\end{verbatim}
\normalsize

%\section{Trouble Shotting}
%\be
%\item any \eyedb client core dumped immediately,
%\item the \eyedb server crashed at a given or at any database opening:
%the client
%displays \ttv{The EyeDB server has probably crashed or timed out}.
%\item the client displays a message such as \ttv{storage manager error:
%no space left on device} at a given database opening or at any database
%opening,
%\item a client freezes on a database opening.
%\item the client displays a message such as \ttv{storage manager error:
%no space left on device} during a long transaction.
%\ee

\end{document}
