\documentclass[dvips,10pt]{book}
%\documentclass[dvips,10pt]{report}

\usepackage{latexsym}
%\usepackage{amstex}
%\usepackage{amssymb}
%\usepackage{amsfonts}
\usepackage{amstext}
\usepackage{amsbsy}
\usepackage{amscd}
\usepackage{makeidx}
\usepackage{graphicx}
%\usepackage{calc}
\usepackage{longtable}

% -- to reduce the margin pages --
\topmargin 2pt
\headsep 20pt
\headheight 2pt

%\textwidth 440pt
%\textheight 690pt
%\oddsidemargin 0pt
%\evensidemargin 0pt

% 19/01/06: EV
\textwidth 490pt
\textheight 700pt
\oddsidemargin -20pt
\evensidemargin -20pt

% --------------------------------

\makeindex

\newcommand{\eyedb}{\textsc{EyeDB} }
\newcommand{\eyedbX}{\textsc{EyeDB}}
\newcommand{\sysra}{SYSRA}
\newcommand{\indexentry}[2]{\item #1 #2}
\newcommand{\chap}[1]{\chapter{#1} \index{#1}}
\newcommand{\sect}[1]{\section{#1} \index{#1}}
\newcommand{\subsect}[1]{\subsection{#1} \index{#1}}
\newcommand{\subsubsect}[1]{\subsubsection{#1} \index{#1}}
%% WARNING \Box does not work with latex2html
%%\newcommand{\iii}{\item[{$\Box$}]}
\newcommand{\rrr}{\item[{$\Box$}]}
\newcommand{\iii}{\item}
\newcommand{\jjj}{\item[-]}
\newcommand{\bi}{\begin{itemize}}
\newcommand{\ei}{\end{itemize}}
\newcommand{\be}{\begin{enumerate}}
\newcommand{\ee}{\end{enumerate}}
\newcommand{\kkk}{\item[ ]}
\newcommand{\sh}[1]{\bf #1}
\newcommand{\ttt}{$\tilde{ }$}
\newcommand{\ttv}[1]{\texttt{#1}}
\newcommand{\ttu}[1]{{\verbsize\texttt{#1}}}
\newcommand{\verbsizex}{\small}
\newcommand{\verbsizey}{\fontsize{8}{\baselineskip}\selectfont}
\newcommand{\verbsize}{\normalsize}
\newcommand{\bugreport}{bug-report@eyedb.org }
\newcommand{\idt}{\mbox{ } - }
\newcommand{\ident}{{\bf \texttt{<}identifier\texttt{>}}}
\newcommand{\grindent}{xxxxxxxxxxxxxxx \= : \kill}
\newcommand{\arr}{\texttt{-$>$}}
\newcommand{\ul}[1]{\underline{#1}}
\newcommand{\uul}[1]{\textit{\texttt{#1}}}
\newcommand{\ixx}{\makebox[3mm]{ }}
\newcommand{\ixy}{\makebox[10mm]{ }}
\newcommand{\ixz}{\\\ixy {- }}

\renewcommand{\normalsize}{\fontsize{9}{11pt}\selectfont}

%\renewcommand{\normalsize}{\fontsize{8}{10pt}\selectfont}

%% The following commands comes from corbaref.tex

\newcommand{\bt}{\begin{tabular}{|p{7cm}|p{7cm}|}}
\newcommand{\et}{\hline \end{tabular} \\ {\vspace{0.6cm}} \\ }

\newcommand{\mt}[1]{\hline \multicolumn{2}{|c|}{\centerline{{\bf interface #1}
   \emph{mapped from} {\bf class #1}}}}

\newcommand{\Desc}[1]{\hline \multicolumn{2}{|l|}{\emph{#1}}}

\newcommand{\mTT}[2]{\hline \multicolumn{2}{|c|}
   {\centerline{{\bf interface #1} \emph{extends} {\bf #2}
   \emph{mapped from} {\bf #1}}}}


\newcommand{\mT}[2]{\hline \multicolumn{2}{|c|}{\centerline{{\bf #1}
\emph{mapped from} {\bf #1}}}}

\newcommand{\att}{\hline \hline {\bf Attribute} & {\bf Description}}
\newcommand{\mtt}{\hline \hline {\bf Method} & {\bf Description}}
\newcommand{\mf}{\emph{mapped from }}

\newcommand{\satt}[1]{\hline \multicolumn{2}{|c|}
   {\centerline{\emph{attributes inherited from} {\bf #1}}}}

\newcommand{\smtt}[1]{\hline \multicolumn{2}{|c|}{
   \centerline{\emph{methods inherited from} {\bf #1}}}}

\newcommand{\sets}[1]{sets the #1 \emph{value} in the Object \emph{o} at the \emph{ind}}

\newcommand{\Gets}[1]{gets the #1 value in the Object \emph{o} at the \emph{ind}}

%\setcounter{secnumdepth}{-1}
%\addtocounter{secnumdepth}{2}

\renewcommand{\thesection}{\arabic{section}}
\renewcommand{\thesubsection}{\arabic{section}.\arabic{subsection}}
\renewcommand{\thechapter}{}

\begin{document}
\normalsize


\input{version}
\newcommand{\mantitle}{\textsc{Getting Started}}

\thispagestyle{empty}

\mbox{ }
\\
\vspace{3 cm}
\\
%{\bf{\Huge E{\LARGE YE}DB} \hspace{0.cm} {\LARGE Programming Manual}}
{\bf{\Huge E{\LARGE YE}DB} \hspace{0.cm} {\Huge \mantitle}}
\\
%\newcommand{\rulewidth}{12.2cm}
\newcommand{\rulewidth}{15.4cm}
\rule{\rulewidth}{1.6mm}
\begin{flushright}
{\large Version \eyedbversion}
\end{flushright}
\mbox{ }
\\
\vspace{11 cm}
\\
\begin{flushright}
{\large January 2006}
\end{flushright}

\newpage

\thispagestyle{empty}

\mbox{ }
\\
\vspace{1cm}
\\
Copyright {\copyright} 2001-2006 SYSRA
\\
\vspace{1cm}
\\
Published by \sysra
\\
30, avenue G\'en\'eral Leclerc
\\
91330 Yerres - France
\\
\\
home page: http://www.eyedb.org
\\
\\
bug report: \bugreport


\tableofcontents
%\listoftables

\newcommand{\EYEDBARCH}{\emph{eyedbtop}/\texttt{eyedb-V\eyedbversion}/\emph{archdir}}

\chapter{Getting Started}

We will introduce \eyedb by working through some simple administration
operations such as adding a user, creating a database, and through
a simple example of defining an ODL schema, and then writing a C++
client program.
\\
At this early stage, we will explain only what is necessary to understand
the examples to avoid being swamped by unnecessary details.
\\
\\
We assume that \eyedb has been correctly installed on your
computer.
Refer to the installation guide in the annex for installation information.
\\
\\
Depending on who you are (the DBA, the installer, a lambda user), you do not need
to read all this chapter. In particular, the first sections about the
distribution, the environment and administrative operations 
can be skipped if you are a lambda user. In this case, ask your DBA to create
for you an \eyedb user that has the \emph{database creation} system privilege. 
Then you can jump directly to the section
\emph{Using the default configuration} and then to the section
\emph{Creating a database using a specific user}.
\sect{The Distribution}
When you installed \eyedbX, you specified a target directory, for instance
\texttt{/usr/local/eyedb}.
\\
If you look at the directory contents, you will find the following directories
and files:
\begin{tabbing}
- \texttt{eyedb-V\eyedbversion}\mbox{ } \= : directory \\
- \texttt{eyedb} \> : symbolic link to eyedb-V\eyedbversion \\
- \texttt{etc} \> : directory containing environment files \\
- \texttt{db} \> : default directory containing the databases \\
- \texttt{log} \> : directory containing log files
\end{tabbing}
The actual binary distribution is under the directory
\texttt{/usr/local/eyedb/eyedb-V\eyedbversion} which is equivalent (because of the
symbolic link) to \texttt{/usr/local/eyedb/eyedb}.
\\
\\
This distribution architecture, and in particular the symbolic link
\texttt{eyedb} has been introduced to facilitate the installation of the
next \eyedb versions.
\\
\\
Under the binary distribution directory \texttt{eyedb-V\eyedbversion}, you will find
the following directories:
\begin{tabbing}
- \texttt{examples} \= : a few examples \kill
- \texttt{doc} \> : documentation \\
- \texttt{etc} \> : environment files \\
- \texttt{examples} \> : a few examples \\
- \texttt{include} \> : C++ binding include files \\
- \texttt{java} \> : Java classes \\
- \emph{archdir} \> : one or more architecture directories
\end{tabbing}
The architecture directories contain the libraries and executables that
are suitable to your computer, operating system and compiler.
\\
The name of an architecture directory is built as follows:
%\[debug-\]\emph{os}-\emph{proc}\[-\emph{compiler}\]
%[debug-\]\emph{os}-\emph{proc}[-\emph{compiler}]
\begin{verbatim}
[debug-]os-proc[-compiler]
\end{verbatim}
for instance:\\
- \texttt{debug-solaris-sparcv7-gcc} for the Solaris Sparc 32 bits debug version
using the GNU C++ compiler.\\
- \texttt{solaris-sparcv9-cc} for the Solaris Sparc 64 bits optimized version
using the SUN C++ compiler.\\
- \texttt{linux-x86} for the Linux X86 optimized version
using the default C++ compiler (GNU).\\
\\
Under an architecture directory, you will find
the following directories:
\begin{tabbing}
- \texttt{xetc} \= : contains the architecture environment files \kill
- \texttt{etc} \> : contains the architecture environment files \\
- \texttt{lib} \> : contains the C++ libraries for CC and gcc \\
- \texttt{bin} \> : contains the \eyedb tools prefixed with \texttt{eyedb} \\
- \texttt{xbin} \> : contains the \eyedb tools not prefixed with \texttt{eyedb} \\
\end{tabbing}
\subsect{The Environment}
All what concern the environment is under the directories \texttt{etc}.
This directory {\EYEDBARCH} contains the following
files:
\begin{tabbing}
- \texttt{Makefile}.eyedb \= : \kill
- \texttt{environ}.sh \> : the Bourne shell environment script \\
- \texttt{environ}.csh \> : the C Shell environment script \\
- \texttt{ArchConfig} \> : the configuration file for your architecture \\
- \texttt{Makefile}.eyedb \> : the template Makefile \\
\end{tabbing}
\subsection{Using the default configuration}
To use \eyedb with its default configuration, you only need to source
the shell script file environ.sh (Bourne Shell) or environ.csh (C Shell)
as follows:
\begin{tabbing}
\mbox{ } - Bourne shell environment \= : {\LARGE .} \emph{eyedbtop}/\texttt{eyedb-V\eyedbversion}/\emph{archdir}/etc/environ.sh \\
%\mbox{ } - C shell environment \> : source {\EYEDBARCH}/etc/environ.csh
\mbox{ } - C shell environment \> : source {\EYEDBARCH}/etc/environ.csh
\end{tabbing}
where \emph{eyedbtop} denotes the directory in which \eyedb has been installed, for instance \texttt{/usr/local/eyedb} and \emph{archdir}, the architecture
directory, for instance \texttt{solaris-sparcv7-cc}.
\\
These shell scripts set a few environment variables, \texttt{EYEDBROOT},
\texttt{EYEDBARCH} and \texttt{EYEDBCLASSPATH} and modify the \texttt{PATH} 
and the \texttt{LD\_LIBRARY\_PATH} environment variables to allow you to execute properly the \eyedb
tools.
\\
Once you have sourced one of this environment file, you are able to use
\eyedbX.
\subsection{The \eyedb tools}
If you use a modern shell such as \texttt{bash} or \texttt{tcsh}
which provides an interactive completion for programs, by typing:
\\
\mbox{ } - \texttt{eyedb}\emph{TAB TAB} on your terminal if you use \texttt{bash} \\
\mbox{ } - \texttt{eyedb}\emph{Ctl-D} on your terminal if you use \texttt{tcsh} \\
all the \eyedb tools will be displayed on your terminal as follows:
\begin{tabbing}
xxxxxxxxxxxxxxxxxxxx \= xxxxxxxxxxxxxxxxxxxx \= xxxxxxxxxxxxxxxxxxxx \= xxxxxxx
xxxxxxxxxxxxx \kill
\texttt{eyedbadmin} \> \texttt{eyedbdbmcreate} \> \texttt{eyedbdbmove} \> \texttt{eyedbjrun} \\
\texttt{eyedbdbrename} \> \texttt{eyedbloca} \> \texttt{eyedbcollimpladmin} \> \texttt{eyedbdspcreate} \\
\texttt{eyedbmvattr} \> \texttt{eyedbcollimplgetdef} \> \texttt{eyedbdspdelete} \> \texttt{eyedbmvcoll} \\
\texttt{eyedbcollimpllist} \> \texttt{eyedbdspgetcurdat} \> \texttt{eyedbmvidx} \> \texttt{eyedbcollimplsetdef} \\
\texttt{eyedbdspgetdefault} \> \texttt{eyedbmvinst} \> \texttt{eyedbcollimplsimulate} \> \texttt{eyedbdsplist} \\
\texttt{eyedbmvobj} \> \texttt{eyedbcollimplstats} \> \texttt{eyedbdsprename} \> \texttt{eyedbodl} \\
\texttt{eyedbcollimplupdate} \> \texttt{eyedbdspsetcurdat} \> \texttt{eyedboql} \> \texttt{eyedbconsadmin} \\
\texttt{eyedbdspsetdefault} \> \texttt{eyedbpasswd} \> \texttt{eyedbconscreate} \> \texttt{eyedbdspupdate} \\
\texttt{eyedbprotadmin} \> \texttt{eyedbconsdelete} \> \texttt{eyedbgetattrloca} \> \texttt{eyedbconslist} \\
\texttt{eyedbgetcollloca} \> \texttt{eyedbgetdefattrdsp} \> \texttt{eyedbsetdefattrdsp} \> \texttt{eyedbgetdefcolldsp} \\
\texttt{eyedbsetdefcolldsp} \> \texttt{eyedbdatcreate} \> \texttt{eyedbgetdefdsp} \> \texttt{eyedbsetdefdsp} \\
\texttt{eyedbdatdefragment} \> \texttt{eyedbgetdefidxdsp} \> \texttt{eyedbsetdefidxdsp} \> \texttt{eyedbdatdelete} \\
\texttt{eyedbgetdefinstdsp} \> \texttt{eyedbsetdefinstdsp} \> \texttt{eyedbdatlist} \> \texttt{eyedbgetenv} \\
\texttt{eyedbdatmove} \> \texttt{eyedbgetidxloca} \> \texttt{eyedbsmtool} \> \texttt{eyedbdatrename} \\
\texttt{eyedbgetinstloca} \> \texttt{eyedbsysaccess} \> \texttt{eyedbdatresize} \> \texttt{eyedbgetobjloca} \\
\texttt{eyedbuseradd} \> \texttt{eyedbdbaccess} \> \texttt{eyedbidxadmin} \> \texttt{eyedbuserdbaccess} \\
\texttt{eyedbdbcopy} \> \texttt{eyedbidxcreate} \> \texttt{eyedbuserdelete} \> \texttt{eyedbdbcreate} \\
\texttt{eyedbidxdelete} \> \texttt{eyedbuserlist} \> \texttt{eyedbdbdelete} \> \texttt{eyedbidxlist} \\
\texttt{eyedbuserpasswd} \> \texttt{eyedbdbexport} \> \texttt{eyedbidxsimulate} \> \texttt{eyedbwconf} \\
\texttt{eyedbdbimport} \> \texttt{eyedbidxstats} \> \texttt{eyedbwwwd} \> \texttt{eyedbdblist} \\
\texttt{eyedbidxupdate} \\

%\texttt{eyedbadmin} \> \texttt{eyedbdbimport} \> \texttt{eyedbgetdefcolldsp} \>\texttt{eyedbpasswd} \\
%\texttt{eyedbcgiclean} \> \texttt{eyedbdblist} \> \texttt{eyedbgetdefdsp} \>\texttt{eyedbpatchbin} \\
%\texttt{eyedbcgife} \> \texttt{eyedbdbmcreate} \> \texttt{eyedbgetdefidxdsp} \>\texttt{eyedbrc} \\
%\texttt{eyedbcrypt} \> \texttt{eyedbdbmove} \> \texttt{eyedbgetdefinstdsp} \>\texttt{eyedbsetdefattrdsp} \\
%\texttt{eyedbd} \> \texttt{eyedbdbrename} \> \texttt{eyedbgetenv} \>\texttt{eyedbsetdefcolldsp} \\
%\texttt{eyedbdatcreate} \> \texttt{eyedbdspcreate} \> \texttt{eyedbgetidxloca} \>\texttt{eyedbsetdefdsp} \\
%\texttt{eyedbdatdefragment} \> \texttt{eyedbdspdelete} \> \texttt{eyedbgetinstloca} \>\texttt{eyedbsetdefidxdsp} \\
%\texttt{eyedbdatdelete} \> \texttt{eyedbdspgetcurdat} \> \texttt{eyedbgetobjloca} \>\texttt{eyedbsetdefinstdsp} \\
%\texttt{eyedbdatlist} \> \texttt{eyedbdspgetdefault} \> \texttt{eyedbjrun} \>\texttt{eyedbsmtool} \\
%\texttt{eyedbdatmove} \> \texttt{eyedbdsplist} \> \texttt{eyedbloca} \>\texttt{eyedbsysaccess} \\
%\texttt{eyedbdatrename} \> \texttt{eyedbdsprename} \> \texttt{eyedbmvattr} \>\texttt{eyedbuseradd} \\
%\texttt{eyedbdatresize} \> \texttt{eyedbdspsetcurdat} \> \texttt{eyedbmvcoll} \>\texttt{eyedbuserdbaccess} \\
%\texttt{eyedbdbaccess} \> \texttt{eyedbdspsetdefault} \> \texttt{eyedbmvidx} \>\texttt{eyedbuserdelete} \\
%\texttt{eyedbdbcopy} \> \texttt{eyedbdspupdate} \> \texttt{eyedbmvinst} \>\texttt{eyedbuserlist} \\
%\texttt{eyedbdbcreate} \> \texttt{eyedbgetattrloca} \> \texttt{eyedbmvobj} \>\texttt{eyedbuserpasswd} \\
%\texttt{eyedbdbdelete} \> \texttt{eyedbgetcollloca} \> \texttt{eyedbodl} \>\texttt{eyedbwconf} \\
%\texttt{eyedbdbexport} \> \texttt{eyedbgetdefattrdsp} \> \texttt{eyedboql} \>\texttt{eyedbwwwd} \\

%\texttt{eyedbadmin} \>          \texttt{eyedbdbcreate}      \> \texttt{eyedbpasswd} \> \texttt{eyedbcgiclean}\\
%\texttt{eyedbdbdelete} \>       \texttt{eyedbpatchbin}\> \texttt{eyedbcgife} \>          \texttt{eyedbdbexport}\\
%\texttt{eyedbrc} \> \texttt{eyedbcrypt}         \> \texttt{eyedbdbimport} \>       \texttt{eyedbsmtool}\\
%\texttt{eyedbd} \>              \texttt{eyedbdblist}        \> \texttt{eyedbsysaccess} \> \texttt{eyedbdatcreate}\\
%\texttt{eyedbdbmcreate} \>      \texttt{eyedbuseradd}\> \texttt{eyedbdatdefragment} \>  \texttt{eyedbdbmove}\\
%\texttt{eyedbuserdbaccess} \> \texttt{eyedbdatdelete}     \> \texttt{eyedbdbrename} \>       \texttt{eyedbuserdelete}\\
%\texttt{eyedbdatlist} \>        \texttt{eyedbgetenv}        \> \texttt{eyedbuserlist} \> \texttt{eyedbdatmove}\\
%\texttt{eyedbjrun} \>           \texttt{eyedbuserpasswd}\> \texttt{eyedbdatresize} \>      \texttt{eyedbodl}\\
%\texttt{eyedbwconf} \> \texttt{eyedbdbaccess}      \> \texttt{eyedboql} \>            \texttt{eyedbwwwd}\\
%\texttt{eyedbdbcopy} \>         \texttt{eyedborbixd}\\      

\end{tabbing}
But do not worry, you do not need to know all these tools to use
\eyedbX, only a few of them are commonly used.
\\
The directory {\EYEDBARCH}/\texttt{xbin} contains symbolic links to 
these main tools; the name links are not prefixed with \texttt{eyedb}.
For instance, in this directory
you will find \texttt{odl} which is a symbolic link to {\EYEDBARCH}/\texttt{bin}/\texttt{eyedbodl}. As the directory \texttt{xbin} has been set in
your \texttt{PATH} environment variables, you can use either \texttt{eyedbodl}
or \texttt{odl}.
\\
Here is the contents of the {\EYEDBARCH}/\texttt{xbin} directory:
\begin{tabbing}
xxxxxxxxxxxxxxxxxxxx \= xxxxxxxxxxxxxxxxxxxx \= xxxxxxxxxxxxxxxxxxxx \= xxxxxxxxxxxxxxxxxxxx \kill
\texttt{collimpladmin} \> \texttt{collimplgetdef} \> \texttt{collimpllist} \> \texttt{collimplsetdef} \\
\texttt{collimplsimulate} \> \texttt{collimplstats} \> \texttt{collimplupdate} \> \texttt{consadmin} \\
\texttt{conscreate} \> \texttt{consdelete} \> \texttt{conslist} \> \texttt{datcreate} \\
\texttt{datdefragment} \> \texttt{datdelete} \> \texttt{datlist} \> \texttt{datmove} \\
\texttt{datrename} \> \texttt{datresize} \> \texttt{dbaccess} \> \texttt{dbadmin} \\
\texttt{dbcopy} \> \texttt{dbcreate} \> \texttt{dbdelete} \> \texttt{dbexport} \\
\texttt{dbimport} \> \texttt{dblist} \> \texttt{dbmcreate} \> \texttt{dbmove} \\
\texttt{dbrename} \> \texttt{dspcreate} \> \texttt{dspdelete} \> \texttt{dspgetcurdat} \\
\texttt{dspgetdefault} \> \texttt{dsplist} \> \texttt{dsprename} \> \texttt{dspsetcurdat} \\
\texttt{dspsetdefault} \> \texttt{dspupdate} \> \texttt{getattrloca} \> \texttt{getcollloca} \\
\texttt{getdefattrdsp} \> \texttt{getdefcolldsp} \> \texttt{getdefdsp} \> \texttt{getdefidxdsp} \\
\texttt{getdefinstdsp} \> \texttt{getidxloca} \> \texttt{getinstloca} \> \texttt{getobjloca} \\
\texttt{idxadmin} \> \texttt{idxcreate} \> \texttt{idxdelete} \> \texttt{idxlist} \\
\texttt{idxsimulate} \> \texttt{idxstats} \> \texttt{idxupdate} \> \texttt{mvattr} \\
\texttt{mvcoll} \> \texttt{mvidx} \> \texttt{mvinst} \> \texttt{mvobj} \\
\texttt{odl} \> \texttt{oql} \> \texttt{setdefattrdsp} \> \texttt{setdefcolldsp} \\
\texttt{setdefdsp} \> \texttt{setdefidxdsp} \> \texttt{setdefinstdsp} \> \texttt{smtool} \\
\texttt{sysaccess} \> \texttt{useradd} \> \texttt{userdbaccess} \> \texttt{userdelete} \\
\texttt{userlist} \> \texttt{userpasswd} \\
%\texttt{datcreate} \>      \texttt{dbaccess} \>       \texttt{dbimport} \>       \texttt{oql}\\
%\texttt{userlist} \> \texttt{datdefragment} \>  \texttt{dbadmin} \>        \texttt{dblist}\\
%\texttt{smtool} \>         \texttt{userpasswd} \> \texttt{datdelete} \>      \texttt{dbcopy}\\
%\texttt{dbmcreate} \>      \texttt{sysaccess} \> \texttt{datlist} \>        \texttt{dbcreate}\\
%\texttt{dbmove} \>         \texttt{useradd} \> \texttt{datmove} \>        \texttt{dbdelete}\\
%\texttt{dbrename} \>       \texttt{userdbaccess} \> \texttt{datresize} \>      \texttt{dbexport}\\
%\texttt{odl} \>            \texttt{userdelete}
\end{tabbing}
\sect{Starting the server}
%\emph{This Section concerns DBAs}.\\
For any \eyedb operation, a server must run on your computer.
\\
To see if a server is running, type this:
\verbsize \begin{verbatim}
% eyedbrc status
\end{verbatim}
\normalsize
Two kinds of messages could be displayed:
\be
\item 
\verbsize \begin{verbatim}
EyeDB Server is up from Mon Nov  2 13:06:24 2000
 
  Version 2.6.6
  Program Pid 765
  Running Under 'eyedb'
 
  Listening Ports 6140
  Volume Dir '/eyedb/db'
 
  No Clients connected.
\end{verbatim}
\normalsize
means that an \eyedb server is running and all is fine.
\item \verbsize \begin{verbatim}
EyeDB server, port 6140, is down from Mon Nov  2 16:04:15 2000
\end{verbatim}
\normalsize
means that no \eyedb server is running, and you must start one.
\ee
To start a server, just do this:
\verbsize \begin{verbatim}
% eyedbrc start
\end{verbatim}
\normalsize
Then, you may try again \texttt{eyedbrc status}.\\
\\
If you get any trouble at this step, refer to the administration chapter.
\sect{Creating a database}
The next step is to create a database to perform our tests.
\\
This is very simple, you can try as follows:
\verbsize
\begin{verbatim}
% eyedbdbcreate foo
\end{verbatim}
or more simply:
\begin{verbatim}
% dbcreate foo
\end{verbatim}
\normalsize
If you use the default installation configuration, the following message
will be displayed:
\verbsize \begin{verbatim}
eyedbdbcreate: insufficient privileges: user 'guest': creating database 'foo'
\end{verbatim}
\normalsize
This mean that all is ok: you - in fact user \texttt{guest} - was not
authorized to create a database.
\\
\\
Two questions: \\
\mbox{ } 1- why are you user \texttt{guest}? \\
\mbox{ } 2- why the user \texttt{guest} does not have the privilege to create a database? \\
\\
The answer to the first question is very simple: because the default
configuration file is\\{\EYEDBARCH}\texttt{/etc/ArchConfig}
which includes the following lines:
\verbsize \begin{verbatim}
#
# Default Authentication
#
 
user   = guest;
passwd = guest;
\end{verbatim}
\normalsize
The answer to the second question is a little bit more complex.
\\
If you display the user \texttt{guest} as follows:
\verbsize \begin{verbatim}
% eyedbuserlist guest
name      : "guest"
sysaccess : idbNoSysAccessMode
\end{verbatim}
\normalsize
the second line indicates that the user \texttt{guest} has no system access
privileges:
\\
\\
The system access privileges control the following operations: \\
\mbox{ } - any database creation \\
\mbox{ } - any user creation \\
\mbox{ } - any user deletion \\
\mbox{ } - any user password setting \\
\mbox{ } - any database administration \\
\mbox{ } - any operation (or superuser access) \\
\\
To create a database, one needs the first system access privilege.
\\
\\
So how to create a database?\\
There are two ways:\\
\mbox{ } 1- using an \eyedb superuser, \\
\mbox{ } 2- creating a new user that has the appropriate system privilege
and use it.
\\
\\
We are going to introduce these two ways.

\subsect{Using an \eyedb superuser}
We need first to know the name of an \eyedb superuser. During the installation
process, three \eyedb users were created as shown by using the tool
\texttt{eyedbuserlist} as follows:
\verbsize
\begin{verbatim}
% eyedbuserlist
name      : "eyedbroot"
sysaccess : idbSuperUserSysAccessMode
 
name      : "eyedb" [strict unix user]
sysaccess : idbSuperUserSysAccessMode

name      : "guest"
sysaccess : idbNoSysAccessMode
\end{verbatim}
\normalsize
You can notice that \texttt{eyedbroot} and \texttt{eyedb}
- mentionned as a \texttt{strict unix user} - have
the \emph{super user} system privilege which means that they can perform
any operation on any database. The difference between these two users
is beyond the scope of this chapter. In this Section, we will use 
the \texttt{eyedbroot} user.
\\
\\
To create a database using the \texttt{eyedbroot} user, you need to do
as follows:
\verbsize
\begin{verbatim}
% eyedbdbcreate foo -U eyedbroot -P
password authentication : {eyedbroot password}
\end{verbatim}
\normalsize
You need to type the \texttt{eyedbroot} password that has been entered
during the installation process (the embraces around the password means that
these characters are not echoed on your terminal).
\subsect{Creating a user with a system privilege}
To create a user, you need to be a user that has the system privilege
\emph{any user creation}.
\\
The simplest way is to use the \eyedb superuser \texttt{eyedbroot}.
\\
Note that an \eyedb user is not a UNIX user. \eyedb provides its own
user management. But an \eyedb user can be mapped on a UNIX user.
Refer to the security section of the administation chapter to get
all information about that.
\\
\\
Let's assume you want to add the user \texttt{bill}, and try this:
\verbsize \begin{verbatim}
% eyedbuseradd bill
\end{verbatim}
\normalsize
Once again, an error message will be displayed:
\verbsize \begin{verbatim}
eyedbuseradd: insufficient privileges: user 'guest': adding user
\end{verbatim}
\normalsize
because you need to perform this operation under the user \texttt{eyedbroot}
and you are still user \texttt{guest}.
\\
So, you need to use the option \texttt{-eyedbauth?} as follows:
\verbsize \begin{verbatim}
% eyedbuseradd -U eyedbroot -P {eyedbroot password} bill
bill password: {thekid}
retype bill password: {thekid}
\end{verbatim}
\normalsize
If no error messages appear, the operation is sucessful.
\\
\\
Now, you need to add the \emph{any database creation} system privilege
to the user bill.
\\
Try this:
\verbsize \begin{verbatim}
% eyedbsysaccess
usage: eyedbsysaccess <username> ['+' combination of] dbcreate|adduser|deleteuser|setuserpasswd|admin|superuser|no
\end{verbatim}
\normalsize
As shown above, the usage is displayed on your terminal.
\\
For instance if you want to set the \emph{database creation} and the
\emph{user creation} system privileges:
\verbsize \begin{verbatim}
% eyedbsysaccess -U eyedbroot -P {eyedbroot password} bill dbcreate+adduser 
password authentication: {eyedbroot}
\end{verbatim}
\normalsize
\subsect{Creating a database using a specific user}
Using an \eyedb user (for instance \texttt{bill}) that has the \emph{database
creation} system privilege, you can create a database as follows:
\verbsize \begin{verbatim}
% eyedbdbcreate -U bill -P {bill password} foo
\end{verbatim}
\normalsize
or as follows:
\verbsize \begin{verbatim}
% eyedbdbcreate foo --user=bill --passwd=thekid
\end{verbatim}
\normalsize
Nice, isn't it?
\\
\\
The problem is that you need to type your user and password each time you
want to become the user \texttt{bill} because of the default configuration
you are still the user \texttt{guest}, and this user cannot acces the
database \texttt{foo} for read or write operations.
\\
For instance, try:
\verbsize \begin{verbatim}
% eyedboql -d foo
insufficient privileges: user 'guest': opening database 'foo'
\end{verbatim}
\normalsize
Of course, it is possible to allow the user \texttt{guest} to perform
read or write operations on the database \texttt{foo}, but it will be
more clever to become the user \texttt{bill}.
\subsect{Customizing your environment}
It is very sumple to customize your environment:
create a file, for instance EyeDBConfig in your home
directory which contains the following lines:
\verbsize \begin{verbatim}
include //etc/ArchConfig
 
user   = bill;
passwd = thekid;
\end{verbatim}
\normalsize
Change the file access mode to 0600 for security reasons, as follows:
\verbsize \begin{verbatim}
% chmod 0600 EyeDBConfig
\end{verbatim}
\normalsize
then, set the environment variable EYEDBCONF to this file as follows:
\verbsize \begin{verbatim}
% EYEDBCONF=~/EyeDBConfig; export EYEDBCONF # for Bourne Shells
% setenv EYEDBCONF ~/EyeDBConfig # for C Shells
\end{verbatim}
\normalsize
To check, if this operation worked, try:
\verbsize \begin{verbatim}
% eyedbgetenv version
2.6.6
% eyedbgetenv user
bill
\end{verbatim}
\normalsize
if the good version number \texttt{\eyedbversion} and \texttt{bill} are displayed,
the operation has succeeded.
\\
Now you can try again:
\verbsize \begin{verbatim}
% eyedboql -d foo
Welcome to eyedboql. Type `!help' to display the command list.
? 
\end{verbatim}
\normalsize
this is better. Type \texttt{!quit} or \texttt{Control-D} to exit from
eyedboql.
\sect{Defining a simple schema}
Now that a database has been created, a legitimate wish is to input data
in it.
\\
The first step is to define the database schema.
\\
A standard example in database documentations is the well known
\texttt{Person} class (or table in relational system) which contains
a few attributes such as a firstname, a lastname, an age, an address, a spouse
and a set of children.
\\
We will show the inheritance feature through the simple class \texttt{Employee}
which inherits from the \texttt{Person} class and will contains a simple
attribute: salary.
\\
\\
There are several ways to create such a schema:\\
\mbox{ } 1- define the schema in ODL and use the \texttt{eyedbodl} tool. \\
\mbox{ } 2- use the C++ binding. \\
\mbox{ } 3- use the Java binding. \\
\\
All these choices are valid however the last *two* are too complicated
to describe here. Please refer to the appropriate chapters.
\\
\\
So we opt for the first choice which is the most pratical
and flexible way to define and create a schema within a database.
\\
\\
Here is simple ODL schema for the classes \texttt{Address}, \texttt{Person} 
and \texttt{Employee}:
\verbsize \begin{verbatim}
//
// person.odl
//

class Address {
  int num;
  string street;
  string town;
  string country;
};

class Person {
  string firstname;
  string lastname;
  int age;
  Address addr;
  Person * spouse inverse Person::spouse;
  set<Person *> children;
};

class Employee extends Person {
  long salary;
};
\end{verbatim}
\normalsize
A few comments about this schema:
\bi
\item the \texttt{Address} class contains four attributes, one integer
and three strings.
\bi
\item {\bf integer}: there are three types of ODL integers: \\
\mbox{ } 1- 16-bits integer, names \texttt{int16} or \texttt{short} \\
\mbox{ } 2- 32-bits integer, names \texttt{int32} or \texttt{int} \\
\mbox{ } 3- 64-bits integer, names \texttt{int64} or \texttt{long} \\
so the \texttt{num} attribute is a 32-bits integer.
\item {\bf string}: an ODL string is under the form: \texttt{string} or
\texttt{string<N>}. The
first form means that the string is not bounded, the second form means
that the string contains at most \texttt{N} characters.
\ei
\item the \texttt{Person} class contains six attributes: two strings, one
32-bits integer, one \texttt{Person} object and one set of \texttt{Person}
objects.
\bi
\item the third attribute \texttt{addr} is of \texttt{Address} type and is
a litteral because there is no \texttt{*} before the attribute name.\\
A litteral is an object without identifier: the \texttt{addr} attribute
is tied to a \texttt{Person} instance, it has no proper existence.
\item the \texttt{spouse} attribute is an object, not a litteral, because
it is preceded by a \texttt{*}. An object has an identifier and has its
proper existence. The \texttt{*} means a reference or pointer to an
object.\\
The directive after the attribute name \texttt{inverse Person::spouse}
is a relationship directive.
\item the \texttt{children} attribute is a collection set of
\texttt{Person} objects.
\ei
\item the \texttt{Employee} contains seven attributes: the six \texttt{Person}
attributes because \texttt{Employee} inherits from \texttt{Person} and 
64-bits integer attribute: \texttt{salary}.
\ei

To add the previous schema in the \texttt{foo} database, you need to
use the \texttt{eyedbodl} tool as follows:
\verbsize \begin{verbatim}
% eyedbodl -d foo -u --package=person person.odl
Updating 'person' schema in database foo...
Adding class Address
Adding class Person
Adding class Employee

Done
\end{verbatim}
\normalsize
To verify that the update has correctly worked, you can generate
the ODL schema from the database, as follows:
\verbsize \begin{verbatim}
% eyedbodl -d foo --gencode=ODL

//
// EyeDB Version 2.7.0 Copyright (c) 1995-2005 SYSRA
//
// UNTITLED Schema
//
// Automatically Generated by eyedbodl at Sat Dec  3 19:02:00 2005
//

#if defined(EYEDBNUMVERSION) && EYEDBNUMVERSION != 207000
#error "This file is being compiled with a version of eyedb different from that used to create it (2.7.0)"
#endif

class Address (implementation <hash, hints = "key_count = 2048;">) {
	attribute int32 num;
	attribute string street;
	attribute string town;
	attribute string country;
};

class Person (implementation <hash, hints = "key_count = 2048;">) {
	attribute string firstname;
	attribute string lastname;
	attribute int32 age;
	attribute Address addr;
	relationship Person* spouse inverse Person::spouse;
	attribute set<Person*> children;
};

class Employee (implementation <hash, hints = "key_count = 2048;">) extends Person {
	attribute int64 salary;
};
\end{verbatim}
\normalsize

Note that the exact output may differ a bit from what is displayed above, depending on the \eyedb version.

By default, \texttt{eyedbodl} generates the ODL on the standard output.\\
You see here that the displayed ODL is very similar to the original ODL except
that the keywords \texttt{attribute} and \texttt{relationship} have been added
before each attribute declaration.\\
The \texttt{relationship} keyword means that the attribute has an \texttt{inverse}
directive.
\\
Note that these two keywords are optional: it is why we have not use
them in our example.
\\
\\
Another way to check that the schema has been created within the database,
is to use the \texttt{eyedboql} tool, as follows:
\verbsize \begin{verbatim}
% eyedboql -d foo -c "select schema" --print
= bag(2544.2.881020:oid, 2552.2.291360:oid, 2568.2.164397:oid)
struct Address {2544.2.881020:oid} : struct : agregat : instance : object { 
        attribute int32 num;
        attribute string street;
        attribute string town;
        attribute string country;
};
struct Person {2552.2.291360:oid} : struct : agregat : instance : object { 
        attribute string firstname;
        attribute string lastname;
        attribute int32 age;
        attribute Address addr;
        relationship Person* spouse inverse Person::spouse;
        attribute set<Person*> children;
};
struct Employee {2568.2.164397:oid} : Person : struct : agregat : instance : object { 
        attribute string Person::firstname;
        attribute string Person::lastname;
        attribute int32 Person::age;
        attribute Address Person::addr;
        relationship Person* Person::spouse inverse Person::spouse;
        attribute set<Person*> Person::children;
        attribute int64 salary;
};
\end{verbatim}
\normalsize

Again, note that the exact output may differ a bit from what is displayed above, depending on the \eyedb version.

Note that the object identifiers (\texttt{oid}) of the classes are displayed.
\sect{Data manipulation}
Now the schema has been created in the database, we want to create, update
and query \texttt{Person} and \texttt{Employee} instances.
\subsect{Creating and updating objects}
To create and update objects, there are four ways:\\
\idt the OQL language\\
\idt the C++ binding\\
\idt the Java binding\\
\\
Contrary to the schema creation, all these solutions are quite reasonnable.
\\
The first way is the most attractive because one can create and update objects
in a very intuitive and interprated way using the \texttt{eyedboql} monitor.
\\
\\
We will introduce in the chapter the two first solutions, the OQL language 
and the C++ binding, and just briefly expose the last two solutions.
\subsect{Querying objects}
To query objects, there are only one way: use the OQL language.\\
\\
But note that the OQL language can be used within the C++ and Java binding
in a very well integrated way.
\sect{Manipulating objects using OQL}
Using the \texttt{eyedboql} monitor, we are going to perform the following
operations:\\
\mbox{ } 1- create a person named ``john wayne''\\
\mbox{ } 2- create a person named ``mary poppins''\\
\mbox{ } 3- mary them\\
\mbox{ } 4- create 3 ``john wayne'' children named ``baby1'', ``baby2'' and
``baby3''\\
\mbox{ } 5- perform queries on the name and age\\
\mbox{ } 6- increment the age of each person in the database\\
\\
Here is the way to perform the first three step:
\verbsize \begin{verbatim}
% eyedboql -U bill -P {bill password, i.e. 'thekid'} -d foo -w
Welcome to eyedboql.
  Type `\help' to display the command list.
  Type `\copyright' to display the copyright.
? john := Person(firstname : "john", lastname : "wayne", age : 72);
46382.3.1979263:oid
? mary := Person(firstname : "mary", lastname : "poppins", age : 68);
46386.3.643071:oid
? john.spouse := mary;
46386.3.643071:oid
\end{verbatim}
\normalsize
As you see, this is a very intuitive way to perform such operations.
\\
A few comments:\\
\idt \texttt{?} is the \texttt{eyedboql} prompt: of course, do not type
this string!\\
\idt \texttt{:=} is the affectation operator.\\
\idt each time you create an object, its identifier (\texttt{oid}) is displayed
on your terminal.\\
\idt because of the relationship integrity constraint on the \texttt{spouse}
attribute, the operation \texttt{john.spouse := mary} is equivalent to
\texttt{mary.spouse := john}.\\
\\
To create the three ``john wayne'' children:
\verbsize \begin{verbatim}
? add Person(firstname : "baby1", age : 2) to john->children;
46390.3.3985535:oid
? add Person(firstname : "baby2", age : 3) to john->children;
46394.3.360831:oid
? add Person(firstname : "baby3", age : 4) to john->children;
46398.3.1380607:oid
\end{verbatim}
\normalsize
Once again, this is very intuitive and simple.\\
\\
To query all persons in the database:
\verbsize \begin{verbatim}
? select Person;
= bag(2580.2.955941:oid, 2576.2.881020:oid, 2578.2.821166:oid, 2588.2.382585:oid, 2586.2.500653:oid)
\end{verbatim}
\normalsize
To query all persons whose firstname is ``john'':
\verbsize \begin{verbatim}
? select Person.firstname = "john";
= bag(2576.2.881020:oid)
? \print
Person {2576.2.881020:oid} = { 
        firstname = "john";
        lastname = "wayne";
        age = 72;
        addr Address = { 
                num = NULL;
                street = NULL;
                town = NULL;
                country = NULL;
        };
        *spouse = {2578.2.821166:oid};
        children set<Person*> = set { 
                name = "";
                count = 3;
        };
};
\end{verbatim}
\normalsize
Note that the \texttt{!print} command allows to display the contains of the
last objects returned on your terminal.
\\
To query all persons whose firstname contains a \texttt{y}:
\verbsize \begin{verbatim}
? select Person.firstname ~ "y";
= bag(2580.2.955941:oid, 2578.2.821166:oid, 2588.2.382585:oid, 2586.2.500653:oid)
\end{verbatim}
\normalsize

At this stage, it is interesting to perform the following operation:\\
in another terminal opens an \texttt{eyedboql} session on the database \texttt{foo}
and query all persons, as follows:
\verbsize \begin{verbatim}
% eyedboql -U bill -P {bill password, i.e. 'thekid'} -d foo -w
? select Person;
= bag()
\end{verbatim}
\normalsize
It will perharps surprise you that no person instances will be returned.
\\
But this is quite normal as you have not committed the transaction
in your first \texttt{eyedboql} session.
\\
To commit the transaction, type \texttt{!commit} in your first \texttt{eyedboql}
session:
\verbsize \begin{verbatim}
? \commit
\end{verbatim}
\normalsize
if you query once more the person instances in your second \texttt{eyedboql}
session, the five person instances will be returned.
\sect{Adding indexes}
To introduce the necessity of indexes, we propose to perform the following
operations:
\verbsize \begin{verbatim}
? for (x in 1 <= 2000) Person(firstname : "xx" + string(x));
? select Person.firstname = "xx20";
= bag(2628.2.201654:oid)
? select Person.firstname = "xx10";
= bag(2608.2.642448:oid)
\end{verbatim}
\normalsize
The first operation creates 2000 person instances: as you can notice,
this operation takes only a few seconds on a reasonnable computer.
\\
The two last operations query person instance according to their firstname
attribute.
\\
As you can notice, these query operations take a while: about 3 seconds
for each operation on a ULTRA Sparc: of course, this is too much.
\\
\\
A good idea is to affect an index on the attributes - for instance
\texttt{firstname}, \texttt{lastname} and \texttt{age} - for which one wants
to perform efficient query.
\\
\\
This is very simple:
\be
\item add index specification to the class \texttt{Person}edit within
the \texttt{person.odl} file as follows:
\verbsize \begin{verbatim}
class Person {
  string firstname;
  char lastname;
  int age;
  Address addr;
  ...
  set<Person *> children;

  index on firstname;
  index on lastname;
  index on age;
};
\end{verbatim}
\normalsize
\item then, use the \texttt{eyedbodl} tool to update the database schema:
\verbsize \begin{verbatim}
% eyedbodl -U bill -P {bill password, i.e. 'thekid'} -d foo -u --package=person person.odl
Updating 'person' schema in database foo...
Creating [NULL] hashindex 'index<type = hash, propagate = on> on Person.firstname' on class 'Person'...
Creating [NULL] hashindex 'index<type = hash, propagate = on> on Person.lastname' on class 'Person'...
Creating [NULL] btreeindex 'index<type = btree, propagate = on> on Person.age' on class 'Person'...

Done
\end{verbatim}
\normalsize
\ee
Now, you can try again to query Person instances according to its
firstname, lastname or age:
\verbsize \begin{verbatim}
% eyedboql -U bill -P thekid -d foo -w
? select Person.firstname = "xx20";
= bag(2628.2.201654:oid)
? select Person.firstname = "xx10";
= bag(2608.2.642448:oid)
\end{verbatim}
\normalsize
and you will notice that these operations are immediate.
\\
\\
%Note that an index specification takes a different form according
%to the type for which it is applicable:\\
%\idt \texttt{index} : applicable to the types :
%        \texttt{int, char, byte, float, int[n], byte[n], float[n]} \\
%\idt \texttt{index[]} : applicable to the types : \texttt{char[]} and \texttt{char[n]}
\sect{Adding constraints}
In the same way, you can add a \texttt{notnull} and an \texttt{unique} constraint
on the \texttt{lastname} attribute within the class \texttt{Person}:
\be
\item add the constraint specification to the class \texttt{Person} within
the \texttt{person.odl} file as follows:
\verbsize \begin{verbatim}
class Person {
  string firstname;
  string lastname;
  int age;
  Address addr;
  ...

  index on firstname;
  index on lastname;
  index on age;
  constraint<notnull> on lastname;
  constraint<unique> on lastname;
};
\end{verbatim}
\normalsize
\item then, use the \texttt{eyedbodl} tool to update the database schema:
\verbsize \begin{verbatim}
% eyedbodl -U bill -P thekid -d foo -u --package=person person.odl
Updating 'person' schema in database foo...
Creating [NULL] notnull_constraint 'constraint<notnull, propagate = on> on Person.lastname' on class 'Person'...
Creating [NULL] unique_constraint 'constraint<unique, propagate = on> on Person.lastname' on class 'Person'...

Done
\end{verbatim}
\normalsize
\ee
Now try to create two person instances with the same \texttt{lastname} attribute:
\verbsize \begin{verbatim}
% eyedboql -U bill -P {bill password, i.e. 'thekid'} -d foo -w
? Person(lastname : "curtis");
= 6792.2.881020:oid
? Person(lastname : "curtis");
near line 3: 'Person()' => oql error: new operator <<new<oql\$db> Person()>> : notnull[] constraint error: attribute path 'Person.lastname'.
\end{verbatim}
\normalsize
or with no \texttt{lastname} attribute:
\verbsize \begin{verbatim}
? Person();
near line 3: 'Person()' => oql error: new operator <<new<oql\$db> Person()>> : notnull[] constraint error: attribute path 'Person.lastname'.
\end{verbatim}
\normalsize

\sect{Using the C++ Binding}
We are going to introduce now the C++ binding through the same schema
and examples as previously.
\\
There are two ways to use the C++ binding for such operations: \\
\mbox{ } 1- using the generic C++ binding\\
\mbox{ } 2- using both the generic C++ binding and the generated specific
\texttt{Person} C++ binding.\\
\\
In this chapter, we expose only the second way, as it is far more
simple and pratical than the first one.
\\
\\
To manipulate person instances using the second way, the following steps
are necessary:
\\
\mbox{ } 1- generates the specific \texttt{Person} binding using the \texttt{eyedbodl}
tool\\
\mbox{ } 2- write the C++ client program\\
\mbox{ } 3- compile the generated binding and the client program\\
\\
Note that this example can be found in the directory \$EYEDBROOT/examples/GettingStarted.
\subsect{Generating the specific C++ binding}
To generate the specific C++ binding, do as follows:
\verbsize \begin{verbatim}
% eyedbodl -gencode C++ -package person person.odl
\end{verbatim}
\normalsize
the \texttt{-package} option is mandatory: you may give any name you want.
This name will be used as the generated prefix files.
\\
\texttt{eyedbodl} generates a few files, all prefixed by \texttt{person}.\\
The two main generated files are \texttt{person.h} and \texttt{person.cc}.
\\
If you have a look to the file \texttt{person.h},
you will notice that the following classes have been generated:\\
\mbox{ } 1- the class \texttt{person}\\
\mbox{ } 2- the class \texttt{personDatabase}\\
\mbox{ } 3- the class \texttt{Root}\\
\mbox{ } 4- the class \texttt{Address}\\
\mbox{ } 5- the class \texttt{Person}\\
\mbox{ } 6- the class \texttt{Employee}\\
\\
The first class, \texttt{person}, is the package class:
\verbsize \begin{verbatim}
class person {
 public:
  static void init();
  static void release();
  static idbStatus updateSchema(idbDatabase *db);
  static idbStatus updateSchema(idbSchema *m);
};
\end{verbatim}
\normalsize
it is used to perform package initialization and schema update.\\
Before any use of the \texttt{person} package, you need to call
\texttt{person::init}. \\
\\
The second class, \texttt{personDatabase} is used to open, close and
manipulate objects within a database which contains the \texttt{person} schema.
\\
The \texttt{open} method has two
purposes: the first one is to open the database, as the standard
\texttt{idbDatabase} will do; the second one is to check that the database
schema is consistant with the generated runtime schema.\\
Although it is possible to use the standard \texttt{idbDatabase} class to open
a database containing the \texttt{person} schema, it is strongly recommended
to use the \texttt{personDatabase} class.
\\
\\
The third class, \texttt{Root}, is the root class for all the generated classes.
This class is useful to perform safe down-casting during object loading.
\\
\\
The three last classes, \texttt{Address}, \texttt{Person} and \texttt{Employee}
are generated from the \texttt{person.odl} class specifications: for each
attribute in the \texttt{person.odl}, a set of get and set methods is generated.
\\
For instance, for the \texttt{firstname} attribute, the following methods
are generated:
\verbsize \begin{verbatim}
  idbStatus setFirstname(const char *);
  const char *getFirstname(idbBool *isnull = 0, idbStatus * = 0) const;
  idbStatus setFirstname(unsigned int a0, char);
  char getFirstname(unsigned int a0, idbBool *isnull = 0, idbStatus * = 0)
                    const;
\end{verbatim}
\normalsize
The two first methods manipulate the \texttt{firstname} attribute as a string
while the two last ones manipulate each character within this string.
\\
There are two \texttt{set} methods and two \texttt{get} methods.
\subsect{A minimal client program}
We are now going to write a minimal client program which will performed
the following operations:\\
\mbox{ } 1- initialize the \eyedb package and the \texttt{person} package\\
\mbox{ } 3- open a connection with the \eyedb server\\
\mbox{ } 4- open a database\\
\mbox{ } 5- perform error management\\
\mbox{ } 6- release the \eyedb package and the \texttt{person} package\\
\\
Here is the code for this minimal client:
\verbsize \begin{verbatim}
#include "person.h"

int
main(int argc, char *argv[])
{
  EyeDB::init(argc, argv);      // initializes EyeDB package
  person::init();               // initializes person package

  idbException::setMode(idbException::ExceptionMode); // use exception mode

  try {
    idbConnection conn;
    conn.open();                // opens the connection
    personDatabase db(argv[1]); // creates a database handle
    db.open(&conn, idbDatabase::DBRW); // opens the database in read/write mode
  }

  catch(idbException &e) {      // catch any exception and print it
    e.print();
  }

  person::release();            // releases person package
  EyeDB::release();             // releases EyeDB package

  return 0;
}
\end{verbatim}
\normalsize
Note that statement \texttt{idbException::setMode(...)} is mandatory
if you want to use the exception error policy.
\\
To use this client:\\
\mbox{ } 1- compile it using a standard Makefile, as follows:
\verbsize \begin{verbatim}
include $(EYEDBROOT)/$(EYEDBARCH)/etc/Makefile.eyedb

all: persontest

persontest: persontest.o person.o
          $(CXX) -o $@ $+ $(LDXXFLAGS)
\end{verbatim}
\normalsize
Important note: you need a recent version of GNU make to use this makefile.
This makefile does not work with the standard SUN make.
\mbox{ } 2- execute the program as follows:
\verbsize \begin{verbatim}
% ./persontest foo
\end{verbatim}
\normalsize
If you try to open a database which does not contain the \texttt{person} schema,
some error messages will be displayed, for instance:
\verbsize \begin{verbatim}
% ./persontest EYEDBDM
eyedb error: 'Address' class does not exist
'Person' class does not exist
'Employee' class does not exist
\end{verbatim}
\normalsize
We are going now to add a function to manipulate \texttt{Person} instances:\\
\mbox{ } 1- create a person named ``john wayne''\\
\mbox{ } 2- create a person named ``mary poppins''\\
\mbox{ } 3- mary them\\
\mbox{ } 4- create 3 ``john wayne'' children named ``baby1'', ``baby2'' and
``baby3''\\
\\
These operations are very simple as you will see here:
\verbsize \begin{verbatim}
static void
create(idbDatabase *db)
{
  db->transactionBegin(); // starts a new transaction

  Person *john = new Person(db);
  john->setFirstname("john");
  john->setLastname("wayne");
  john->setAge(32);
  john->getAddr()->setStreet("courcelles");
  john->getAddr()->setTown("Paris");

  Person *mary = new Person(db);
  mary->setFirstname("mary");
  mary->setLastname("poppins");
  mary->setAge(30);
  mary->getAddr()->setStreet("courcelles");
  mary->getAddr()->setTown("Paris");

  // mary them
  john->setSpouse(mary);

  // creates children
  for (int i = 0; i < 5; i++)
    {
      char name[32];
      sprintf(name, "baby%d", i+1);
      Person *child = new Person(db);
      child->setFirstname(name);
      child->setLastname(name);
      child->setAge(1+i);
      john->addToChildrenColl(child);
      child->release(); // release the allocated pointer
    }

  // store john and all its related instances within the database
  john->store(idbFullRecurs);

  // release the allocated pointers
  mary->release();
  john->release();

  db->transactionCommit(); // commits the current transaction
}
\end{verbatim}
\normalsize
A few remarks about this code:
\be
\item all operations - setting, getting attributes, storing, querying instances
in a database - must be performed within a transaction. \\
A transaction is initiated using the \texttt{idbDatabase::transactionBegin}
method and is committed (resp. aborted) using the
\texttt{idbDatabase::transactionCommit}
(resp. \texttt{idbDatabase::transactionAbort}) method.
\item to store any instance in the database, you need to call the 
emph{store} (or \texttt{realize}) method on this instance.\\
In our case, we use the argument \texttt{idbFullRecurs} indicating that
we want all related instances (through relationship or indirect
attribute) to be stored in the database.
\item all runtime pointers allocated with the \texttt{new} operator must
be deleted using the \texttt{release} method. The \texttt{delete} operator
is forbidden: if you try to use it, an exception will be thrown at
runtime.
\ee
We are now going to query and display all the person instances 
whose \texttt{lastname} attribute matches a regular expression \texttt{s}.
\\
\\
Here is the code for such an operation:
\verbsize \begin{verbatim}
static void
read(idbDatabase *db, const char *s)
{
  db->transactionBegin();

  idbOQL q(db, "select Person.lastname ~ \"%s\"", s);

  idbObjectArray obj_arr;
  q.execute(obj_arr);

  for (int i = 0; i < obj_arr.getCount(); i++)
    {
      Person *p = Person_c(obj_arr[i]);
      if (p)
        printf("person = %s %s, age = %d\n", p->getFirstname(),
               p->getLastname(), p->getAge());
    }

  db->transactionCommit();
}
\end{verbatim}
\normalsize
A few remarks about this code:
\be
\item an OQL construct can be used within the C++ code using the
\texttt{idbOQL(idbDatabase *, const char *fmt, ...)} constructor.
\\
Assuming \texttt{s} is equal to \texttt{baby}, the code:
\verbsize \begin{verbatim}
  idbOQL q(db, "select Person.lastname ~ \"%s\"", s);
\end{verbatim}
\normalsize
will send the query \texttt{select Person.lastname ~ "baby"} to the OQL interpreter.\\
This interpreter will perform the query and returned all the found objects.
The returned objects can be found using the the \texttt{idbOQL::execute} method
as follows:
\verbsize \begin{verbatim}
  idbObjectArray obj_arr;
  q.execute(obj_arr);
\end{verbatim}
\normalsize
\item the returned objects are of type \texttt{idbObject}, so you cannot use the
\texttt{Person} methods such as \texttt{getFirstname()}, \texttt{getAge()} and
so on.\\
To use them, you need to perform a down-cast using the \texttt{Person\_c} static
function as follows:
\verbsize \begin{verbatim}
  for (int i = 0; i < obj_arr.getCount(); i++)
    {
      Person *p = Person_c(obj_arr[i]);
      if (p) ...
\end{verbatim}
\normalsize
If the object \texttt{obj\_arr[i]} is not of \texttt{Person} type, the returned
pointer will be null. It is why we make a test on the value of \texttt{p}.\\
If \texttt{p} is not null, we can use all the \texttt{Person} methods as follows:
\verbsize \begin{verbatim}
    printf("person = %s %s, age = %d\n", p->getFirstname(),
           p->getLastname(), p->getAge());
\end{verbatim}
\normalsize
\item note that you can use the \texttt{trace} method on any \texttt{idbObject}
instance to display all its attributes and relationships:
\verbsize \begin{verbatim}
  for (int i = 0; i < obj_arr.getCount(); i++)
    {
      Person *p = Person_c(obj_arr[i]);
      if (p)
        p->trace();
      else
        obj_arr[i]->trace();
    }
\end{verbatim}
\normalsize
note that the two calls: \texttt{p->trace()} and \texttt{obj\_arr[i]->trace()}
give the same output if the object is of \texttt{Person} type.
\ee
To have more information about the C++ binding, please refer to the
appriopriate chapter.

\sect{Using the Java Binding}
Although the C++ binding is more complete than the Java binding
- essentially according to the administrative operations - the Java
bindings allow to manipulate data without limitations.
For instance, to use the Java binding, subsitute the word \texttt{Java}
to the word \texttt{C++} in the \texttt{eyedbodl} use, as follows:
\verbsize \begin{verbatim}
% eyedbodl -gencode Java -package person person.odl
\end{verbatim}
\normalsize
the following files will be generated:\\
\mbox{ } 1- Database.java\\
\mbox{ } 2- Root.java\\
\mbox{ } 3- Address.java\\
\mbox{ } 4- Person.java\\
\mbox{ } 5- Employee.java\\
\\
Each generated file contains a Java class of the same name.
\\
\\
In the same way than for the C++ binding, you can write a Java client and
compile it with the generated Java binding.
The Java layer support both the standalone applications and the applets.
\\
\\
Here is a simple example which show how to:\\
\mbox{ } 1- initialize the \eyedb and \texttt{person} packages\\
\mbox{ } 2- connect to the \eyedb server\\
\mbox{ } 3- open a database\\
\mbox{ } 4- creates two person instances and mary them
\verbsize \begin{verbatim}
//
// persontest.java
//

import COM.sysra.eyedb.*;
import person.*;

class Test {
  public static void main(String args[]) {

    // Initialize the eyedb package and parse the default eyedb options
    // on the command line
    String[] outargs = idbRoot.init("Test", args);
     
    // Check that a database name is given on the command line
    int argc = outargs.length;
    if (argc != 1)
      {
        System.err.println("usage: java Test dbname");
        System.exit(1);
      }

    try {
      // Initialize the person package
      person.Database.init();

      // Open the connection with the backend
      idbConnection conn = new idbConnection();

      // Open the database named outargs[0]
      person.Database db = new person.Database(outargs[0]);
      db.open(conn, idbDatabase.DBRW);

      db.transactionBegin();
      // Create two persons john and mary
      Person john = new Person(db);
      john.setFirstname("john");
      john.setAge(26);
     
      Person mary = new Person(db);
      mary.setFirstname("mary");
      mary.setAge(22);
     
      // Mary them :-)
      john.setSpouse(mary);

      // Store john and mary in the database
      john.store(idbRecMode.FullRecurs);

      db.transactionCommit();
    }
    catch(idbException e) { // Catch any eyedb exception
       e.print();
       System.exit(1);
    }
  }
}
\end{verbatim}
\normalsize

\sect{Learning more about \eyedb}
We have briefly introduce in this chapter some of the main features
of \eyedbX.
\\
\\
The next chapters describes the features of \eyedb in more detail.


%\texttt{eyedbadmin} \> \texttt{eyedbdbcreate} \> \texttt{eyedbuseradd} \> \texttt{eyedbdbdelete}\\
%\texttt{eyedbodl} \> \texttt{eyedbuserdbaccess} \> \texttt{eyedbchklicense} \> \texttt{eyedbdbexport}\\
%\texttt{eyedboql} \> \texttt{eyedbuserdelete} \> \texttt{eyedbcrypt} \> \texttt{eyedbdbimport}\\
%\texttt{eyedbuserlist} \> \texttt{eyedbd} \> \texttt{eyedbdblist} \> \texttt{eyedbpasswd} \\
%\texttt{eyedbuserpasswd} \> \texttt{eyedbd}-gcc \> \texttt{eyedbdbmcreate} \> \texttt{eyedbdbaccess}\\
%\texttt{eyedbdbmove} \> \texttt{eyedbwwwd} \> \texttt{eyedbdbadmin} \> \texttt{eyedbdbrename}\\
%\texttt{eyedbrc} \> \texttt{eyedbdbcopy} \> \texttt{eyedbgetenv} \> \texttt{eyedbsysaccess}  \\


\end{document}
